%% appendix.tex -- Appendix to the Diplomarbeit.
%%
%% Copyright (C) 2003 Martin Grabmueller <mgrabmue@cs.tu-berlin.de>


\begin{appendix}

%%%%%%%%%%%%%%%%%%%%%%%%%%%%%%%%%%%%%%%%%%%%%%%%%%%%%%%%%%%%%%%%%%%%%%

\newenvironment{grammar}%
  {\newcommand\produces[2]{##1 \> $\rightarrow$ \> ##2 \\}
   \newcommand\orproduces[1]{\> \> \makebox[0pt][r]{$|$ }##1 \\}
   \newcommand\res[1]{'{\bf ##1}'}
%   \newcommand\res[1]{\underline{\bf ##1}}
   \newcommand\emptyprod{{$\varepsilon$}}
   \newcommand\heading[1]{\rule{\linewidth}{1pt} \\{\bf ##1}\\[2ex]}
   \newcommand\separator{\rule{\linewidth}{1pt} \\}
   \begin{tabbing}
   \qquad\qquad\qquad\qquad\qquad \= \qquad \= \kill}
  {\end{tabbing}}

%%%%%%%%%%%%%%%%%%%%%%%%%%%%%%%%%%%%%%%%%%%%%%%%%%%%%%%%%%%%%%%%%%%%%%

\chapter{\turtle{} Grammar}
\label{cha:turtle-grammar}
\index{Turtle grammar}
\index{grammar!Turtle}
\index{syntax definition}

The lexical and grammatical structure of Turtle is described using
Extended Backus Naur Form (EBNF).  This notation is summarized below.
\index{EBNF}
\index{Extended Backus Naur Form}

\vskip0.5em
\noindent
\begin{tabular}{ll}
$A \rightarrow E$&
The non-terminal $A$ produces the form $E$.\\

$E | F$&
Alternative; either $E$ or $F$ is produced.\\

$\{ E \}$&
The form $E$ is repeated, possibly zero times.\\

$[E]$&
The form $E$ is optional, it may be omitted completely.\\

$(E)$&
For grouping, denotes $E$ itself.\\

$\varepsilon$& The empty string.\\
\end{tabular}
\vskip0.5em

\noindent
Nonterminals are printed in normal roman font, terminals appear bold
and in single quotes.

\section{Lexical Entities}
\index{lexical entities}
\index{token}

The lexical entities (also called {\em tokens}) represent the regular
part of the \turtle{} grammar.  They are recognized by the scanner
module and then processed by the parser, which recognizes the
context-free part of the syntax.  In addition to the token classes
defined below, there is a number of one- or two-character symbols.  We
do not define names for them in order to keep the grammar for the
context-free syntax readable.

\begin{grammar}
\produces{IntConst}{Digit \{Digit\}}
\produces{LongConst}{Digit \{Digit\} \res{L}}
\produces{RealConst}{Digit \{Digit\} \res{.} Digit \{Digit\} [ Exponent ]}
\produces{Exponent}{( \res{e} $|$ \res{E} )[ \res{$+$} $|$ \res{$-$} ] Digit \{ Digit \}}
\produces{StringConst}{\res{\dq} \{ StringChar \} \res{\dq}}
\produces{CharConst}{\res{'} StringChar \res{'}}
\produces{Ident}{( Letter $|$ \res{\_} ) \{ Letter $|$ Digit $|$ \res{\_} $|$ \res{?} $|$ \res{!} \}}
\produces{Letter}{\res{A} $|$ \dots $|$ \res{Z} $|$ \res{a} $|$ \dots $|$ 
\res{z}}
\produces{Digit}{\res{0} $|$ \dots $|$ \res{9}}

\produces{StringChar}{ {\em (any except \res{\,$\backslash$}, \res{\,\dq} and newline)} $|$ EscapeChar }
\produces{EscapeChar}{ \res{$\backslash\backslash$} $|$
 \res{$\backslash$\dq} $|$
 \res{$\backslash$'} $|$
 \res{$\backslash$n} $|$
 \res{$\backslash$r} $|$
 \res{$\backslash$t} $|$ 
 \res{$\backslash$b} $|$
 \res{$\backslash$v} $|$
 \res{$\backslash$a} $|$
 \res{$\backslash$f\,} }
\produces{Comment}{\res{//} \{ {\em (any except newline)} \}}
\orproduces{\res{/*} \{ {\em (any except '*/') \} \res{\,*/}}}
\end{grammar}

\index{whitespace}
Spaces, tab characters, newlines and comments serve as separators
between tokens and are otherwise ignored.

\index{string literals}
String literals are not allowed to cross line boundaries, so a literal
newline character inside a string is an error.

\index{reserved words}
Some of the tokens in the token class {\em Ident} are reserved and
have a special meaning in the \turtle{} syntax.  They may not be used
except where allowed in the context-free syntax.  The reserved words
in \turtle{} are:
%
\vskip1ex
\begin{tabular}{llllll}
and&
array&
const&
constraint&
datatype&
do\\
else&
elsif&
end&
export&
false&
foreign\\
fun&
hd&
if&
import&
in&
list\\
module&
not&
null&
of&
or&
public\\
require&
return&
sizeof&
string&
then&
tl\\
true&
type&
var&
while&&\\
\end{tabular}
\vskip1ex
\noindent
The following is a list of the non-alphanumeric symbols:
\index{non-alphanumeric symbols}
\vskip1em
\begin{tabular}{llllllllllll}
.&
,&
(&
)&
[&
]&
\{&
\}&
;&
!&
$+$&
$-$\\
*&
/&
\%&
$=$&
$<>$&
$<$&
$<=$&
$>$&
$>=$&
:&
::&
:=
\end{tabular}
\vskip1em

\section{Context-free Syntax}
\index{context-free syntax}
This is the context-free part of the \turtle{} grammar.

%\enlargethispage{-1.5cm}
\begin{grammar}
\heading{Module Definition\index{module definition syntax}}

\produces{CompilationUnit}{Module}

\produces{Module}{\res{module} QualIdent [ FormalTypeParams ] \res{;} \\\>\>ModDecls Declarations}

\produces{ModDecls}{[ ModImports ] [ ModExports ]}

\produces{ModImports}{\res{import} ImportIdent \{ \res{,} ImportIdent \} \res{;}}

\produces{ModExports}{\res{export} QualIdent \{ \res{,} QualIdent \} \res{;}}

\produces{ImportIdent}{QualIdent [ TypeParams ] [ OpenIdentList ] }
\produces{OpenIdentList}{\res{(} Ident \{\res{,} Ident\} \res{)}}

\heading{Declarations\index{declaration syntax}}

\produces{Declarations}{\{ Declaration \res{;} \}}

\produces{Declaration}{TypeDecl $|$ DatatypeDecl $|$ VarDecl $|$ ConstDecl}
\orproduces{FunDecl $|$ ConstraintDecl}

\produces{TypeDecl}{[ \res{public} ] \res{type} Ident \res{=} Type}

\produces{DatatypeDecl}{[ \res{public} ] \res{datatype} Ident [ FormalTypeParams ] \res{=} \\\>\> DatatypeVariant \{ \res{or} DatatypeVariant \}}

\produces{DatatypeVariant}{Ident [\res{(} FieldList \res{)}]}

\produces{FieldList}{Field \{\res{,} Field\}}
\produces{Field}{Ident \res{:} Type}

\produces{VarDecl}{[ \res{public} ] \res{var} VariableList}
\produces{ConstDecl}{[ \res{public} ] \res{const} VariableList}

\produces{VariableList}{Variable \{\res{,} Variable\}}
\produces{Variable}{Ident \res{:} Type [\res{:=} ConsExpression]}

\produces{FunDecl}{[ \res{public} ] \res{fun} Ident ParamList [\res{:} Type] SubrBody}

\produces{ConstraintDecl}{[ \res{public} ] \res{constraint} Ident ParamList SubrBody}
\produces{ParamList}{\res{(} [ Parameter \{\res{,} Parameter\} ] \res{)}}
\produces{Parameter}{Ident \res{:} Type}

\produces{SubrBody}{StmtList \res{end}}

\heading{Type expressions\index{type expression syntax}}

\produces{Type}{QualIdent [ TypeParams ]}
\orproduces{\res{!} Type $|$  \res{()} $|$ \res{(} Type \{\res{,} Type\} \res{)}}
\orproduces{\res{array} \res{of\,} Type
                 $|$  \res{list} \res{of\,} Type
                 $|$  \res{string}}
\orproduces{\res{fun} \res{(} [Type \{\res{,} Type\}] \res{)} [\res{:} Type]}

\heading{Statements\index{statement syntax}}

\produces{StmtList}{\{ Stmt \res{;} \}}

\produces{Stmt}{VarDecl $|$ ConstDecl $|$ FunDecl $|$ ConstraintDecl}
\orproduces{IfStmt $|$ WhileStmt $|$ ReturnStmt $|$ InStmt}
\orproduces{RequireStmt $|$ ExpressionStmt}

\produces{IfStmt}{\res{if\,} CompareExpr \res{then} StmtList \\\>\>\{ \res{elsif\,} CompareExpr \res{then} StmtList \} \\\>\>[\res{else} StmtList] \res{end}}

\produces{WhileStmt}{\res{while} CompareExpr \res{do} StmtList \res{end}}

\produces{InStmt}{\res{in} StmtList \res{end}}

\produces{ReturnStmt}{\res{return} [TupleExpr]}

\produces{ExpressionStmt}{Expression}

\heading{Constraints\index{constraint syntax}}

\produces{RequireStmt}{\res{require} ConstraintConj [ InStmt ]}
\produces{ConstraintConj}{Constraint [ \res{:} Strength ]\\\>\>\{ \res{and} Constraint [ \res{:} Strength ] \}}
\produces{Constraint}{CompareExpression}
\produces{Strength}{IntConst}

\heading{Expressions\index{expression syntax}}

\produces{Expression}{AssignExpr}

\produces{AssignExpr}{TupleExpr [ \res{:=} TupleExpr ]}

\produces{TupleExpr}{ConsExpr \{ \res{,} ConsExpr \}}

\produces{ConsExpr}{OrExpr [\res{::} ConsExpr]}

\produces{OrExpr}{AndExpr \{ \res{or} AndExpr \}}

\produces{AndExpr}{CompareExpr \{ \res{and} CompareExpr \}}

\produces{CompareExpr}{AddExpr [CompareOp AddExpr]}

\produces{AddExpr}{MulExpr \{ AddOp MulExpr \}}

\produces{MulExpr}{Factor \{ MulOp Factor \}}

\produces{Factor}{SimpleExpr $|$ UnOp Factor}

\produces{SimpleExpr}{AtomicExpr \{ ActualParameters $|$ Index \}}

\produces{ActualParameters}{\res{(} [ ConsExpr \{\res{,} ConsExpr \}] \res{)}}
\produces{Index}{\res{[} AddExpr \res{]}}

\produces{AtomicExpr}{QualIdent $|$ IntConst $|$ LongConst $|$ RealConst}
\orproduces{StringConst $|$ CharConst $|$ BoolConst}
\orproduces{ArrayExpr $|$ ListExpr $|$ \res{null}}
\orproduces{FunExpr $|$ ConstraintExpr}
\orproduces{\res{array} AddExpr \res{of\,} TupleExpr}
\orproduces{\res{list} AddExpr \res{of\,} TupleExpr}
\orproduces{\res{string} AddExpr \res{of\,} SimpleExpr}
\orproduces{\res{(} TupleExpr \res{)}}
\orproduces{\res{var} AtomicExpr $|$ \res{!} AtomicExpr}
\orproduces{\res{foreign} StringConst}

\produces{BoolConst}{\res{false} $|$ \res{true}}

\produces{FunExpr}{\res{fun} ParamList \res{:} Type SubrBody}
\produces{ConstraintExpr}{\res{constraint} ParamList SubrBody}

\produces{ArrayExpr}{\res{\{} ConsExpr \{\res{,} ConsExpr\} \res{\}}}

\produces{ListExpr}{\res{[} ConsExpr \{\res{,} ConsExpr\} \res{]}}

\heading{Operators\index{operators}}

\produces{CompareOp}{\res{$=$} $|$ \res{$<>$} $|$ \res{$<$} $|$ \res{$<=$}
$|$ \res{$>$} $|$ \res{$>=$}}

\produces{AddOp}{\res{$+$} $|$ \res{$-$}}

\produces{MulOp}{\res{$*$} $|$ \res{$/$} $|$ \res{\%}}

\produces{UnOp}{\res{$-$} $|$ \res{not} $|$ \res{hd} $|$ \res{tl} $|$ \res{sizeof\,}}

\heading{Miscellaneous}

\produces{QualIdent}{Ident \{ \res{.} Ident \}}
\produces{TypeParams}{\res{$<$} Type \{\res{,} Type\} \res{$>$}}
\produces{FormalTypeParams}{\res{$<$} Ident \{\res{,} Ident\} \res{$>$}}

\end{grammar}

\noindent
Note that some symbols have been replaced by more readable symbols in
the other chapters of this thesis (e.g., {\bf :=} has been replaced by
$\leftarrow$).  The compiler only accepts the symbols defined in this
appendix.

\chapter{Example modules}
\label{cha:example-modules}

The program listings in this appendix will serve as examples on how
complete programs written in Turtle look like.  The different examples
were chosen to demonstrate how Turtle can be used to program in
different programming paradigms.

The first is a module for list sorting, taken from the Turtle standard
library.  Because of its size, the module is split up into
programs~\ref{prog:example-module1} and~\ref{prog:example-module2}.
The module is written in pure functional style.  In lines 1 to 3 we
can see the module header, which consists of the {\bf module}
declaration, stating the name of the module and the name of the module
parameter $\alpha$.  Following, we see that the module imports a
module from the library called {\em lists}, which is instantiated with
the module parameter.  Only one name is exported, the sorting function
{\em sort}.  Line 4 contains the function header, which lists the
parameters and return types of the function.  Note that all types in
this example depend on the module parameter $\alpha$.

\index{quicksort example}
\index{examples!quicksort}
\begin{Program}
% module listsort<A>;
% import lists<A>;
% export sort;
% fun sort (a: list of A, cmp: fun (A, A): int): list of A
%   fun smaller (elem1: A): fun (A): bool
%     return fun (elem2: A): bool
%              return cmp (elem1, elem2) > 0;
%            end;
%   end;
%   fun greatereq (elem1: A): fun (A): bool
%     return fun (elem2: A): bool
%              return cmp (elem1, elem2) <= 0;
%            end;
%   end;
%   fun sort (a: list of A): list of A
%     if a = null then
%       return null;
%     else
%       if tl a = null then
%         return a;
%       else
%         if tl tl a = null then
%           if cmp (hd a, hd tl a) > 0 then
%             return hd tl a :: hd a :: null;
%           else
%             return a;
%           end;
%         else
%           var l1: list of A := lists.filter (smaller (hd a), tl a);
%           var l2: list of A := lists.filter (greatereq (hd a), tl a);
%           return lists.append (sort (l1), lists.append (hd a :: null, sort (l2)));
%         end;
%       end;
%     end;
%   end;
%   return sort (a);
% end;
\begin{ttlprog}
1\>\ttlModule{} listsort$<$$\alpha$$>$;\\
2\>\ttlImport{} lists$<$$\alpha$$>$;\\
3\>\ttlExport{} sort;\\
4\>\ttlFun{} sort (a: \ttlList{} \ttlOf{} $\alpha$, cmp: \ttlFun{} ($\alpha$, $\alpha$): int): \ttlList{} \ttlOf{} $\alpha$\\
5\>\>\ttlFun{} smaller (elem1: $\alpha$): \ttlFun{} ($\alpha$): bool\\
6\>\>\>\ttlReturn{} \ttlFun{} (elem2: $\alpha$): bool\\
7\>\>\>\>\>\>\> \ttlReturn{} cmp (elem1, elem2) $>$ 0;\\
8\>\>\>\>\>\> \ttlEnd{};\\
9\>\>\ttlEnd{};\\
10\>\>\ttlFun{} greatereq (elem1: $\alpha$): \ttlFun{} ($\alpha$): bool\\
11\>\>\>\ttlReturn{} \ttlFun{} (elem2: $\alpha$): bool\\
12\>\>\>\>\>\>\> \ttlReturn{} cmp (elem1, elem2) $\leq$ 0;\\
13\>\>\>\>\>\> \ttlEnd{};\\
14\>\>\ttlEnd{};
\end{ttlprog}
\caption{Quicksort on lists}
\label{prog:example-module1}
\end{Program}

Below, in lines 5 to 14, the internal helper functions {\em smaller}
and {\em greatereq} are defined.  These are higher-order functions
which return functions for determining whether their argument {\em
  elem2} is smaller or greater or equal to the outer argument {\em
  elem1}, respectively.

The longest part of the functions, from line 15 to line 35, consists
of the recursive sorting procedure, which implements a naive version
of Quicksort: empty or one-element lists are already sorted, and the
elements of lists of length two are swapped if they are not already in
order.  All lists longer than two elements are split into three parts:
the pivot element, which is the first element of the list for
simplicity, a list of all elements smaller than the pivot and a list
of all elements greater or equal to the pivot.  The two lists are
created using the {\em filter} function from module {\em lists}, are
sorted recursively and are finally appended to form the resulting
sorted list.

\begin{Program}
\begin{ttlprog}
15\>\>\ttlFun{} sort (a: \ttlList{} \ttlOf{} $\alpha$): \ttlList{} \ttlOf{} $\alpha$\\
16\>\>\>\ttlIf{} a = \ttlNull{} \ttlThen{}\\
17\>\>\>\>\ttlReturn{} \ttlNull{};\\
18\>\>\>\ttlElse{}\\
19\>\>\>\>\ttlIf{} \ttlTl{} a = \ttlNull{} \ttlThen{}\\
20\>\>\>\>\>\ttlReturn{} a;\\
21\>\>\>\>\ttlElse{}\\
22\>\>\>\>\>\ttlIf{} \ttlTl{} \ttlTl{} a = \ttlNull{} \ttlThen{}\\
23\>\>\>\>\>\>\ttlIf{} cmp (\ttlHd{} a, \ttlHd{} \ttlTl{} a) $>$ 0 \ttlThen{}\\
24\>\>\>\>\>\>\>\ttlReturn{} \ttlHd{} \ttlTl{} a :: \ttlHd{} a :: \ttlNull{};\\
25\>\>\>\>\>\>\ttlElse{}\\
26\>\>\>\>\>\>\>\ttlReturn{} a;\\
27\>\>\>\>\>\>\ttlEnd{};\\
28\>\>\>\>\>\ttlElse{}\\
29\>\>\>\>\>\>\ttlVar{} l1: \ttlList{} \ttlOf{} $\alpha$ $\leftarrow$ lists.filter (smaller (\ttlHd{} a), \ttlTl{} a);\\
30\>\>\>\>\>\>\ttlVar{} l2: \ttlList{} \ttlOf{} $\alpha$ $\leftarrow$ lists.filter (greatereq (\ttlHd{} a), \ttlTl{} a);\\
31\>\>\>\>\>\>\ttlReturn{} lists.append (sort (l1), lists.append (\ttlHd{} a :: \ttlNull{}, sort (l2)));\\
32\>\>\>\>\>\ttlEnd{};\\
33\>\>\>\>\ttlEnd{};\\
34\>\>\>\ttlEnd{};\\
35\>\>\ttlEnd{};\\
36\>\>\ttlReturn{} sort (a);\\
37\>\ttlEnd{};
\end{ttlprog}
\caption{Quicksort on lists (continued)}
\label{prog:example-module2}
\end{Program}

The next example will illustrate how imperative programming is done in
Turtle.  The program was adapted from the example program {\tt
  QUEENS.TIG} in~\cite{appel98moderncompiler} and solves the
$N$-queens problem, which is how to place $N$ queens on a chess board
of size $N\times N$ without any two queens attacking each other.

The first function {\em print\_board} simply prints the current board
configuration.  More interesting is the function {\em try}, which
tries to place a queen in column $c$.  If it succeeds, it places the
queen and calls itself recursively to place a queen in column $c + 1$,
and it backtracks on failure.  By trying all rows for each column, all
possibilities are eventually tried and all solutions are found.

\index{n-queens example}
\index{examples!n-queens}
% module queens;
% import io;
% var row: array of int, col: array of int;
% var diag1: array of int, diag2: array of int;
% var size: int;
% var solutions: int;
% fun print_board ()
%   var i: int, j: int;
%   i := 0;
%   while i < size do
%     j := 0;
%     while j < size do
%       if col[i] = j then
%       io.put (" o");
%       else
%       io.put (" .");
%       end;
%       j := j + 1;
%     end;
%     io.nl ();
%     i := i + 1;
%   end;
%   io.nl ();
% end;
% fun try (c: int)
%   if c = size then
%     print_board ();
%     solutions := solutions + 1;
%   else
%     var r: int;
%     r := 0;
%     while r < size do
%       if (row[r] = 0) and (diag1[r + c] = 0) and 
%       (diag2[r + (size - 1) - c] = 0) 
%       then
%       row[r] := 1;
%       diag1[r + c] := 1;
%       diag2[r + (size - 1) - c] := 1;
%       col[c] := r;
%       try (c + 1);
%       row[r] := 0;
%       diag1[r + c] := 0;
%       diag2[r + (size - 1) - c] := 0;
%       end;
%       r := r + 1;
%     end;
%   end;
% end;
% fun main (argv: list of string): int
%   size := 8;
%   solutions := 0;
%   row := array size of 0;
%   col := array size of 0;
%   diag1 := array size * 2 - 1 of 0;
%   diag2 := array size * 2 - 1 of 0;
%   try (0);
%   io.put (solutions);
%   io.put (" solutions found.\n");
%   return 0;
% end;
\begin{Program}
\begin{ttlprog}
1\>\ttlModule{} queens;\\
2\>\ttlImport{} io;\\
3\>\ttlVar{} row: \ttlArray{} \ttlOf{} int, col: \ttlArray{} \ttlOf{} int;\\
4\>\ttlVar{} diag1: \ttlArray{} \ttlOf{} int, diag2: \ttlArray{} \ttlOf{} int;\\
5\>\ttlVar{} size: int;\\
6\>\ttlVar{} solutions: int;\\
7\>\ttlFun{} print\_board ()\\
8\>\>\ttlVar{} i: int, j: int;\\
9\>\>i $\leftarrow$ 0;\\
10\>\>\ttlWhile{} i $<$ size \ttlDo{}\\
11\>\>\>j $\leftarrow$ 0;\\
12\>\>\>\ttlWhile{} j $<$ size \ttlDo{}\\
13\>\>\>\>\ttlIf{} col[i] = j \ttlThen{}\\
14\>\>\>\>\>io.put ("o");\\
15\>\>\>\>\ttlElse{}\\
16\>\>\>\>\>io.put (".");\\
17\>\>\>\>\ttlEnd{};\\
18\>\>\>\>j $\leftarrow$ j $+$ 1;\\
19\>\>\>\ttlEnd{};\\
20\>\>\>io.nl ();\\
21\>\>\>i $\leftarrow$ i $+$ 1;\\
22\>\>\ttlEnd{};\\
23\>\>io.nl ();\\
24\>\ttlEnd{};
\end{ttlprog}
\caption{N-queens problem}
\label{prog:example-queens1}
\end{Program}

\begin{Program}
\begin{ttlprog}
25\>\ttlFun{} try (c: int)\\
26\>\>\ttlIf{} c = size \ttlThen{}\\
27\>\>\>print\_board ();\\
28\>\>\>solutions $\leftarrow$ solutions $+$ 1;\\
29\>\>\ttlElse{}\\
30\>\>\>\ttlVar{} r: int;\\
31\>\>\>r $\leftarrow$ 0;\\
32\>\>\>\ttlWhile{} r $<$ size \ttlDo{}\\
33\>\>\>\>\ttlIf{} (row[r] = 0) \ttlAnd{} (diag1[r $+$ c] = 0) \ttlAnd{} \\
34\>\>\>\>\>(diag2[r $+$ (size $-$ 1) $-$ c] = 0) \\
35\>\>\>\>\ttlThen{}\\
36\>\>\>\>\>row[r] $\leftarrow$ 1;\\
37\>\>\>\>\>diag1[r $+$ c] $\leftarrow$ 1;\\
38\>\>\>\>\>diag2[r $+$ (size $-$ 1) $-$ c] $\leftarrow$ 1;\\
39\>\>\>\>\>col[c] $\leftarrow$ r;\\
40\>\>\>\>\>try (c $+$ 1);\\
41\>\>\>\>\>row[r] $\leftarrow$ 0;\\
42\>\>\>\>\>diag1[r $+$ c] $\leftarrow$ 0;\\
43\>\>\>\>\>diag2[r $+$ (size $-$ 1) $-$ c] $\leftarrow$ 0;\\
44\>\>\>\>\ttlEnd{};\\
45\>\>\>\>r $\leftarrow$ r $+$ 1;\\
46\>\>\>\ttlEnd{};\\
47\>\>\ttlEnd{};\\
48\>\ttlEnd{};\\
49\>\ttlFun{} main (argv: \ttlList{} \ttlOf{} \ttlString{}): int\\
50\>\>size $\leftarrow$ 8;\\
51\>\>solutions $\leftarrow$ 0;\\
52\>\>row $\leftarrow$ \ttlArray{} size \ttlOf{} 0;\\
53\>\>col $\leftarrow$ \ttlArray{} size \ttlOf{} 0;\\
54\>\>diag1 $\leftarrow$ \ttlArray{} size * 2 $-$ 1 \ttlOf{} 0;\\
55\>\>diag2 $\leftarrow$ \ttlArray{} size * 2 $-$ 1 \ttlOf{} 0;\\
56\>\>try (0);\\
57\>\>io.put (solutions);\\
58\>\>io.put ("solutions found.$\backslash$n");\\
59\>\>\ttlReturn{} 0;\\
60\>\ttlEnd{};\\
\end{ttlprog}
\caption{N-queens problem (continued)}
\label{prog:example-queens2}
\end{Program}

\newpage
The last example program shows how constraint imperative programs are
written in \turtle{}.  Program~\ref{prog:send-more-money} solves the
famous crypto-arithmetic puzzle:
\vskip1ex
\begin{tabular}{rrrrr}
&S&E&N&D\\
+&M&O&R&E\\
\hline
M&O&N&E&Y
\end{tabular}
\vskip1ex The task is to assign a number between 0 and 9 to each
letter, so that the sum of the first two lines equals the last line,
under the constraint that all letters must get assigned different
numbers.

The example program first defines the two user-defined constraints
{\em all\_different} and {\em domain}.  The former constrains a list
of constrainable variables to be pairwise distinct and the latter
constrains a variable to lie within a lower and an upper bound.  The
function {\em main} declares and initializes variables for each letter
and models the puzzle using constraints.  In the constraint statement
body, the solution is finally displayed.

\index{send-more-money}
\index{examples!send-more-money}
\begin{Program}
\begin{ttlprog}
1\>\ttlModule{} sendmory2;\\
2\>\ttlImport{} io;\\
3\>\ttlConstraint{} all\_different (l: \ttlList{} \ttlOf{} {\bf!}int)\\
4\>\>\ttlWhile{} \ttlTl{} l $\neq$ \ttlNull{} \ttlDo{}\\
5\>\>\>\ttlVar{} ll: \ttlList{} \ttlOf{} {\bf!}int $\leftarrow$ \ttlTl{} l;\\
6\>\>\>\ttlWhile{} ll $\neq$ \ttlNull{} \ttlDo{}\\
7\>\>\>\>\ttlRequire{} \ttlHd{} l $\neq$ \ttlHd{} ll;\\
8\>\>\>\>ll $\leftarrow$ \ttlTl{} ll;\\
9\>\>\>\ttlEnd{};\\
10\>\>\>l $\leftarrow$ \ttlTl{} l;\\
11\>\>\ttlEnd{};\\
12\>\ttlEnd{};\\
13\>\ttlConstraint{} domain (v: {\bf!}int, min: int, max: int)\\
14\>\>\ttlRequire{} v $\geq$ min \ttlAnd{} v $\leq$ max;\\
15\>\ttlEnd{};\\
16\>\ttlFun{} main(args: \ttlList{} \ttlOf{} \ttlString{}): int\\
17\>\>\ttlVar{} s: {\bf!}int $\leftarrow$ \ttlVar{} 0;\\
18\>\>\ttlVar{} e: {\bf!}int $\leftarrow$ \ttlVar{} 0;\\
19\>\>\ttlVar{} n: {\bf!}int $\leftarrow$ \ttlVar{} 0;\\
20\>\>\ttlVar{} d: {\bf!}int $\leftarrow$ \ttlVar{} 0;\\
21\>\>\ttlVar{} m: {\bf!}int $\leftarrow$ \ttlVar{} 0;\\
22\>\>\ttlVar{} o: {\bf!}int $\leftarrow$ \ttlVar{} 0;\\
23\>\>\ttlVar{} r: {\bf!}int $\leftarrow$ \ttlVar{} 0;\\
24\>\>\ttlVar{} y: {\bf!}int $\leftarrow$ \ttlVar{} 0;\\
25\>\>\ttlRequire{} domain (s, 0, 9) \ttlAnd{} domain (e, 0, 9) \ttlAnd{} domain (n, 0, 9) \ttlAnd{}\\
26\>\>\>domain (d, 0, 9) \ttlAnd{} domain (m, 1, 9) \ttlAnd{} domain (o, 0, 9) \ttlAnd{}\\
27\>\>\>domain (r, 0, 9) \ttlAnd{} domain (y, 0, 9) \ttlAnd{}\\
28\>\>\>all\_different ([s, e, n, d, m, o, r, y]) \ttlAnd{}\\
29\>\>\>(s * 1000 $+$ e * 100 $+$ n * 10 $+$ d) $+$ (m * 1000 $+$ o * 100 $+$ r * 10 $+$ e) =\\
30\>\>\>(m * 10000 $+$ o * 1000 $+$ n * 100 $+$ e * 10 $+$ y)\\
31\>\>\ttlIn{}\\
32\>\>\>io.put ("s = "); io.put ({\bf!}s); io.nl ();\\
33\>\>\>io.put ("e = "); io.put ({\bf!}e); io.nl ();\\
34\>\>\>io.put ("n = "); io.put ({\bf!}n); io.nl ();\\
35\>\>\>io.put ("d = "); io.put ({\bf!}d); io.nl ();\\
36\>\>\>io.put ("m = "); io.put ({\bf!}m); io.nl ();\\
37\>\>\>io.put ("o = "); io.put ({\bf!}o); io.nl ();\\
38\>\>\>io.put ("r = "); io.put ({\bf!}r); io.nl ();\\
39\>\>\>io.put ("y = "); io.put ({\bf!}y); io.nl ();\\
40\>\>\ttlEnd{};\\
41\>\>\ttlReturn{} 0;\\
42\>\ttlEnd{};
\end{ttlprog}
\caption{Send-more-money example}
\label{prog:send-more-money}
\end{Program}

\chapter{The Standard Library}
\label{cha:turtle-library}

\newenvironment{modules}%
  {\newcommand\descr[2]{\index{##1@{\tt ##1} (Module)}%
\index{modules!##1@{\tt ##1}}%
{\tt ##1} \> ##2 \\}
   \begin{tabbing}
   \qquad\qquad\qquad\qquad\qquad \= \kill}
  {\end{tabbing}}

Besides the language definition and implementation, a standard library
is extremely important for the usefulness of a programming language.
The reference implementation of \turtle{} includes all modules
mentioned in this appendix.

The modules of the standard library are grouped into functional
categories and a short description is given for each.  The complete
documentation for the modules and their exported functions, data types
and variables is contained in the reference manual which comes with
the distribution.

\index{modules}
\index{standard library}

\subsection*{General}
\index{general modules}

\begin{modules}
\descr{math}{Mathematical constants and trigonometric functions}
\descr{random}{Random number generator}
\descr{compare}{Comparison functions for the basic data types}
\descr{compose}{Function composition}
\descr{identity}{Identity function}
\descr{option}{The {\tt option} data type, as known from Standard ML}
\descr{trees}{Binary trees}
\descr{bstrees}{Binary search trees}
\descr{cmdline}{Command line parsing and option handling}
\descr{hash}{Calculating hash values for basic data types}
\descr{hashtab}{Hash tables}
\descr{binary}{Byte-array handling}
\descr{exceptions}{Exception raising and handling}
\descr{filenames}{Filename manipulation}
\end{modules}


\subsection*{Input and Output}
\index{input and output modules}

\begin{modules}
\descr{io}{Basic input/output facilities for basic data types}
\end{modules}

\subsection*{Data Type Related}
\index{data type related modules}

\begin{modules}
\descr{ints}{Integer constants and functions}
\descr{longs}{Long integer functions}
\descr{reals}{Real constants and functions}
\descr{bools}{Boolean value functions}
\descr{chars}{Character handling}
\descr{strings}{String processing and conversion functions}
\descr{union}{Union data type of the basic data types}
\descr{strformat}{String formatting}
\end{modules}

\subsection*{List Utilities}
\index{list utility modules}

\begin{modules}
\descr{lists}{General list manipulation functions}
\descr{listmap}{Mapping functions over lists}
\descr{listsort}{List sorting}
\descr{listsearch}{Linear search on lists}
\descr{listfold}{Folding functions over lists}
\descr{listreduce}{Reducing lists with initial value}
\descr{listzip}{Combining two lists into one}
\descr{listindex}{List functions operating with indices}
\end{modules}


\subsection*{Array Utilities}
\index{array utility modules}

\begin{modules}
\descr{arrays}{General array manipulation functions}
\descr{arraymap}{Mapping a function over an array}
\descr{arraysort}{Array sorting}
\descr{arraysearch}{Linear and binary search on arrays}
\end{modules}

\subsection*{Tuple Utilities}
\index{tuple utility modules}

\begin{modules}
\descr{pairs}{Handling 2-tuples}
\descr{triples}{Handling 3-tuples}
\end{modules}


\subsection*{Low-Level}
\index{low-level modules}

\begin{modules}
\descr{core}{Low-level functions, like real number formatting}
\end{modules}


\subsection*{System-dependant Modules}
\index{system-dependant modules}

\begin{modules}
\descr{sys.files}{File handling}
\descr{sys.dirs}{Directory handling}
\descr{sys.net}{Network programming}
\descr{sys.times}{Time and date functions}
\descr{sys.users}{Accessing the user data base}
\descr{sys.procs}{Process handling}
\descr{sys.errno}{Operating system error codes}
\descr{sys.signal}{Unix signal handling}
\end{modules}


\subsection*{Internal Modules}
\index{internal modules}

\begin{modules}
\descr{internal.version}{Version numbers}
\descr{internal.random}{Random number generation}
\descr{internal.stats}{Run-time system statistics}
\descr{internal.gc}{Garbage collector interface}
\descr{internal.ex}{Low-level exception raising and handling}
\descr{internal.timeout}{Functions to be run in the background}
\descr{internal.limits}{System limit constants}
\end{modules}

\chapter{The \turtle{} System}
\label{cha:turtle-compiler}

During the work on this thesis, a compiler for the \turtle{} language
was developed which supports all of the language features described in
Chapter~\ref{cha:turtle}.\footnote{Note the limitations of the
  implementation as described in section~\ref{sec:impl-discussion}.}
The compiler translates \turtle{} source code into C source code which
in turn is compiled to executable machine code by a C compiler.

The compiler consists of a command line program which runs on
GNU/Linux and Solaris, and should be portable to other Unices and
Windows without much work.

The source code is available as a gzip'ed tar archive from the
\index{Turtle home page}
\turtle{} home page at {\tt
  http://user.cs.tu-berlin.de/\~{}mgrabmue/turtle/}.

\index{installation}
After downloading the file {\tt turtle-1.0.0.tar.gz}, the archive is
unpacked with the command

{\tt
zcat turtle-1.0.0.tar.gz | tar xf -
}

\noindent
or, if you use GNU tar:

{\tt
tar xzf turtle-1.0.0.tar.gz
}

\noindent
Then the source code must be configured and built with the commands

{\tt ./configure}

{\tt make}

\noindent
After completion, the compiler, the run-time system and the standard
library have been compiled and can be installed (after possibly
switching to administrator privileges) with the command

{\tt make install}

\noindent
Before installation, if you want to make sure everything worked fine,
you can compile and run a few test programs and examples by executing

{\tt make check}

\noindent
The installation requires a C compiler for compiling the compiler and
the run-time system, and for compiling the output of the \turtle{}
compiler.  After the installation, a C compiler is also necessary for
compiling \turtle{} code.  The \turtle{} system was tested with GCC
2.95.2.

Please read the files {\tt README} and {\tt INSTALL} after unpacking
the distribution for more detailed instructions on how to compile and
install the system.  The online documentation for the compiler, the
run-time system and the standard library is available in GNU info
format in the subdirectory {\tt doc}.

The subdirectory {\tt examples} contains some example programs,
ranging from {\tt minimal.t}, which is the minimal executable
\turtle{} program, to {\tt webserver.t} and utility modules, which
implement a simple, working web server.  Some constraint
im\-per\-ative programs for solving crypto-arithmetic puzzles and
simple mathematical problems are also included.

\end{appendix}

%%% Local Variables: 
%%% mode: latex
%%% TeX-master: "da.tex"
%%% End: 

%% End of appendix.tex.
