%% constraint-imperative.tex -- Introduction to CIP.
%%
%% Copyright (C) 2003 Martin Grabmueller <mgrabmue@cs.tu-berlin.de>

\chapter{Constraint Imperative Programming}
\label{cha:constraint-imperative-programming}
\index{constraint imperative programming}

Constraint imperative programming combines language elements of
traditional imperative programming languages with techniques from
constraint programming.

The advantages of constraint imperative programming will be introduced
by presenting an example adapted from literature \cite{lopez97phd,
  lopez94kaleidoscope}.  Fig.~\ref{pic:temperature} shows a
thermometer widget\footnote{widget = {\em wi}ndow ga{\em dget}, a
  graphical metaphor for human-machine interaction.}, whose ``mercury
column'' can be adjusted by dragging the mouse.  Input elements like
this thermometer are very common in graphical user interfaces (GUIs)
and lend themselves perfectly to demonstrating the advantages of
constraint support in interactive applications.  The upper end of the
column can be dragged up and down, but is of course restricted to stay
within the bounds of the scale.  Program~\ref{prog:imperative-example}
is a solution to this problem, presented in a language with typical
imperative features.  This example shows how closely connected the
program logic (reading the position of the mouse pointer and adjusting
the mercury column) and the maintenance of the program constraints
are.
%
\begin{figure}[h]
\begin{center}
\input{temperature.epic}
\end{center}
\caption{Thermometer widget}
\label{pic:temperature}
\end{figure}

%
\begin{Program}
\begin{ttlprog}
1\>\ttlWhile{} {\em mouse.pressed} \ttlDo{}\\
2\>\>\ttlVar{} {\em y}: int;\\
3\>\>{\em y} $\leftarrow$ {\em mouse.y};\`{\em read the position of the mouse pointer}\\
4\>\>\ttlIf{} {\em y} $>$ {\em scale.max} \ttlThen{}\`{\em restrict y-component to valid range}\\
5\>\>\>{\em y} $\leftarrow$ {\em scale.max};\\
6\>\>\ttlElse{}\\
7\>\>\>\ttlIf{} {\em y} $<$ {\em scale.min} \ttlThen{}\\
8\>\>\>\>{\em y} $\leftarrow$ {\em scale.min};\\
9\>\>\>\ttlEnd{};\\
10\>\>\ttlEnd{};\\
11\>\>{\em temp.max} $\leftarrow$ {\em y};\`{\em adjust the column}\\
12\>\ttlEnd{};
\end{ttlprog}
\caption{Imperative approach}
\label{prog:imperative-example}
\end{Program}
%% while mouse.pressed do
%%   var y: int;
%%   y := mouse.z;
%%   if y < scale.max then
%%     y := scale.max;
%%   else
%%     if y > scale.min then
%%       y := scale.min;
%%     end;
%%   end;
%%   temp.max := y;
%% end;
%
Program~\ref{prog:constrained-example} presents the constraint
imperative approach to this problem.  In this typical constraint
imperative language, the constraints are explicitly formulated and
their maintenance is delegated to the run-time constraint solver.
Since the constraints which limit the range of the column are
required, the equality constraint between the upper end of the mercury
column and the mouse pointer will only be enforced if the other two
constraints are not violated.
%
\begin{Program}
\begin{ttlprog}
1\>\ttlRequire{} {\em temp.max} $\leq$ {\em scale.max};\\
2\>\ttlRequire{} {\em temp.max} $\geq$ {\em scale.min};\\
3\>\ttlWhile{} {\em mouse.pressed} \ttlPrefer{} {\em temp.max} = {\em mouse.y};
\end{ttlprog}
\caption{Constraint imperative solution}
\label{prog:constrained-example}
\end{Program}
%% require temp.max <= scale.max;
%% require temp.max >= scale.min;
%% while mouse.pressed prefer temp.max = mouse.y;

The constraint imperative version is not only shorter and easier to
read and modify than the imperative one, but it is also easier to
check its correctness.  These are the main advantages of constraint
imperative programming.  This example shows how well constraints and
interactive, graphical applications interact, but as
Lopez~\cite{lopez97phd} noted, constraint imperative languages are
actually general-purpose languages, since they are a superset of
traditional programming languages.\footnote{Lopez writes about
  object-oriented programming languages, whereas this work is more
  targeted to traditional imperative languages without
  object-orientation.}

In this chapter, we will first review the existing literature about
constraint imperative programming.  Then we will identify the general
characteristics of constraint imperative programming languages and
summarize the main problems of this approach.

\section{Constraint Imperative Languages and Systems}

This section summarizes literature on constraint imperative
programming and on related topics like object-oriented programming
languages with constraint integration.  Constraint imperative systems
have been developed both as integrated programming languages where
constraints are added to some imperative base language, and as
programming libraries in various languages, which can be added to
existing programs without requiring the programmer to learn a new
language.

\subsection{\cool}
\index{COOL}
\index{Avesani}
\index{Perini}
\index{Ricci}
%
Avesani, Perini and Ricci~\cite{avesani90cool} describe the constraint
language \cool{} which is built on top of an object-oriented system.
Their goal was to develop an architecture for solving combinatorial
optimization problems.  In \cool{}, it is possible to create classes
and objects, and both of these can have slots.  Slots are used to
store values, and it is possible to declare slots as {\em
  constrainable}, what means constraints can be placed on slots, thus
constraining the values which can be stored into these slots.
Constraints are placed on slots by calling the built-in function {\em
  assert-constraint}.  The same function, called with different
parameters, is also used for removing constraints again. When all the
necessary constraints for modelling a problem have been placed on
slots, the built-in function {\em find} will calculate all solutions.
Interestingly, constraints can have an associated strength, similar to
the ones used in constraint hierarchies, but they are only used in
internal algorithms to the solver, and their semantics are not
precisely specified.

COOL is very tightly integrated with the underlying object-oriented
system and very specific to combinatorial problem solving.  It is not
well suited as a general purpose programming language.

\subsection{Alma-0: Embedded Search}
\index{Alma-0}
\index{Apt}

Alma-0 is the first language which was designed by Apt et al. for the
Alma project~\cite{apt98almaproject}.  The goal of this ongoing
project is the combination of imperative and declarative programming.
The designers decided to start with an imperative programming language
(\modula) and to add declarative features step by step.  First, only a
few language constructs for non-deterministic choice, succeeding and
failing statements and built-in backtracking were added to the
imperative base language, and the resulting language was called
Alma-0, showing that the language is only the first version of a
series of improving languages.

\index{trail}
%
With the embedded search capabilities, Alma-0 is already well suited
for solving many combinatorial problems and many illustrating examples
can be found in literature \cite{apt97search, apt98alma}.  The
implementation is that of traditional imperative languages with the
addition of a {\em trail} for storing variable assignments which might
need to be undone on backtracking.

The Alma project designers have written about the integration of
constraint into the language~\cite{apt98almaproject}, but there does
not exist a detailed description of these plans nor any implementation
yet.

The Alma-0 language could be called a logic imperative language,
rather than a constraint imperative language, because its semantics is
closer to logic programming languages than to constraint languages.
The planned extensions are more suited for constraint imperative
programming.  In~\cite{apt98almaproject}, the language is called an
``imperative constraint programming language'' and is introduced as
the natural extension of the concept of uninitialized variables as
they exist in Alma-0.  In Alma-0, variables can be uninitialized, and
when such a variable is used in an equality test, it gets initialized
with the other operand of the equality.  This concept will be
generalized for the planned constraint integration: variables will
then be initialized whenever they appear in constraint statements, of
which the equality test is a simple example.

\index{unknown}
\index{known, but varying}
%
In the planned language extension, variables are either ``unknowns''
or ``known, but varying''~\cite[p.~5]{apt98almaproject}.  The former
are variables in the mathematical sense, whereas the latter are
variables as in traditional imperative languages, where they stand for
storage locations which hold values.  Only ``unknowns'' can be
constrained in the proposed language.

The Alma project ``is a proposal for integrating constraints into ANY
imperative language''~\cite[p.~6]{apt98almaproject}, and some of the
proposed language elements will be used in the present thesis as well
(see Chapter~\ref{cha:turtle}).

\subsection{Kaleidoscope}
\index{Kaleidoscope}

\index{Lopez}
\index{Borning}
\index{Freeman-Benson}

Kaleidoscope is an object-oriented programming language which
integrates constraints as a fundamental concept.  It is described
in~\cite{lopez94kaleidoscope}, and the implementation is presented in
\cite{lopez94implementing}.  Benson et al. describe the principles
upon which it is based in~\cite{benson92int, benson91cip}.

\index{constraint!constructor}

Most of the imperative language constructs are modelled as
constraints, which are solved by a constraint solver integrated into
the run-time system.  The language provides primitive constraints, and
the user can define higher-level constraints which are then executed
in terms of the primitive ones by writing so-called {\em constraint
  constructors}.

A Kaleidoscope program consists of class definitions, which contain
attribute and method declarations.  In method bodies, all traditional
imperative statements like assignment, conditionals and loops can be
used.  Additionally, three different constraint statements are
available for placing constraints on object attributes.  The first one
enforces a constraint only once ({\em once}), the second for the rest
of the program execution ({\em always}) and the third as long as a
given condition remains true ({\em assert-during}).  The life-time of
constraints is also called {\em duration} (see
section~\ref{sec:constraints-in-an-imperative-environment}).

Constraints are tagged by {\em constraint strengths}, which indicate
the importance of the individual constraints.  Since Kaleidoscope
supports constraint hierarchies, weaker (less important) constraints
are not enforced when this is necessary to make stronger (more
important) constraints satisfiable.  For example, in the two
constraint statements,
%
\begin{ttlprog}
\>{\bf always} required {\em x} $>$ 0;\\
\>{\bf always} weak {\em x} = {\em y};
\end{ttlprog}
%
the second constraint will only have an effect on $x$ when the first
is not violated, that is, when $y$ has a value greater than $0$.  (The
words {\em required} and {\em weak} are symbolic names for constraint
strengths and can be defined by the programmer.  See also section
\ref{sec:constraints-in-an-imperative-environment}.)

The most interesting aspect of the Kaleidoscope approach is that the
semantics of imperative constructs such as assignment is modelled as
constraint statements.  This simplifies the language semantics as well
as the implementation, because fewer language constructs need to be
specified and implemented.  Variables in Kaleidoscope do not contain a
single value which varies, instead they contain an infinite sequence
of fixed values, where each element in this sequence corresponds to a
given point in time.  That means that a variable assignment as the
following
%
\begin{ttlprog}
\>$x$ $\leftarrow$ $x$ $+$ 1;
\end{ttlprog}
%
can be modelled as a constraint statement, where the indices indicate
the time points for which the variables stand:
%
\begin{ttlprog}
\>{\bf always} required $x_{i}$ $=$ $x_{i-1}$ $+$ 1.
\end{ttlprog}
%
The system automatically maintains the constraint
%
\begin{ttlprog}
\>{\bf always} weak $x_i = x_{i-1}$
\end{ttlprog}
%
for all variables, which ensures that each variable which is not
changed by an assignment between two time points has the same value as
it had before the last time step.

In constraint statements, data can flow in any direction between the
variables contained in constraints.  Sometimes, this is not what the
programmer wants, and Kaleidoscope provides {\em read-only}%
\index{read-only annotation}%
\index{?!read-only annotation} and {\em
  write-only annotations}%
\index{write-only annotation}%
\index{"!!write-only annotation}.  These specify which variables can
only be read but not modified, and which can only be stored into, but
not be used otherwise.  This is best shown by describing how an
assignment statement is translated from Kaleidoscope code
%
\begin{ttlprog}
\>$x \leftarrow y$;
\end{ttlprog}
%
into more basic constraint statements:
%
\begin{ttlprog}
\>{\bf always} required $tmp$ $=$ $y_{i-1}$?\\
\>{\bf always} required $x_i$ $=$ $tmp$?
\end{ttlprog}
%
The read-only annotations (written as question marks) specify that in
the first assignment, the information must flow from $y$ to $tmp$, and
in the second assignment from $tmp$ to $x$ because the variables on
the right can only be read, not written.  Write-only annotations are
written with an exclamation mark.

\index{constructor}
\index{constraint!constructor}
\index{splitting}
%
Besides the built-in primitive constraints, such as equality, identity
and linear arithmetic constraints over integers and real numbers,
Kaleidoscope supports the definition of higher-level constraints
(constraint constructors).  These can constrain variables of more
complex types, such as user-defined objects.  Whenever a constraint
constructor on complex data types appears in a constraint statement
and the statement is executed, the constructor is called in order to
place primitive or other user-defined constraints on the parts of the
complex object.  This evaluation scheme for constructors is called
{\em splitting} \cite{lopez97phd}, because it splits complex
constraints into simpler ones.  Figure~\ref{pic:splitting} shows how
splitting works for an equality constraint on two point objects, both
having attributes $x$ and $y$, standing for cartesian coordinates.
%
\begin{figure}[h]
\begin{center}
\input{splitting.epic}
\end{center}
\caption{Splitting}
\label{pic:splitting}
\end{figure}

It should be noted that splitting makes it possible to handle
constraints on complex data structures even when the constraint
solvers are not powerful enough for working on these data structures,
since they must only provide constraints on numbers.  On the other
hand, the splitting of complex data structures loses higher-level
information on the overall problem specification which could be useful
for more powerful constraint solvers.

The philosophy behind the Kaleidoscope design is quite different from
all the other systems presented in this section, and from the
\turtle{} language developed in this thesis.  In Kaleidoscope,
constraints are employed to automatically maintain invariants on the
relations between objects and the values of their attributes.  The
other constraint programming systems are more oriented towards problem
solving: they support the modelling of problems using constraints and
the search for solutions by global resolution algorithms.  They
concentrate on inferring values for variables or attributes and ignore
the relations between complex objects.

Kaleidoscope is designed to be a general purpose language, but only
research prototypes have been implemented, and it is not clear from
literature whether they are useful for real-life applications.

\subsection{JACK}
\index{JACK}
\index{Java Constraint Kit}
\index{Abdennadher}

\jack{} (\java{} Constraint Kit) \cite{abdennadher01:jack:inp,Jack2}
is a preprocessor and library which supports constraint programming in
\java{}.

Constraint problems are formulated by writing \java{} constraint
handling rules (JCHR), which are similar to the constraint handling
rules developed by Fr\"uhwirth \cite{fruehwirth98chr}. CHR programs
define rules for the propagation and simplification of constraints and
are well suited for writing constraint solvers, or for extending
existing constraint solvers.  In \jack{}, the programmer defines
handlers (also called constraint solvers) which include a set of rules
for handling constraints.  These CHR programs can be translated to
\java{} code and compiled by a \java{} compiler.

In addition to JCHR, \jack{} contains a tool for graphically
displaying constraint stores and the relations between the constraints
in the store.  This visualization can be used for debugging JCHR
programs and for improving their efficiency by investigating how the
various constraint handling rules are applied.  With this knowledge,
the rules can be re-written to make them more efficient.

\subsection{\djava{}: Declarative Java}

\djava{} is also an extension of the \java{} language with integration
of constraints~\cite{zhou98dj}.  In \djava{}, classes can include
component declarations, attribute declarations, constraints and
actions.  Components are the graphical components which are parts of
the class, and attributes are similar to member variables in \java{},
but their values are determined by the system as specified by the
constraints.  Actions are used to add behaviour to components.  As in
Kaleidoscope, user-defined constraints combine built-in and other
user-defined constraints to form more complex constraints on primitive
data types or classes.

\djava{} is implemented as a compiler which translates \djava{}
programs to \java{} code.  The emphasis of the \djava{} approach is on
the creation of graphical programs, but the language can also be used
for solving arbitrary combinatorial constraint problems.  The system
also includes facilities for specifying how the solutions to these
problems are displayed graphically.

The constraint capabilities of \djava{} are designed for graphical
user interface programming, so only the imperative part of the
language (which is essentially \java{}) is useful as a general purpose
language.

\subsection{Constraint Libraries}

\index{Koalog}
\index{ILOG}

There are several commercial constraint solving libraries available,
for example from the companies ILOG~\cite{ilogwww} and
Koalog~\cite{koalogwww}.  The ILOG Solver is a library for
\cplusplus{} \cite{ILOG}, whereas the Koalog Solver is for
\java{}~\cite{koalog}.

The ILOG Solver is a commercially successful library for constraint
programming which makes extensive use of language features of
\cplusplus{}, such as object-oriented programming and overloading.
Thus constraints can be created by simply writing arithmetic
expressions involving objects of the predefined classes for
representing variables.
 
The Koalog Solver consists of a library of classes for the two main
tasks in constraint programming: modelling and solving.  The first
task is supported by providing a set of classes for representing
variables and constraints, the second task by providing constraint
solvers and constraint optimizers for searching for any, all or
optimal solutions.  The system can be extended by creating new
subclasses of existing constraints with new behaviour and by
customizing and extending the solving algorithms.

Because of the integration of constraints into these imperative
languages, one could call these systems constraint imperative, but it
is important to note that this is not the kind of seamless integration
aimed at by the other systems, where constraints are an important
aspect of both the language syntax and semantics.


\subsection{Other Systems}

\index{K\"ok\'eny}
\index{YAFCRS}

K\"ok\'eny~\cite{ny94yet} describes an open architecture for embedding
constraints into object-oriented programs.  The system, called YAFCRS
(Yet Another Framebased Constraint Resolution System) mainly provides
representation methods for finite domain constraint problems and
solving algorithms.  It aims at being both usable for inexperienced
users by providing predefined algorithms for problem solvers, and at
being extensible and flexible for advanced users.  Therefore, there
are many built-in constraints to choose from, and there are mechanisms
for defining user-defined constraints.  These user-defined constraints
are also solved by splitting.

\index{Lamport}
\index{Schneider}
\index{aliasing}
\index{typing}

Lamport and Schneider~\cite{lamport84aliasing} invented a proof system
for a language in which typing and aliasing of variables can be
specified using constraints.  {\em Aliasing} means that two or more
variables share the same representation, so that a modification of one
variable also changes the others.  This system is based on Hoare
Triples and is intended for proving the correctness of imperative
programs.  Their system for handling aliasing is not restricted to
simple equality of variables, but can deal with arbitrary relations
between program variables, as long as they are expressible by
constraints.  Type declarations for variables are also treated as
constraints on these variables, although they permit the traditional
way of stating variable types by type annotations.

As an example (taken from~\cite{lamport84aliasing}), consider the
following variable declaration which defines the variables $f$ and $g$
(intended as temperatures in degrees Fahrenheit and degrees Celsius)
and relates them with constraints so that they always refer to the
same temperature.
%
\begin{ttlprog}
\>\ttlVar{} {\em f}, {\em c}: real {\bf constraints} $f=g\times c/5+32$ \ttlIn{} $S$
\end{ttlprog}
%
$S$ stands for an arbitrary statement and the constraint is enforced
as long as this statement executes.

Pachet and Roy~\cite{pachet95mixing} propose an extension of
\smalltalk{}~\cite{goldberg83smalltalk} with finite-domain constraint
satisfaction mechanisms.  In their system, constraints can be defined
over arbitrary \smalltalk{} objects, so that it is possible to re-use
existing components and combine them with constraints to form new
programs, instead of requiring that everything should be re-written
completely.

\section[Imperative and Declarative Language Characteristics]{Imperative and Declarative Language\\Characteristics}

There are various aspects which are important for understanding the
problems of integrating imperative languages with higher-level
features such as higher-order functions or constraints.

The interpretation of an imperative program relies on the notions of
state and time.  A program has an associated state (represented as the
values of the program variables) at a given time.  This state changes
over time, as assignments to the variables are being made.  Therefore,
it is necessary to take time into account when trying to find out what
an imperative program does.  Declarative languages do not have this
problem, as their computations are specified independent of time.

Related to this, imperative languages are statement-oriented, whereas
declarative languages are expression-oriented.  Imperative programs
contain a sequence of statements which are to be executed in order,
whereas declarative programs consist of expressions, where the order
of evaluation is not important as long as mathematical and otherwise
semantic restrictions do not interfere.  In declarative languages, it
does not even matter whether an expression is evaluated more than
once, because the evaluation has no effects except for producing a
value.  This freedom makes correctness proving much easier for
declarative programs, and the language implementation can benefit as
well since optimization is less difficult.

The main problem with integrating constraints and imperative languages
is the interaction between the imperative program flow, which alters
the state of the program, and the declarative interpretation of
constraints, which does not know anything about state or time.  One or
both of the two interpretations of computation need to be modified in
order to bring them together.


\section{Constraints in an Imperative Environment}
\label{sec:constraints-in-an-imperative-environment}

When constraints are used in an imperative environment, many problems
arise which do not exist in either of the two paradigms, as long as
these are treated in isolation.  Some of these problems appear
whenever declarative and imperative languages are combined, others are
specific to constraint imperative languages.

% Because of the notion of state and time in imperative systems, they
% have to be considered not only when specifying constraints, but when
% solving them, too.

\paragraph{Side Effects.}
\index{side effects}

As soon as imperative languages are part of a system, it is necessary
to define the semantics of side effects such as assignment statements
or imperative in-/output.  In a constraint imperative language, the
possibilities of interaction between side effects and constraint
statements must be examined, and rules for these interactions have to
be defined.  An example where this problem occurs is when a variable
is constrained by one or more constraints and the program assigns a
value to the variable which does violate the constraint(s).

\paragraph{Object Identity.}
\index{object identity}

In imperative languages, all objects created during a program run are
identified by their identity, so that more than one object which looks
the same can actually be different objects occupying different areas
of the computer's memory.  In higher-level languages, different
objects with the same contents are normally indistinguishable, thus
leading to more freedom in the implementation of those languages.

In non-imperative constraint languages, for example, an equality
constraint between two variables can simply be satisfied by assigning
the same object reference to both of them, or by copying the whole
object.  Since there is no way the programmer can notice the
difference, the effect is the same. In imperative languages
side-effects can later make the equality invalid in the latter case,
whereas in the former case, the equality is preserved because the
objects in both variables are simultaneously modified.

Lopez, Freeman-Benson and Borning have examined the implications of
object identity \cite{lopez94identity}, and the design of
Kaleidoscope was influenced by this so that it provides both {\em
  identity} constraints (written as ``=='') and {\em equality}
constraints (written as ``='').

\paragraph{Duration.}  
\index{duration}

The duration of a constraint is the time span in which the constraint
must be enforced.  This problem does not exist for declarative
constraint languages, because they specify constraints between
variables which have to hold as long as the variables exist.  For
imperative languages, where the flow of control over time is
explicitly expressed, it is necessary to state not only which
constraints need to hold, but also {\em when}.  On the one hand, this
adds much complexity to constraint imperative programs, on the other,
it makes them more flexible.

Kaleidoscope offers three possible durations for constraints: {\em
  once} means that a constraint is added to the store, the store is
re-solved and then the constraint is immediately retracted.  This is,
for example, used for implementing assignment with constraints.  {\em
  always} constraints are defined once and then the constraint is
enforced for the rest of the program run.  {\em assert-during}
constraints are active while a statement (which may contain loops, so
the duration is not known when asserting the constraint) is executed,
and then retracted.

% It is not clear from literature whether {\em always} constraints
% really exist forever, or whether their lifetime depends on the objects
% which are constrained by the constraints.


\paragraph{Strengths.}
\index{strength}
\index{stay constraint}
\index{default constraint}
\index{constraint!strength}
\index{constraint!default}
\index{constraint!stay}
%
Constraints need to be labelled with strengths when they shall form a
constraint hierarchy, in order to deal with under- and
over-constrained problems.  Although a constraint imperative system
without constraint hierarchies could be designed, its usefulness would
be drastically reduced, because it would be difficult to constrain
variables while the program dynamically adds or retracts constraints.
So-called {\em default constraints} or {\em stay constraints}
\cite{borning2000splitstay} are very useful, especially for graphical
applications, where the user expects the graphical objects to stay
where they are as long as no other (stronger) constraints are placed
on them, such as alignment constraints or movement with the mouse.

It should be noted that constraint hierarchies are useful in the
context of other constraint languages too, and they have been
integrated into constraint logic languages~\cite{wilson94hclp}.

In Kaleidoscope, at least one constraint strength called {\em
  required} does exist, and the user can define additional strengths
symbolically in a {\em strength declaration}, which lists all
available strengths for the program, strongest to weakest.  The
following example shows a strength declaration which declares five
strengths:
%
\begin{ttlprog}
\>{\bf strength} required, strong, medium, weak, weakest.
\end{ttlprog}
%
In all constraint statements following this declaration, the
constraint strengths {\em required}, {\em strong}, {\em medium}, {\em
  weak} and {\em weakest} can be used.

\subsubsection{Consequences}

Constraint imperative programming systems introduce a complete new set
of problems.  One could argue that they combine the problems of
declarative systems with the problems of imperative ones.  Existing
constraint imperative systems are slower than traditional imperative
systems, and they do not have the same clean semantics as the
declarative constraint programming languages for a mathematical
treatment for proving their correctness.  The first of these arguments
can be addressed by the fact that none of the existing systems had the
same development time as widespread imperative systems, and could be
well optimized to run faster.  For example, the Kaleidoscope'93
implementation was reported to run much faster than the first
implementation, Kaleidoscope'90.  Additionally, it must be said that
most of the systems run on top of other high-level systems, such as
Smalltalk or Lisp and are research projects.  Implementing them on
bare machines and optimizing them would improve their performance
further.

The argument about losing correctness is strong, but considering the
fact that the goal is to get the user to use more and more of the
declarative features of the language, the correctness is expected to
increase with the programmer's ability to use more powerful
techniques.


%%% Local Variables: 
%%% mode: latex
%%% TeX-master: "da.tex"
%%% End: 

%% End of constraint-imperative.tex.
