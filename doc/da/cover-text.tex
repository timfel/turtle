%% cover-text.tex -- German cover and summary for the Diplomarbeit.
%%
%% Copyright (C) 2003 Martin Grabmueller <mgrabmue@cs.tu-berlin.de>

\def\germantoday{\number\day. \ifcase\month\or
  Januar\or Februar\or M\"arz\or April\or Mai\or Juni\or
  Juli\or August\or September\or Oktober\or November\or Dezember\fi
  \space\number\year}

\thispagestyle{empty}

\noindent
Technische Universit\"at Berlin\\
Fakult\"at IV (Elektrotechnik und Informatik)\\
Institut f\"ur Softwaretechnik und Theoretische Informatik\\
Fachgebiet \"Ubersetzerbau und Programmiersprachen\\
Franklinstr. 28/29\\
10587 Berlin

\vskip3cm

\begin{center}
{\Large 

% \fbox{\vbox{{\Huge \sc Vorl\"aufige Version}\\
% {\sc vom \today}}}

\vskip1cm

Diplomarbeit

\vskip1cm

{\huge \sf Constraint-imperative Programmierung}

\vskip1cm

Martin Grabm\"uller

%\vskip0.01cm

Matr-Nr. 183283

\vskip0.5cm

28. Februar 2003

\vskip2cm

\begin{tabular}{ll}
Pr\"ufer: &Prof. Dr. Peter Pepper\\
Betreuerin: &Dr. Petra Hofstedt
\end{tabular}
}

\end{center}

\cleardoublepage

{}
% \begin{center}
% {\large \bfseries Eidesstattliche Versicherung}
% \end{center}

% \vskip0.5cm

% \noindent
% Die selbst\"andige und eigenh\"andige Anfertigung versichere ich an
% Eides statt.

% \vskip1cm

% \noindent
% Berlin, den 28. Februar 2003

% \vskip2cm

% \noindent
% \rule{7cm}{0.5pt}

% \noindent
% Martin Grabm\"uller

\begin{tabbing}
Load and store instructions\=\kill\\
\end{tabbing}
\cleardoublepage

\chapter*{Aufgabenstellung}
%% thema.tex -- Aufgabenstellung.
%%

%\begin{center}{\bf \Large AUFGABENSTELLUNG F\"UR DIE DIPLOMARBEIT}\end{center}

%\vspace*{5mm}

\hrule

\vspace*{2mm}

\noindent Fachgebiet \"Ubersetzerbau und Programmiersprachen 

\vspace*{2mm}

\hrule

\vspace*{2mm}

\noindent Diplomand: Martin Grabm\"uller (Matr-Nr. 183283)

\vspace*{2mm}

\hrule

\vspace*{2mm}

\noindent Betreuer/Innen: Peter Pepper, Petra Hofstedt

\vspace*{2mm}

\hrule

\vspace*{4mm}

%----------------------------------------------------------------------------

\noindent {\bf \underline{Thema:}} 
\hspace*{5mm} 
\begin{minipage}[t]{0.85\linewidth}
Constraint-imperative Programmierung
\end{minipage} 
 
%\vspace{3mm}
\section*{CIP -- Constraint-imperative Programmierung}

W\"ahrend in imperativen Programmiersprachen Algorithmen in
Anweisungen zerlegt werden, die angeben, {\em wie} durch die Modifikation
des Programmzustands die L"osung eines Problems berechnet wird, 
zeichnen sich deklarative Programmiersprachen dadurch aus,
dass spezifiziert wird, {\em was} berechnet wird.  

Eine der j"ungsten Entwicklungen der deklarativen Programmiersprachen
ist das {\em Constraint Programming}, das sich aus der logischen
Programmierung entwickelt hat und mittlerweile als eine Obermenge
dieser betrachtet wird.  Constraint-Programme bestehen aus den
Spezifikationen der Programmvariablen, der Wertebereiche dieser
Variablen und der Bedingungen, die f"ur die Variablen gelten m"ussen.
Da hier die Eigenschaften einer {\em L"osung} des Problems beschrieben
werden, aber nicht ein oder der {\em L"osungsweg}, ist es die Aufgabe
des Compilers und der Laufzeitumgebung, einen solchen Weg zu ermitteln
und L"osungen zu berechnen.

Constraint-Sprachen sind inzwischen um Konzepte verschiedener anderer 
-- meist deklarativer -- Programmiersprachen erweitert worden.
%
Wenig untersucht ist dagegen die Kombination aus imperativen und
Constraint-Sprachen.  Borning und Freeman-Benson
\cite{benson92int,benson91cip} pr"agten den Begriff der {\em
  constraint-imperativen Programmierung} \cite{CIP1,CIP2} und
entwickelten die Sprache Kaleidoscope, die objektorientierte und
Constraint-Konzepte in eine Sprache integriert.

Im folgenden soll der Begriff der constraint-imperativen
Programmiersprachen f"ur alle Sprachen verwendet werden, die sowohl
Constraint-Programmierung als auch imperative Programmierung
unterst"utzen.  Objektorientierung soll dabei keine Voraussetzung
sein.  Stattdessen interessiert uns vor allem das Zusammenspiel
zwischen Constraints und Constraintl"osern auf der einen und
imperativen Sprachkonstrukten auf der anderen Seite.

Die Kombination dieser beiden Programmierparadigmen bietet eine Reihe
von Vorteilen: Imperative Programmierung ist sehr verbreitet und wird
von vielen Programmierern verstanden und als nat"urlich angesehen.
Weiterhin ist die imperative Programmierung gut untersucht und es gibt
verbreitete und effizient implementierte Programmiersysteme.  
%
Viele Problemstellungen, die von Natur aus imperativer Art sind
(Steuerungs- und Regelsysteme, interaktive Systeme), lassen sich sehr
leicht imperativ l"osen.  
%
Auf der anderen Seite lassen sich komplexe Algorithmen nur sehr
umst"andlich auf dem niedrigen Abstraktionsniveau imperativer Sprachen
formulieren und f"uhren so zu schwer verst"andlichen Programmen.
Genau bei diesen Problemen haben sich deklarative (funktionale,
logische und constraintbasierte) Programmiersprachen bew"ahrt.  Beim
Constraint Programming werden nur die Beziehungen formuliert, die
zwischen den betrachteten Objekten bestehen, nicht aber die
Algorithmen, die diese Beziehungen aufrechterhalten.  Auf diese Weise
bleiben die formulierten Programme nahe an der Spezifikation, was
nicht nur die Verst"andlichkeit, sondern auch die Korrektheit
erh"oht.

Die constraint-imperative Programmierung versucht, die Vorteile dieser
beiden Paradigmen zu verbinden, um so intuitive, verst"andliche und
effiziente Programme zu erm"oglichen.

\section*{Aufgabenstellung}

In der angestrebten Diplomarbeit soll eine constraint-imperative 
Programmiersprache entwickelt werden.

Dazu sind zun\"achst existierende Ans\"atze, wie die
Programmiersprachen Alma-0 \cite{apt97search,apt98alma} und
Kaleidoscope \cite{lopez94kaleidoscope,lopez94implementing},
Erweiterungen objektorientierter Sprachen, wie COOL
\cite{avesani90cool}, und Bibliotheken, die Constraintl"oser
nachtr"aglich in bestehende Programmiersprachen einbetten, wie
Smalltalk \cite{pachet95mixing}, Java
\cite{abdennadher01:jack:inp,Jack2} und der ILOG-Solver \cite{ILOG} zu
analysieren und zu ber\"ucksichtigen.

Weiterhin ist der Einsatz von Constrainthierarchien, in denen
erzwungene (required) und gew"unschte (preferential) Constraints
unterschieden werden
\cite{borning92constrainthierarchies,bartakHierarchies1}, zu
diskutieren.  Constrainthierarchien erlauben es vor"ubergehend oder
l"anger andauernd mit widerspr"uchlichen Constraints umzugehen, was in
einer constraint-imperativen Sprache von Vorteil sein k\"onnte.

In der zu entwickelnden Sprache sollen die "ublichen Steueranweisungen und
Datenstrukturen imperativer Sprachen enthalten sein, 
ebenso wie Konstrukte, die die Constraint-Programmierung unterst"utzen.

Die Sprache soll gro"s genug sein, um mit con\-straint-im\-perativen
Programmen experimentieren zu k"onnen, sich allerdings auf die
imperativen Kernkonstrukte beschr"anken. 
Um die Sprache klein zu halten, soll Objektorientierung nicht implementiert werden.
%
Die Programmiersprache ist zu entwerfen, zu definieren und
prototypisch zu implementieren.

Die Arbeit muss sowohl theoretische als auch praktische Gesichtspunkte 
ber\"ucksichtigen. Es sind bei nachfolgenden Teilaufgaben jeweils 
%
eine kurze Einf\"uhrung in das entsprechende Thema zu geben, 
notwendige Begriffe zu definieren und an Beispielen zu illustrieren.
%
Entwurfsentscheidungen sind zu begr\"unden und Alternativen zu diskutieren. 

\vskip2ex
%\newpage

\noindent {\underline{Teilaufgaben:}}

\begin{itemize}
  
\item Vorstellung und Einordnung des Themas, Bezug zu existierenden
  Ans\"atzen herstellen.
 
\item Sprachentwurf:

  \begin{itemize}
  \item Auswahl der notwendigen/gew\"unschten Konstrukte der zu
    integrierenden Paradigmen,
    
  \item Demonstration ihres Zusammenspiels an Beispielen und
    
  \item Angabe von Syntax und Semantik (in angemessenem Umfang).

  \end{itemize}
 
\item Die Implementierung eines Compilers f\"ur die Sprache soll
  lediglich prototypisch erfolgen.
      
\item Sp\"atere Erweiterungen der Sprache und ihrer Implementierung
  sollen diskutiert werden.

\end{itemize}

%\bibliography{biblio.thema}

%---------------------------------------------------------------------------

%%% Local Variables: 
%%% mode: latex
%%% TeX-master: "da.tex"
%%% End: 

%% End of cover-text.tex.


\cleardoublepage


\chapter*{Zusammenfassung}

In Computeranwendungen werden Objekte der realen Welt vereinfacht auf
Modelle abgebildet, um diese dann durch Datenstrukturen zu
repr\"asentieren und mittels geeigneter Algorithmen zu verarbeiten.
Auf diese Weise k\"onnen die Ver\"anderungen der Wirklichkeit
modelliert werden.  Die Beziehungen zwischen den einzelnen
Datenobjekten sind ein wichtiger Bestandteil des Modells, und es ist
w\"unschenswert, diese Beziehungen pr\"azise zu spe\-zi\-fi\-zie\-ren.
Da einzelne Objekte und die Beziehungen zwischen Objekten im Laufe
eines Programms ver\"andert werden (z.B. in interaktiven Anwendungen
oder Simulationen), ist es notwendig, laufend sicherzustellen, ob
durch eine Zustands\"anderung die Spezifikation verletzt wurde und
gegebenenfalls einen konsistenten Zustand wiederherzustellen.
Weiterhin muss es auch m\"oglich sein, neu erzeugte Objekte und deren
Beziehungen verarbeiten bzw. bestehende Beziehungen erweitern zu
k\"onnen.  Genau diese Objektbeziehungen sind es, die in der
Constraint-Programmierung nicht nur genutzt werden, um die Konsistenz
einer Problembeschreibung zu wahren, sondern sogar, um die L\"osungen
f\"ur das Problems zu berechnen.

Constraints sind Bedingungen im mathematischen Sinne und eignen sich
hervorragend, um Einschr\"ankungen von Werten und Beziehungen zwischen
Objekten zu spezifizieren, w\"ahrend Constraint-L\"oser als
Programmkomponenten geeignet sind, verletzte Einschr\"ankungen zu
erkennen und ggf. durch Anpassungen der beteiligten Objekte wieder zu
erf\"ullen.  Diese Eigenschaften lassen es sehr erstrebenswert
erscheinen, Programme unter der Verwendung von Constraints zu
entwickeln und die Vorteile der deklarativen Programmierung zu nutzen:
\"Ubersichtlichkeit, mathematische Fundierung, Effizienz und effektive
Programmentwicklung.  Da allerdings ein schneller Umstieg von
bestehenden Programmiergewohnheiten auf neue, abstraktere
Vorgehensweisen in der Breite nur schwer umzusetzen ist, empfiehlt es
sich, existierende Programmiersprachen bzw. Programmierstile zu
erweitern.  Auf diese Weise lassen sich bestehendes Wissen und
bestehende Fertigkeiten weiter nutzen, w\"ahrend erweiterte Verfahren
auf abstrakterer Ebene eine Steigerung der Korrektheit und
(Programmier-)Effizienz erm\"oglichen.  Dies ist das Ziel der
Entwicklung so genannter Multiparadigma-Programmiersprachen, die
wesentliche Elemente verschiedener Programmierstile in einer einzigen
Sprache kombinieren, um deren Vorteile gemeinsam nutzen zu k\"onnen.

\vskip1ex

Diese Arbeit befasst sich mit der Verschmelzung zweier sehr
unterschiedlicher Programmierparadigmen in einer Programmiersprache.
Zum einen betrachten wir die deklarative Programmierung mit
Constraints, also die Spezifizierung der Eigenschaften, die zwischen
den Objekten eines Programms gelten m\"ussen, zum anderen die
imperative Programmierung, die auf Zustandsver\"anderungen durch
Anweisungen basiert.  Auf der einen Seite steht also die
Spezifikation, {\em was}\/ berechnet werden soll, auf der anderen
Seite der Algorithmus, {\em wie}\/ dies geschehen soll.  Die Frage,
wie diese zun\"achst gegens\"atzlich scheinenden Konzepte miteinander
verkn\"upft werden k\"onnen und unter welchen Voraussetzungen dieses
Vorgehen \"uberhaupt sinnvoll ist, soll der Schwerpunkt der Arbeit
sein.

\vskip1ex

Nach einer Einf\"uhrung in das Thema und das Problemumfeld in
Kapitel~\ref{cha:introduction} beschreibt die Arbeit in
Kapitel~\ref{cha:constraint-programming} zun\"achst die
constraint-basierte Programmierung.  Dabei werden grundlegende
Begriffe der Constraint-Programmierung, wie z.B.  Constraints und
Constraint-L\"oser, erkl\"art sowie verschiedene Ans\"atze der
Implementierung constraint-basierter Systeme beschrieben.  Weiterhin
werden existierende Arbeiten zu Verbindungen von Constraint-Sprachen
mit anderen (vor allem deklarativen) Programmiersprachen untersucht.

Eine besondere Verbindung zweier Programmierparadigmen, n\"amlich die
constraint-imperative Programmierung, wird in
Kapitel~\ref{cha:constraint-imperative-programming} beschrieben.  Da
dies der Schwerpunkt dieser Arbeit ist, sollen bisherige Arbeiten auf
diesem Gebiet ausf\"uhrlicher behandelt werden und charakteristische
Merkmale und grundlegende Elemente dieser Programmiersprachen
herausgearbeitet werden.  Dabei handelt es sich einerseits um
vollwertige Programmiersprachen, die eine vollst\"andige Integration
von constraint-basierter und imperativer Programmierung beabsichtigen,
andererseits um so genannte Constraint-Solving Toolkits, das sind
Programmbibliotheken, die Constraint-L\"oser in Form
wiederverwendbarer Module an imperative Sprachen anbinden. Neben einer
Darstellung der Gemeinsamkeiten dieser Sprachen und Bibliotheken
werden Vor- und Nachteile der einzelnen Systementw\"urfe beleuchtet.

Nach ausf\"uhrlicher Darstellung der Literatur auf diesem Gebiet
werden in Kapitel~\ref{cha:turtle} Ver\-bess\-er\-ungs- und
Erg\"anzungsvorschl\"age als Entwurf der neuen
con\-straint-im\-per\-a\-ti\-ven Programmiersprache \turtle{}
dargestellt.  Neben der constraint-basierten und imperativen
Programmierung haben dar\"uberhinaus einige Konzepte der funktionalen
Programmierung wie algebraische Datentypen, Funktionen h\"oherer
Ordnung und Polymorphie Einfluss auf den Sprachentwurf genommen.
Zun\"achst werden die wesentlichen Sprachkonstrukte benannt, die eine
imperative Programmiersprache zu einer constraint-imperativen Sprache
erweitern, dann werden nach einer kurzen Einf\"uhrung in Syntax und
Semantik von \turtle{} besondere Eigenschaften ausf\"uhrlicher
dargestellt.  Von besonderem Interesse ist dabei nat\"urlich die
Integration von Constraints und funktionalen Konzepten in eine
ansonsten herk\"ommliche imperative Sprache.  Abgeschlossen wird das
Kapitel durch die formale Darstellung der Sprachsemantik in Form einer
operationalen Semantik, die auf der \"Ubersetzung der Quellsprache in
die Sprache einer abstrakten Maschine und der Beschreibung der
Programmausf\"uhrung auf dieser Maschine aufbaut.

Die Ausarbeitung dieser Diplomarbeit umfasst nicht nur den
Sprachentwurf einer con\-straint-im\-pe\-ra\-ti\-ven Sprache, f\"ur
die Sprache wurde auch ein Programmsystem bestehend aus Compiler,
Laufzeitsystem, einer Bibliothek wichtiger Module und zwei
Constraint-L\"osern implementiert.  Diese Implementierung wird in
Kapitel~\ref{cha:turtle-impl} beschrieben.  Dabei werden einige
Aspekte des Kompilierungsvorgangs herausgegriffen, die aufgrund der
Integration mehrerer unterschiedlicher Sprachparadigmen \"uber eine
Compilerbeschreibung in der allgemeinen
\"Uber\-setz\-erbau\-li\-te\-ra\-tur hinausgehen.  Abgeschlossen wird
das Kapitel durch experimentelle Vergleiche der Leistung des
entwickelten Systems mit einer traditionellen imperativen
Programmiersprache.

Das letzte Kapitel fasst die Ergebnisse der Arbeit zusammen und
bewertet die Resultate.  Weiterhin werden alternative Entw\"urfe und
Implementierungsaspekte diskutiert und die \"Uberlegungen dargestellt,
die zur Entscheidung gegen deren Verwendung gef\"uhrt haben.
Anschlie\ss{}end werden zuk\"unftige Arbeiten vorgeschlagen, die
entweder aus Zeitgr\"unden nicht in diese Arbeit eingeflossen sind
oder nicht mehr der Aufgabenstellung der Arbeit zuzuordnen sind, aber
dennoch interessante nahegelegene Themengebiete betreffen.

Der Anhang umfasst die formale Grammatik der entwickelten
Programmiersprache, den Quelltext dreier Beispielmodule zur
Illustration, eine \"Ubersicht \"uber die im Rahmen der
Implementierung entwickelten Bibliotheksmodule sowie Bezugs- und
Installationshinweise f\"ur das entwickelte Programmiersystem.

%%% Local Variables: 
%%% mode: latex
%%% TeX-master: "da.tex"
%%% End: 

%% End of cover-text.tex.
