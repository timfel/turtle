%% da.tex -- Diplomarbeit
%%
%% Copyright (C) 2003 Martin Grabmueller <mgrabmue@cs.tu-berlin.de>

\documentclass[a4paper,12pt,oneside]{book}
\usepackage[german,english]{babel}
\usepackage{a4wide}
\usepackage{amsmath}
\usepackage{float}
\usepackage{epic}
%\usepackage{fancyheadings}
\usepackage{makeidx} \makeindex


\pagenumbering{roman}
\floatstyle{ruled}
\newfloat{Program}{htbp}{lotp}[chapter]

% Commands for adding annotations to the text.
%
\newcommand\fixme[1]{{\em [FIXME: #1]}}
\newcommand\informal[1]{{\em [#1]}}

% The following are the names of various programming languages.
%
\newcommand\turtle{Turtle}
\newcommand\java{Java}
\newcommand\modula{Modula-2}
\newcommand\cool{COOL}
\newcommand\ada{Ada}
\newcommand\jack{JACK}
\newcommand\djava{DJ}
\newcommand\cee{C}
\newcommand\cplusplus{C++}
\newcommand\cobol{Cobol}
\newcommand\fortran{Fortran}
\newcommand\goffin{Goffin}
\newcommand\smalltalk{Smalltalk}
\newcommand\haskell{Haskell}
\newcommand\mllanguage{ML}
\newcommand\lisp{Lisp}
\newcommand\opal{Opal}
\newcommand\pascal{Pascal}
\newcommand\gambit{Gambit}
\newcommand\scheme{Scheme}

% Environment for typesetting Turtle programs and pseudo code
% fragments.
%
\newenvironment{ttlprog}%
  {\newcommand\res[1]{{\bf ##1}}
   \newcommand\ttlFun{\res{fun}}
   \newcommand\ttlConstraint{\res{constraint}}
   \newcommand\ttlConst{\res{const}}
   \newcommand\ttlRequire{\res{require}}
   \newcommand\ttlPrefer{\res{prefer}}
   \newcommand\ttlRetract{\res{retract}}
   \newcommand\ttlEnd{\res{end}}
   \newcommand\ttlModule{\res{module}}
   \newcommand\ttlImport{\res{import}}
   \newcommand\ttlExport{\res{export}}
   \newcommand\ttlOr{\res{or}}
   \newcommand\ttlAnd{\res{and}}
   \newcommand\ttlNot{\res{not}}
   \newcommand\ttlArray{\res{array}}
   \newcommand\ttlOf{\res{of}}
   \newcommand\ttlList{\res{list}}
   \newcommand\ttlType{\res{type}}
   \newcommand\ttlVar{\res{var}}
   \newcommand\ttlWhile{\res{while}}
   \newcommand\ttlDo{\res{do}}
   \newcommand\ttlIf{\res{if}}
   \newcommand\ttlIn{\res{in}}
   \newcommand\ttlThen{\res{then}}
   \newcommand\ttlElse{\res{else}}
   \newcommand\ttlElsif{\res{elsif}}
   \newcommand\ttlDatatype{\res{datatype}}
   \newcommand\ttlReturn{\res{return}}
   \newcommand\ttlString{\res{string}}
   \newcommand\ttlHd{\res{hd}}
   \newcommand\ttlTl{\res{tl}}
   \newcommand\ttlNull{\res{null}}
   \newcommand\ttlTrue{\res{true}}
   \newcommand\ttlFalse{\res{false}}
   \newcommand\ttlPublic{\res{public}}
   \noindent
   \begin{tabbing}
   \qquad \= \quad \= \quad \= \quad \= \quad \= \quad \= \quad \= \quad \= \quad \= \quad \= \quad \quad \quad \quad \quad \= \kill}
  {\end{tabbing}}

\renewcommand{\MakeUppercase}[1]{#1}
% \renewcommand{\chaptermark}[1]%
%              {\markboth{#1}{}}
% \renewcommand{\sectionmark}[1]%
%              {\markright{\thesection\ #1}}
% \lhead[\fancyplain{}{\thepage}]%
%        {\fancyplain{}{\rightmark}}
% \rhead[\fancyplain{}{\leftmark}]%
%       {\fancyplain{}{\thepage}}
% \cfoot{}

% \pagestyle{fancyplain}

%\addtolength{\parskip}{0.4ex}
%\parindent0pt

\title{Diplomarbeit\\
Constraint-imperative Programmierung}

\author{Martin Grabm\"uller\\
\normalsize{\em mgrabmue@cs.tu-berlin.de}}

\begin{document}

\selectlanguage{german}

%% cover-text.tex -- German cover and summary for the Diplomarbeit.
%%
%% Copyright (C) 2003 Martin Grabmueller <mgrabmue@cs.tu-berlin.de>

\def\germantoday{\number\day. \ifcase\month\or
  Januar\or Februar\or M\"arz\or April\or Mai\or Juni\or
  Juli\or August\or September\or Oktober\or November\or Dezember\fi
  \space\number\year}

\thispagestyle{empty}

\noindent
Technische Universit\"at Berlin\\
Fakult\"at IV (Elektrotechnik und Informatik)\\
Institut f\"ur Softwaretechnik und Theoretische Informatik\\
Fachgebiet \"Ubersetzerbau und Programmiersprachen\\
Franklinstr. 28/29\\
10587 Berlin

\vskip3cm

\begin{center}
{\Large 

% \fbox{\vbox{{\Huge \sc Vorl\"aufige Version}\\
% {\sc vom \today}}}

\vskip1cm

Diplomarbeit

\vskip1cm

{\huge \sf Constraint-imperative Programmierung}

\vskip1cm

Martin Grabm\"uller

%\vskip0.01cm

Matr-Nr. 183283

\vskip0.5cm

28. Februar 2003

\vskip2cm

\begin{tabular}{ll}
Pr\"ufer: &Prof. Dr. Peter Pepper\\
Betreuerin: &Dr. Petra Hofstedt
\end{tabular}
}

\end{center}

\cleardoublepage

{}
% \begin{center}
% {\large \bfseries Eidesstattliche Versicherung}
% \end{center}

% \vskip0.5cm

% \noindent
% Die selbst\"andige und eigenh\"andige Anfertigung versichere ich an
% Eides statt.

% \vskip1cm

% \noindent
% Berlin, den 28. Februar 2003

% \vskip2cm

% \noindent
% \rule{7cm}{0.5pt}

% \noindent
% Martin Grabm\"uller

\begin{tabbing}
Load and store instructions\=\kill\\
\end{tabbing}
\cleardoublepage

\chapter*{Aufgabenstellung}
%% thema.tex -- Aufgabenstellung.
%%

%\begin{center}{\bf \Large AUFGABENSTELLUNG F\"UR DIE DIPLOMARBEIT}\end{center}

%\vspace*{5mm}

\hrule

\vspace*{2mm}

\noindent Fachgebiet \"Ubersetzerbau und Programmiersprachen 

\vspace*{2mm}

\hrule

\vspace*{2mm}

\noindent Diplomand: Martin Grabm\"uller (Matr-Nr. 183283)

\vspace*{2mm}

\hrule

\vspace*{2mm}

\noindent Betreuer/Innen: Peter Pepper, Petra Hofstedt

\vspace*{2mm}

\hrule

\vspace*{4mm}

%----------------------------------------------------------------------------

\noindent {\bf \underline{Thema:}} 
\hspace*{5mm} 
\begin{minipage}[t]{0.85\linewidth}
Constraint-imperative Programmierung
\end{minipage} 
 
%\vspace{3mm}
\section*{CIP -- Constraint-imperative Programmierung}

W\"ahrend in imperativen Programmiersprachen Algorithmen in
Anweisungen zerlegt werden, die angeben, {\em wie} durch die Modifikation
des Programmzustands die L"osung eines Problems berechnet wird, 
zeichnen sich deklarative Programmiersprachen dadurch aus,
dass spezifiziert wird, {\em was} berechnet wird.  

Eine der j"ungsten Entwicklungen der deklarativen Programmiersprachen
ist das {\em Constraint Programming}, das sich aus der logischen
Programmierung entwickelt hat und mittlerweile als eine Obermenge
dieser betrachtet wird.  Constraint-Programme bestehen aus den
Spezifikationen der Programmvariablen, der Wertebereiche dieser
Variablen und der Bedingungen, die f"ur die Variablen gelten m"ussen.
Da hier die Eigenschaften einer {\em L"osung} des Problems beschrieben
werden, aber nicht ein oder der {\em L"osungsweg}, ist es die Aufgabe
des Compilers und der Laufzeitumgebung, einen solchen Weg zu ermitteln
und L"osungen zu berechnen.

Constraint-Sprachen sind inzwischen um Konzepte verschiedener anderer 
-- meist deklarativer -- Programmiersprachen erweitert worden.
%
Wenig untersucht ist dagegen die Kombination aus imperativen und
Constraint-Sprachen.  Borning und Freeman-Benson
\cite{benson92int,benson91cip} pr"agten den Begriff der {\em
  constraint-imperativen Programmierung} \cite{CIP1,CIP2} und
entwickelten die Sprache Kaleidoscope, die objektorientierte und
Constraint-Konzepte in eine Sprache integriert.

Im folgenden soll der Begriff der constraint-imperativen
Programmiersprachen f"ur alle Sprachen verwendet werden, die sowohl
Constraint-Programmierung als auch imperative Programmierung
unterst"utzen.  Objektorientierung soll dabei keine Voraussetzung
sein.  Stattdessen interessiert uns vor allem das Zusammenspiel
zwischen Constraints und Constraintl"osern auf der einen und
imperativen Sprachkonstrukten auf der anderen Seite.

Die Kombination dieser beiden Programmierparadigmen bietet eine Reihe
von Vorteilen: Imperative Programmierung ist sehr verbreitet und wird
von vielen Programmierern verstanden und als nat"urlich angesehen.
Weiterhin ist die imperative Programmierung gut untersucht und es gibt
verbreitete und effizient implementierte Programmiersysteme.  
%
Viele Problemstellungen, die von Natur aus imperativer Art sind
(Steuerungs- und Regelsysteme, interaktive Systeme), lassen sich sehr
leicht imperativ l"osen.  
%
Auf der anderen Seite lassen sich komplexe Algorithmen nur sehr
umst"andlich auf dem niedrigen Abstraktionsniveau imperativer Sprachen
formulieren und f"uhren so zu schwer verst"andlichen Programmen.
Genau bei diesen Problemen haben sich deklarative (funktionale,
logische und constraintbasierte) Programmiersprachen bew"ahrt.  Beim
Constraint Programming werden nur die Beziehungen formuliert, die
zwischen den betrachteten Objekten bestehen, nicht aber die
Algorithmen, die diese Beziehungen aufrechterhalten.  Auf diese Weise
bleiben die formulierten Programme nahe an der Spezifikation, was
nicht nur die Verst"andlichkeit, sondern auch die Korrektheit
erh"oht.

Die constraint-imperative Programmierung versucht, die Vorteile dieser
beiden Paradigmen zu verbinden, um so intuitive, verst"andliche und
effiziente Programme zu erm"oglichen.

\section*{Aufgabenstellung}

In der angestrebten Diplomarbeit soll eine constraint-imperative 
Programmiersprache entwickelt werden.

Dazu sind zun\"achst existierende Ans\"atze, wie die
Programmiersprachen Alma-0 \cite{apt97search,apt98alma} und
Kaleidoscope \cite{lopez94kaleidoscope,lopez94implementing},
Erweiterungen objektorientierter Sprachen, wie COOL
\cite{avesani90cool}, und Bibliotheken, die Constraintl"oser
nachtr"aglich in bestehende Programmiersprachen einbetten, wie
Smalltalk \cite{pachet95mixing}, Java
\cite{abdennadher01:jack:inp,Jack2} und der ILOG-Solver \cite{ILOG} zu
analysieren und zu ber\"ucksichtigen.

Weiterhin ist der Einsatz von Constrainthierarchien, in denen
erzwungene (required) und gew"unschte (preferential) Constraints
unterschieden werden
\cite{borning92constrainthierarchies,bartakHierarchies1}, zu
diskutieren.  Constrainthierarchien erlauben es vor"ubergehend oder
l"anger andauernd mit widerspr"uchlichen Constraints umzugehen, was in
einer constraint-imperativen Sprache von Vorteil sein k\"onnte.

In der zu entwickelnden Sprache sollen die "ublichen Steueranweisungen und
Datenstrukturen imperativer Sprachen enthalten sein, 
ebenso wie Konstrukte, die die Constraint-Programmierung unterst"utzen.

Die Sprache soll gro"s genug sein, um mit con\-straint-im\-perativen
Programmen experimentieren zu k"onnen, sich allerdings auf die
imperativen Kernkonstrukte beschr"anken. 
Um die Sprache klein zu halten, soll Objektorientierung nicht implementiert werden.
%
Die Programmiersprache ist zu entwerfen, zu definieren und
prototypisch zu implementieren.

Die Arbeit muss sowohl theoretische als auch praktische Gesichtspunkte 
ber\"ucksichtigen. Es sind bei nachfolgenden Teilaufgaben jeweils 
%
eine kurze Einf\"uhrung in das entsprechende Thema zu geben, 
notwendige Begriffe zu definieren und an Beispielen zu illustrieren.
%
Entwurfsentscheidungen sind zu begr\"unden und Alternativen zu diskutieren. 

\vskip2ex
%\newpage

\noindent {\underline{Teilaufgaben:}}

\begin{itemize}
  
\item Vorstellung und Einordnung des Themas, Bezug zu existierenden
  Ans\"atzen herstellen.
 
\item Sprachentwurf:

  \begin{itemize}
  \item Auswahl der notwendigen/gew\"unschten Konstrukte der zu
    integrierenden Paradigmen,
    
  \item Demonstration ihres Zusammenspiels an Beispielen und
    
  \item Angabe von Syntax und Semantik (in angemessenem Umfang).

  \end{itemize}
 
\item Die Implementierung eines Compilers f\"ur die Sprache soll
  lediglich prototypisch erfolgen.
      
\item Sp\"atere Erweiterungen der Sprache und ihrer Implementierung
  sollen diskutiert werden.

\end{itemize}

%\bibliography{biblio.thema}

%---------------------------------------------------------------------------

%%% Local Variables: 
%%% mode: latex
%%% TeX-master: "da.tex"
%%% End: 

%% End of cover-text.tex.


\cleardoublepage


\chapter*{Zusammenfassung}

In Computeranwendungen werden Objekte der realen Welt vereinfacht auf
Modelle abgebildet, um diese dann durch Datenstrukturen zu
repr\"asentieren und mittels geeigneter Algorithmen zu verarbeiten.
Auf diese Weise k\"onnen die Ver\"anderungen der Wirklichkeit
modelliert werden.  Die Beziehungen zwischen den einzelnen
Datenobjekten sind ein wichtiger Bestandteil des Modells, und es ist
w\"unschenswert, diese Beziehungen pr\"azise zu spe\-zi\-fi\-zie\-ren.
Da einzelne Objekte und die Beziehungen zwischen Objekten im Laufe
eines Programms ver\"andert werden (z.B. in interaktiven Anwendungen
oder Simulationen), ist es notwendig, laufend sicherzustellen, ob
durch eine Zustands\"anderung die Spezifikation verletzt wurde und
gegebenenfalls einen konsistenten Zustand wiederherzustellen.
Weiterhin muss es auch m\"oglich sein, neu erzeugte Objekte und deren
Beziehungen verarbeiten bzw. bestehende Beziehungen erweitern zu
k\"onnen.  Genau diese Objektbeziehungen sind es, die in der
Constraint-Programmierung nicht nur genutzt werden, um die Konsistenz
einer Problembeschreibung zu wahren, sondern sogar, um die L\"osungen
f\"ur das Problems zu berechnen.

Constraints sind Bedingungen im mathematischen Sinne und eignen sich
hervorragend, um Einschr\"ankungen von Werten und Beziehungen zwischen
Objekten zu spezifizieren, w\"ahrend Constraint-L\"oser als
Programmkomponenten geeignet sind, verletzte Einschr\"ankungen zu
erkennen und ggf. durch Anpassungen der beteiligten Objekte wieder zu
erf\"ullen.  Diese Eigenschaften lassen es sehr erstrebenswert
erscheinen, Programme unter der Verwendung von Constraints zu
entwickeln und die Vorteile der deklarativen Programmierung zu nutzen:
\"Ubersichtlichkeit, mathematische Fundierung, Effizienz und effektive
Programmentwicklung.  Da allerdings ein schneller Umstieg von
bestehenden Programmiergewohnheiten auf neue, abstraktere
Vorgehensweisen in der Breite nur schwer umzusetzen ist, empfiehlt es
sich, existierende Programmiersprachen bzw. Programmierstile zu
erweitern.  Auf diese Weise lassen sich bestehendes Wissen und
bestehende Fertigkeiten weiter nutzen, w\"ahrend erweiterte Verfahren
auf abstrakterer Ebene eine Steigerung der Korrektheit und
(Programmier-)Effizienz erm\"oglichen.  Dies ist das Ziel der
Entwicklung so genannter Multiparadigma-Programmiersprachen, die
wesentliche Elemente verschiedener Programmierstile in einer einzigen
Sprache kombinieren, um deren Vorteile gemeinsam nutzen zu k\"onnen.

\vskip1ex

Diese Arbeit befasst sich mit der Verschmelzung zweier sehr
unterschiedlicher Programmierparadigmen in einer Programmiersprache.
Zum einen betrachten wir die deklarative Programmierung mit
Constraints, also die Spezifizierung der Eigenschaften, die zwischen
den Objekten eines Programms gelten m\"ussen, zum anderen die
imperative Programmierung, die auf Zustandsver\"anderungen durch
Anweisungen basiert.  Auf der einen Seite steht also die
Spezifikation, {\em was}\/ berechnet werden soll, auf der anderen
Seite der Algorithmus, {\em wie}\/ dies geschehen soll.  Die Frage,
wie diese zun\"achst gegens\"atzlich scheinenden Konzepte miteinander
verkn\"upft werden k\"onnen und unter welchen Voraussetzungen dieses
Vorgehen \"uberhaupt sinnvoll ist, soll der Schwerpunkt der Arbeit
sein.

\vskip1ex

Nach einer Einf\"uhrung in das Thema und das Problemumfeld in
Kapitel~\ref{cha:introduction} beschreibt die Arbeit in
Kapitel~\ref{cha:constraint-programming} zun\"achst die
constraint-basierte Programmierung.  Dabei werden grundlegende
Begriffe der Constraint-Programmierung, wie z.B.  Constraints und
Constraint-L\"oser, erkl\"art sowie verschiedene Ans\"atze der
Implementierung constraint-basierter Systeme beschrieben.  Weiterhin
werden existierende Arbeiten zu Verbindungen von Constraint-Sprachen
mit anderen (vor allem deklarativen) Programmiersprachen untersucht.

Eine besondere Verbindung zweier Programmierparadigmen, n\"amlich die
constraint-imperative Programmierung, wird in
Kapitel~\ref{cha:constraint-imperative-programming} beschrieben.  Da
dies der Schwerpunkt dieser Arbeit ist, sollen bisherige Arbeiten auf
diesem Gebiet ausf\"uhrlicher behandelt werden und charakteristische
Merkmale und grundlegende Elemente dieser Programmiersprachen
herausgearbeitet werden.  Dabei handelt es sich einerseits um
vollwertige Programmiersprachen, die eine vollst\"andige Integration
von constraint-basierter und imperativer Programmierung beabsichtigen,
andererseits um so genannte Constraint-Solving Toolkits, das sind
Programmbibliotheken, die Constraint-L\"oser in Form
wiederverwendbarer Module an imperative Sprachen anbinden. Neben einer
Darstellung der Gemeinsamkeiten dieser Sprachen und Bibliotheken
werden Vor- und Nachteile der einzelnen Systementw\"urfe beleuchtet.

Nach ausf\"uhrlicher Darstellung der Literatur auf diesem Gebiet
werden in Kapitel~\ref{cha:turtle} Ver\-bess\-er\-ungs- und
Erg\"anzungsvorschl\"age als Entwurf der neuen
con\-straint-im\-per\-a\-ti\-ven Programmiersprache \turtle{}
dargestellt.  Neben der constraint-basierten und imperativen
Programmierung haben dar\"uberhinaus einige Konzepte der funktionalen
Programmierung wie algebraische Datentypen, Funktionen h\"oherer
Ordnung und Polymorphie Einfluss auf den Sprachentwurf genommen.
Zun\"achst werden die wesentlichen Sprachkonstrukte benannt, die eine
imperative Programmiersprache zu einer constraint-imperativen Sprache
erweitern, dann werden nach einer kurzen Einf\"uhrung in Syntax und
Semantik von \turtle{} besondere Eigenschaften ausf\"uhrlicher
dargestellt.  Von besonderem Interesse ist dabei nat\"urlich die
Integration von Constraints und funktionalen Konzepten in eine
ansonsten herk\"ommliche imperative Sprache.  Abgeschlossen wird das
Kapitel durch die formale Darstellung der Sprachsemantik in Form einer
operationalen Semantik, die auf der \"Ubersetzung der Quellsprache in
die Sprache einer abstrakten Maschine und der Beschreibung der
Programmausf\"uhrung auf dieser Maschine aufbaut.

Die Ausarbeitung dieser Diplomarbeit umfasst nicht nur den
Sprachentwurf einer con\-straint-im\-pe\-ra\-ti\-ven Sprache, f\"ur
die Sprache wurde auch ein Programmsystem bestehend aus Compiler,
Laufzeitsystem, einer Bibliothek wichtiger Module und zwei
Constraint-L\"osern implementiert.  Diese Implementierung wird in
Kapitel~\ref{cha:turtle-impl} beschrieben.  Dabei werden einige
Aspekte des Kompilierungsvorgangs herausgegriffen, die aufgrund der
Integration mehrerer unterschiedlicher Sprachparadigmen \"uber eine
Compilerbeschreibung in der allgemeinen
\"Uber\-setz\-erbau\-li\-te\-ra\-tur hinausgehen.  Abgeschlossen wird
das Kapitel durch experimentelle Vergleiche der Leistung des
entwickelten Systems mit einer traditionellen imperativen
Programmiersprache.

Das letzte Kapitel fasst die Ergebnisse der Arbeit zusammen und
bewertet die Resultate.  Weiterhin werden alternative Entw\"urfe und
Implementierungsaspekte diskutiert und die \"Uberlegungen dargestellt,
die zur Entscheidung gegen deren Verwendung gef\"uhrt haben.
Anschlie\ss{}end werden zuk\"unftige Arbeiten vorgeschlagen, die
entweder aus Zeitgr\"unden nicht in diese Arbeit eingeflossen sind
oder nicht mehr der Aufgabenstellung der Arbeit zuzuordnen sind, aber
dennoch interessante nahegelegene Themengebiete betreffen.

Der Anhang umfasst die formale Grammatik der entwickelten
Programmiersprache, den Quelltext dreier Beispielmodule zur
Illustration, eine \"Ubersicht \"uber die im Rahmen der
Implementierung entwickelten Bibliotheksmodule sowie Bezugs- und
Installationshinweise f\"ur das entwickelte Programmiersystem.

%%% Local Variables: 
%%% mode: latex
%%% TeX-master: "da.tex"
%%% End: 

%% End of cover-text.tex.


\cleardoublepage

%%% This is the English title page.
%%%

\pagenumbering{arabic}

\thispagestyle{empty}

\selectlanguage{english}

\noindent
Technische Universit\"at Berlin\\
Fakult\"at IV (Elektrotechnik und Informatik)\\
Institut f\"ur Softwaretechnik und Theoretische Informatik\\
Fachgebiet \"Ubersetzerbau und Programmiersprachen\\
Franklinstr. 28/29\\
10587 Berlin

\vskip3cm

\begin{center}
{\Large Diploma Thesis

\vskip1cm

{\huge \sf Constraint Imperative Programming}

\vskip1cm

Martin Grabm\"uller

\vskip0.5cm

February 28, 2003

\vskip2cm

Advisors: Prof. Dr. Peter Pepper and Dr. Petra Hofstedt
}

\end{center}

\cleardoublepage

\begin{center}{\large Abstract}\end{center}
\begin{quote}
  Constraint-based programming languages are declarative programming
  languages.  In constraint programs, the solutions of a problem are
  obtained by specifying their desired properties, whereas in
  imperative programs, the steps which lead to a solution must be
  defined explicitly, rather than being derived automatically from the
  specification.  This work deals with the design and implementation
  of a programming language which integrates declarative constraints
  and imperative constructs in order to form a powerful programming
  paradigm suitable for solving a wide range of problems.
\end{quote}

\cleardoublepage

\tableofcontents

\listoffigures

\listoftables

\listof{Program}{List of Programs}

\cleardoublepage


%% introduction.tex -- Introduction to the Diplomarbeit.
%%
%% Copyright (C) 2003 Martin Grabmueller <mgrabmue@cs.tu-berlin.de>

\chapter{Introduction}
\label{cha:introduction}

\section{Motivation}
\label{sec:motivation}

Problems are solved by computer programs by first developing
simplified models of the real world which capture the objects and
relations of interest.  These models must be represented inside the
computer in a form suitable for machine processing.  The computer can
then manipulate the models' representations to create one or more
solutions and these results must be interpreted in the context of the
original problem.  The task of the programmer consists of the
necessary modelling and the translation of the model into a language
the computer can work with.  Programming languages should help the
programmer with this task, so that the data models and algorithms for
finding solutions can be expressed correctly and conveniently, and
that the computer can find them as fast as possible.

The desire to have both correct and efficient computer programs often
results in two conflicting goals.  Often the first wish is addressed
by using high-level programming languages, which offer a clear syntax
and well-defined semantics, so that the correctness of a program can
be proved using mathematical properties of the semantics.
Additionally, modern high-level languages provide features which are
convenient for modelling and handling data structures, such as
algebraic data types\index{algebraic data type}, pattern
matching\index{pattern matching}, higher-order
functions\index{higher-order function}, unification\index{unification}
and built-in search\index{built-in seach}\index{search} facilities.
Unfortunately, today's implementations of these high-level languages
are often still slower than their lower-level counterparts, the
imperative programming languages.  Many programmers still neglect
higher-level languages as being too slow for their purposes, despite
the incredible advances in compiler technology\footnote{Of course, the
  advances in hardware are important as well, because they change the
  sets of programs which are considered ``slow'' or ``fast''.}, which
let optimizing compilers produce machine code which performs nearly as
efficient as hand-written imperative code. Even though the difference
in performance between low- and high-level languages is insignificant
for most programming problems, many programmers are still using
imperative languages.  There are several reasons for that:

\begin{description}
\item[Performance.\index{performance}] For some tasks, higher-level
  language implementations still do not have the necessary
  performance, e.g.~for real-time systems\index{real-time system}
  where guaranteed response times and low memory usage are often the
  most important requirements.  Higher-level languages normally need a
  large run-time system\index{run-time system} and garbage
  collection\index{garbage collection}, which are hard to be made
  suitable for real-time systems.
  
\item[Development Environments.\index{development environment}] There
  are very good development environments for imperative languages,
  which are convenient to use and generate very efficient programs,
  for instance the integrated development
  environments\index{integrated development environment} from
  companies like Microsoft\index{Microsoft}, Borland\index{Borland},
  IBM\index{IBM} or Sun\index{Sun} for programming
  \cplusplus{}\index{C++} or \java{}.\index{Java} These environments
  contain very good debuggers\index{debugger}, and the elimination of
  program errors is often thought of as the important last phase
  before delivering software, and the need to avoid such errors early
  in the development process is not considered important.
  
\item[Education.]  Imperative programming languages are often the
  first languages which are taught at school, in university and in
  programming courses.  Since most companies look for imperative
  programmers and not for functional or logic programmers, there is
  some pressure on the educators to teach these languages, even if
  they know about the advantages of higher-level languages.

\item[Availability.] Higher-level language implementations are not as
  widely available as imperative ones, at least from commercial
  vendors.  A lot of software companies do not use them at all.
  
\item[Legacy\index{legacy} Software.]  Most of old but still important
  software is written in imperative languages like
  \cobol{}\index{Cobol}, \fortran{}\index{Fortran} or
  \cee{}\index{C}/\cplusplus{}\index{C++} and still needs to be
  maintained and extended.
  
\item[Tools and Libraries.] For older languages, there exist many well
  tested and efficient tools and libraries which help in programming.
  For newer systems, less libraries for addressing every-day problems
  in industry have been written---not because it is not possible, but
  simply because there has not been enough time and personnel to do
  it.
  
\item[Literature.]  A lot of literature is dedicated to imperative
  languages.  Again, this is mostly caused by the fact that there is
  not yet such a wide audience for books on higher-level languages as
  there is for imperative languages.
\end{description}

\noindent
Despite the wide use of imperative languages, it is consensus that
high-level languages decrease the number of errors in software systems
and lead to faster development.

So what is necessary to make programmers use high-level languages?

Most of the problems mentioned above have already been addressed by
research in these fields and the development of hardware and tools.
Compilers for high-level languages produce acceptable code.  Memory
needs of higher-level languages are normally larger than for
imperative languages, but today every personal computer is equipped
with hundreds of megabytes of memory.  The mainstream and the research
results are approaching the same requirements of hardware as
functional and logic language compilers become better and imperative
language implementations become slower, due to widespread use of byte
codes%
\index{byte code} and virtual machines%
\index{virtual machine} to obtain portability, which seems to be more
important nowadays than pure number-crunching performance.  At the
same time, as high-level language implementations grow more mature,
they gain more interest in the programmer community, which leads to
more tools and libraries (often freely available) and encourages
publishers to publish books on this topic.  That is why the author
believes that most of the problems listed above will be simply solved
by the time which is needed to let ideas spread from academia to the
mainstream, as could be witnessed with the success of object-oriented%
\index{object-oriented programming} programming in recent years, which
took more than twenty years until it was widely used.

Unfortunately, one problem has not been mentioned yet:

\begin{description}
\item[Habits.]  Most programmers are used to their favourite
  programming language, which most probably is an imperative one.  And
  habits do not change easily.
\end{description}

\noindent
The idea for solving this problem is to combine technical measures
with ``psychological'' ones, that is, to design a programming language
which lets the programmer get used to declarative programming
incrementally.  That is why a language should have both low-level
concepts with predictable and guaranteed time and space requirements
as well as high-level concepts such as higher-order functions%
\index{higher-order function} for programming more abstractly. Then
the programmer can choose whatever language feature is suitable for a
given task.

Such a strategy in language design leads to multiparadigm programming
languages%
\index{multiparadigm!programming language}%
\index{multiparadigm}, as described by Budd%
\index{Budd}, Justice%
\index{Justice} and Pandey%
\index{Pandey} in~\cite{buddGeneral}.  The idea is to offer the best
in existing programming paradigms and let the user decide which
language features to use when.  The language Leda%
\index{Leda}, for example, designed by Budd%
\index{Budd} and described in~\cite{budd94leda, budd95mppil}, combines
imperative and relational programming in one language.

The language \turtle{}%
\index{Turtle} developed in this thesis merges imperative and
constraint programming\footnote{Actually, functional language elements
  are also added, as far as they fit into the constraint imperative
  framework.}  into one language, resulting in a powerful language
with a wide area of applicability.

Constraints are relations in the mathematical sense, which are used in
constraint-based programming languages for specifying the relations
which hold between different entities of interest.  Since objects and
their relations change over time (for example in interactive
applications or simulations), it is necessary to check continuously
whether these state transitions violate the specification of the
model.  The specification must then be restored by modifying the
objects or their relations.  Additionally, the set of relations must
be capable of changing over time, as well as the set of objects.

This work investigates the integration of two very different
programming paradigms.  On the one hand declarative programming with
constraints, the specification of the properties of objects and their
relations, and on the other hand imperative programming, which deals
explicitly with statements and the change of the program state over
time.  So we have a programming style in which we tell the machine
{\em what} to calculate, and a style in which we tell {\em how} to do
it.  The main goal of this work is to find out how these different
styles can be integrated smoothly, so that we can obtain ``the optimum
of both worlds.''


\section{Background}
\label{sec:background}

This work was motivated mainly by two existing language designs which
integrate imperative and declarative programming.  The first is the
language Alma-0%
\index{Alma-0}, developed by Apt%
\index{Apt} et al.~\cite{apt97search, apt98alma, apt98almaproject},
which is an intermediate result of the Alma%
\index{Alma} project.  This project aims to lead to the design of a
constraint imperative language.  While Alma-0 is on its way to a
constraint imperative language, its designers decided to proceed
step by step by starting with an imperative language, then adding
declarative language elements such as logical variables, the notion of
success/failure and backtracking to their language and then, at later
stages, to integrate constraints.  Alma-0 is based on Modula-2, both
in syntax and semantics.

Borning%
\index{Borning} et
al.~\cite{lopez94kaleidoscope,freemanbenson90design} developed a
family of constraint imperative languages called Kaleidoscope%
\index{Kaleidoscope}.  The first version, Kaleidoscope '90 was
designed to be a constraint imperative language from the beginning.
The language is object-oriented and allows to place constraints on
primitive data types as well as user-defined objects. Later versions
of the language changed in syntax and semantics to solve various
problems of the original design.

Both languages and other approaches are described in
Chapter~\ref{cha:constraint-imperative-programming}.

As already mentioned in section~\ref{sec:motivation}, Budd's works on
multiparadigm languages also influenced this thesis. General work on
constraint solving and satisfaction, constraint hierarchies,
constraint logic and constraint functional programming languages also
had an impact on the statements following.


\section{Goals}
\label{sec:goals}

First, the existing work in the field of constraint imperative
programming will be summarized, so that the following parts can build
on previous results of research in this area.

The next goal for this work is the design and implementation of a
programming language which is suitable for both imperative and
constraint programming.  The design is guided by the investigation of
previous approaches to this problem and related fields, such as
general programming language design, constraint logic programming and
other work on integrating programming language paradigms.  Besides the
design and the specification of the semantics of our integrated
approach, the implementation problems and possible solutions will be
addressed as well.

The final goal is to investigate how well the resulting language meets
the requirements for such a language, such as ease of programming,
easy adaptability for imperative programmers, reduced sources of
programming errors and reasonable efficiency.  This goal is addressed
by using the resulting language in implementing and examining test
programs.


\section{Outline}
\label{sec:outline}

This thesis is organized as follows: after this Introduction,
Chapter~\ref{cha:constraint-programming} gives a general view on
constraint programming.  We give definitions of the notations which
will be used in subsequent chapters, then we present some variations
of constraint problems and their solution techniques.  Finally the
existing approaches to integrate constraint programming with other
(mostly declarative) languages are summarized.

A special combination of constraints and traditional
programming---constraint im\-pe\-ra\-tive programming---is described
in Chapter~\ref{cha:constraint-imperative-programming}. Here the
origins as well as existing approaches and implementations are
presented and we will review the literature on constraint
im\-pe\-ra\-tive programming.  In this chapter, goals for a successful
merge of low- and high-level languages are defined, and possible
problems are identified.

Chapter~\ref{cha:turtle} describes the constraint imperative
programming language \turtle{}, which was designed to address the
problems arising in the integration of these two paradigms.  After
presenting the syntax we specify the semantics of constraint
imperative programs in \turtle{}.  Following the presentation of the
language design, the implementation of the compiler and the run-time
system will be described in Chapter~\ref{cha:turtle-impl} together
with a discussion of efficiency and effectiveness of constraint
imperative programs written in \turtle{}.

In Chapter~\ref{cha:summary} the thesis is summarized and compared to
previous work on the design and implementation of constraint languages
and constraint imperative languages.  Also, ideas for future
directions in the work on constraint imperative and multiparadigm
programming in general, and especially with regard to \turtle{} and
its implementations are outlined.

The Appendix contains a formal description of the grammar of \turtle,
a description of the modules available in the standard library and the
definitions of three \turtle{} modules for illustrating the syntax of
\turtle{} programs.  The Appendix also contains information about the
availability of the \turtle{} system which was developed as a part of
this thesis.


\section{Notation}

In the following chapters, some terms in the text and the example
programs will be emphasized by typesetting them differently.
%
\begin{itemize}
\item Reserved words are written in boldface, for example {\bf if},
  {\bf string} and {\bf module}.

\item Variables will be written in italics, e.g. {\em x} and {\em
    counter}.

\item Function and module names will be written in italics in the
  text, but in normal font in the example programs.
\end{itemize}
%
Since some type names are reserved names and some are not, the names
of data types will sometimes written in boldface and sometimes in
italics, e.g. {\bf string} and {\em bool}.

%%% Local Variables: 
%%% mode: latex
%%% TeX-master: "da.tex"
%%% End: 

%% End of introduction.tex.



%% constraint-programming.tex -- General introduction to CP.
%%
%% Copyright (C) 2003 Martin Grabmueller <mgrabmue@cs.tu-berlin.de>

\chapter{Constraint Programming}
\label{cha:constraint-programming}

Constraint imperative programming inherits important concepts and
techniques from constraint programming.  Therefore, some fundamental
notations need to be introduced before looking at more specialized
topics in the following text.  This chapter will introduce the most
important concepts and definitions, followed by a brief description of
several variants of constraint problems.  Finally, several paradigms
are presented which evolved from blending classic programming
languages with constraint programming.


\section{Introduction to Constraints}
\label{sec:introduction-to-constraints}

Informally, constraints are relations between variables and constants,
such as equations or inequalities.  For example,
$$X=1\quad\text{or}\quad Z\leq 2Y$$
are arithmetic constraints.  The
first one constrains the variable $X$ to be equal to 1, whereas the
second constrains the variables $Y$ and $Z$, so that $Z$ must be less
than or equal to $2Y$.  As can be seen in the second example,
constraints are not only useful for specifying a solution exactly, but
also for specifying incomplete knowledge, so that there may be an
infinite number of solutions to the problem.  Several constraints can
be combined for describing a problem, and the conjunction of these
constraints can then be used for calculating more precise solutions
than could be derived when investigating the constraints
independently.  An example for this are equation systems%
\index{equation system}, where several equations are examined at the
same time to determine the variable assignment (or assignments) which
makes the system consistent.

One of the advantages of using constraints in program development is
that the specification for a program often is given in the form of
(informal) constraints already.  When using a constraint programming
language, this specification can be translated into machine-executable
form without the need to change from the declarative view to a
procedural view as is needed for imperative programming languages.
The direct conversion of the program specification increases
correctness, because the semantic gap between specification and
implementation is narrower than for languages on a lower abstraction
level.  The same argument applies to all higher-level languages, but
especially for constraint programming, because its abstraction level
is higher than the level for other declarative languages.

Constraints are useful in many ways.  A constraint of the form $a+b=c$
can be used to calculate a value for $c$, but also (depending on other
constraints) to determine values for $a$ and $b$.  It is important to
notice, however, that this multi-directional property of constraints
does not necessarily lead to unique solutions in all cases as can be
illustrated by the constraint $|a|=b$, where $b$ can be calculated
precisely when $a$ is given, but $a$ can only be determined to be
either $b$ or $-b$.

Constraint programming basically consists of specifying constraints to
describe the properties of the solution desired and then searching
among all the solutions which satisfy the constraints.  Depending on
the problem, the programmer is interested in the first solution, all
solutions or a solution which is the best under certain criteria.
Except when only the first solution is needed, the process of
generating the solutions involves search, and the various constraint
solving techniques differ in how they try to minimize the search
space.

The various constraint programming systems can be classified by their
{\em domains}%
\index{domain}.  A constraint domain specifies the kind of constraints
which may be written, the meaning of these constraints, and the set of
values the constrained variables can take.  Another property by which
these systems can be distinguished is which constraints are
syntactically allowed as to make them efficiently solvable.  For
example, a lot of constraint systems allow linear equations, but not
non-linear equations, because there are more efficient solving
algorithms for the former than for the latter.

% \index{functional constraints}

% Another important property for classifying constraints is whether a
% constraint is {\em functional} or not.  A functional constraint can
% determine the values of its variables exactly, as for example equality
% constraints.  Other ({\em non-functional}) constraints, such as
% inequalities, only narrow the set of possible values for a variable.


\section[Constraints and Constraint Programming Languages]{Constraints and Constraint Programming\\Languages}

\index{constraint}
\index{constraint solver}
\index{definition!constraint}
\index{definition!constraint system}
\index{definition!constraint solver}

The existing literature on constraint programming provides a solid
theoretical foundation, both for syntax and semantics of constraint
systems and languages, see for
example~\cite{marriot98programmingwithconstraints,jaffar87clp}.
Therefore, this section only gives informal definitions for the most
important notations needed for understanding the rest of this work.


\subsection{Definitions}

A {\em constraint}%
\index{constraint} is a mathematical relation between one or more {\em
  variables}%
\index{variable} and constants.  A constraint can be an equation, an
inequality, a set inclusion%
\index{set inclusion} or another relation, depending on the
domain of the involved variables.  {\em Variables} are placeholders
for values and have an associated {\em domain}%
\index{domain}, which is the set of
values that can be assigned to the variable.  A {\em constraint
  conjunction}%
\index{constraint!conjunction}%
\index{conjunction} consists of a set of constraints which must all be
satisfied in order to satisfy the constraint conjunction.  A {\em
  valuation}%
\index{valuation} is a mapping from variables to values, also called a
{\em
  variable assignment}%
\index{variable assignment}%
\index{variable!assignment}.  A {\em solution}%
\index{solution} is a valuation of the variables of a constraint
system for which all constraints of the system are satisfied.

All constraints in a constraint program are added to a constraint
conjunction which is called {\em constraint store}%
\index{constraint store}%
\index{constraint!store}.  A {\em constraint
  solver}%
\index{constraint solver}%
\index{constraint!solver} is an algorithm which answers several
questions about a constraint system by maintaining a constraint store,
for example whether it is {\em satisfied}%
\index{satisfied}, whether a constraint is
already implied by the system ({\em entailment}%
\index{entailment}) or which values are
assigned to one or more variables of the system ({\em projection}%
\index{projection}).

Testing for {\em satisfyability}%
\index{satisfyability} is useful to direct the search among
several possible solutions or for conditionals.  {\em Entailment}%
\index{entailment} tells whether a given constraint is redundant with
respect to a constraint store.  Since the values of variables need not
be uniquely determined by the constraint store, {\em projecting} them
does not yield values in the general case, but constraints which
determine the variables.

Constraints containing one variable are called {\em unary
  constraints}%
\index{unary constraint}%
\index{constraint!unary}, constraints with two variables are called
{\em binary constraints}%
\index{binary constraint}%
\index{constraint!binary} and general constraints with more than two
variables are called {\em n-ary constraints}%
\index{n-ary constraint}%
\index{constraint!n-ary}.

% \index{derivation}

% In addition to the means by which constraint programs are written, we
% need to define how they are evaluated.  A {\em derivation} is a
% sequence of state transitions, leading from a start state to either a
% {\em successful} or {\em failing final state}.  The start state of a
% derivation represents the problem description and the final state
% represents the solution, the derivation describes the execution of a
% constraint program.

\subsection{Constraint Domains}

For illustrating the various constraint domains, a short summary of
some domains will be given.  This list is not exhaustive, but should
be sufficient to get a general impression.  Constraint domains are
described in detail in many works,
\cite{marriot98programmingwithconstraints} gives a good introduction.

{\em Finite domain}%
\index{finite domain} constraints are based on domains which are finite
sets%
\index{finite set} and lend themselves to several search strategies
which explore the search space of possible variable assignments.  The
domain can be a finite set of integers, but also sets of enumerable
types%
\index{enumerable} like colours as they are used for colouring maps,
or resources which should be planned for a process.  These constraints
are often used for solving combinatorial problems%
\index{combinatorial}, and much former as well as present research is
devoted to limit the time complexity needed for solving them, because
finite domain constraint solvers have exponential complexity in
general.  Constraint solving for finite domain constraints is also
called {\em constraint satisfaction}%
\index{constraint satisfaction}%
\index{constraint!satisfaction}%
\index{satisfaction}.

{\em Linear arithmetic}%
\index{linear arithmetic} constraints are linear equations and
inequalities.  For this domain efficient solving methods are known
which can also (in extended variations) be used for non-linear%
\index{non-linear}
(in-)equalities.  This is either accomplished by giving knowledge of
some non-linear constraints to the solvers or by delaying the
non-linear parts of the constraints until their variables are
determined by other constraints.  Then the solving process for the
non-linear parts is just a test whether the constraint is satisfied or
not.  This type of constraints is important for graphical applications
like computer aided design%
\index{computer aided design} (CAD%
\index{CAD}) systems or graphical user interfaces,
but also for optimization problems as in linear programming%
\index{linear programming}.

{\em Boolean}%
\index{boolean constraint}%
\index{constraint!boolean} constraints are a special case of finite
domain
constraints where the domain contains only two values, {\em true}%
\index{true} and
{\em false}%
\index{false}.  Even though solvers for this kind of constraints are
also exponential in principle, probabilistic solvers%
\index{probabilistic solver} with polynomial run time have been
developed.  They are incomplete, but nonetheless suitable for certain
problems.

{\em Tree}%
\index{tree constraint}%
\index{constraint!tree}%
\index{tree} constraints can be used for modelling data structures,
such as lists, records and trees, and for expressing algorithms on
these data structures.  Constraint logic programming languages are
based on tree constraints, often combined with one or more of the
other constraint domains.

{\em Set}%
\index{set constraint}%
\index{constraint!set} constraints deal with domains of sets and set
relations.  They have been used for examples for type inference%
\index{type inference}%
\index{inference} and
type checking%
\index{type checking} in compilers.  Aiken%
\index{Aiken}~\cite{aiken94setconstraints} gives a good survey of
applications and solving techniques for set constraints.


\subsection{Solving and Optimization Strategies}

\index{backtracking}

Especially when solving finite domain constraints, it may be necessary
to explore the entire search space by assigning all possible values
from the domains of all variables, and after each assignment testing
whether the constraints are satisfied, and then backtracking if they
are not.  Since this results in exponential time complexity, a lot of
strategies are used to reduce the number of variable manipulations.

One approach is to assign values to variables in an order which
minimizes the width of the search tree.  Another possibility is to
select the values to assign in such a way that inconsistencies are
detected early.  These two strategies can also be combined.

A lot of the optimization strategies are based on the assumption that
any inconsistency between the assignments of two or more variables
should be detected as early as possible.  For example, when selecting
a value for a variable, all constraints involving this variable are
checked to see whether the assignment causes any inconsistency.  If it
does, it is not necessary to try other values for any other variable,
and the solution algorithm can backtrack immediately.  In
Fig.~\ref{pic:binary-constraint} the graph representations%
\index{graph representation} for two constraints are shown.  This
graph representation is often used to explain how the various
consistency testing algorithms work.  The first constraint is a binary
equality constraint between two variables, $x = y$, and the second
shows the constraint $x+y=z$.  For the first constraint, whenever a
value is taken for $x$, it can be checked whether $y$ also has this
value, thus reducing the set of assignable values for the variable $x$
for this path in the search tree to the singleton set $\{y\}$.  In
general, for binary equality constraints, only the values in the
intersection of the domains of the two variables need to be examined.
The constraint $x+y=z$ is shown to illustrate how consistency can be
checked for constraints with more than two variables: whenever a value
is chosen for one of the variables, the domains for the other two
variables can be checked for consistency applying any of the rules
$z=x+y$, $x=z-y$ and $y=z-x$.

\begin{figure}[htp]
\begin{center}
\input{binary-constraint.epic}
\end{center}
\caption{Constraint graphs}
\label{pic:binary-constraint}
\end{figure}

For unary constraints, this consistency check is called {\em node consistency}%
\index{node consistency}%
\index{consistency!node}, for binary constraints {\em arc consistency}%
\index{arc consistency}%
\index{consistency!arc} and for general constraints {\em path
  consistency}%
\index{path consistency}%
\index{consistency!path}.  For a list of these consistency checking strategies
and more complicated ones, such as {\em forward checking}%
\index{forward checking},
see~\cite{marriot98programmingwithconstraints}.

Solvers for other constraint domains can be implemented by using the
constraints to guide the search instead of just checking all
combinations of variable assignments.  Linear arithmetic constraints
are often solved by using algorithms which take advantage of
mathematical properties of the constraints, and which can calculate
the solutions from the constraints instead of trying to find them with
exhaustive search.

Applying such optimization strategies whenever possible drastically
reduces the time and space requirements for solving constraint
problems.  This is the main reason why constraint programming is so
successful, even on large problems.

\subsection{Constraint Hierarchies}
\label{sec:constraint-hierarchies}

Constraint hierarchies%
\index{constraint hierarchies}%
\index{hierarchies}%
\index{constraint!hierarchies} are used for modelling constraint
problems with ``superfluous'' constraints.  Such problems are also
called {\em over-constrained}%
\index{over-constrained}, similar to {\em under-constrained}%
\index{under-constrained} problems which do not have enough
constraints to find a unique solution.

The idea is to assign a {\em strength}%
\index{strength}%
\index{constraint strength}%
\index{constraint!strength} to each constraint, and when the solver
detects an inconsistency, it leaves constraints with weaker strengths
unsatisfied in order to satisfy stronger constraints.  This can be
used to define several levels of preference%
\index{preference}, and only if absolutely necessary, some preferences
are not respected in order to get a solution instead of respecting all
preferences and failing.

In \cite{borning92constrainthierarchies} a theory of constraint
hierarchies has been formally defined.  Several solving algorithms for
constraint hierarchies have been developed, based on both local
propagation \cite{sannella92skyblue, borning98ultraviolet,
  borning95oti} and by formulating them as optimization problems and
then applying methods of linear programming%
\index{linear programming} \cite{badros02cassowary}.  Harvey%
\index{Harvey}, Stuckey%
\index{Stuckey}
and Borning%
\index{Borning}~\cite{harveyCompiling} developed an interesting
algorithm for compiling a subset of constraint hierarchies to
straight-line code which does not require a run-time constraint
solver, but can be executed directly.  The resulting code takes
assignments to some of the variables as its inputs and calculates
values for the remaining variables in a way that ensures that the
constraint hierarchy is consistent for the final variable assignment.

Further research in this field was done by Hosobe%
\index{Hosobe} et al.~\cite{hosobe94locally, hosobe96generalized}, who
generalize the theory of constraint hierarchies and also provide
algorithms for solving them, in their case with local propagation.

A collection of algorithms and solving strategies for constraint
hierarchies can be found in \cite{bartakHierarchies}.


\subsection{Concurrent Constraint Programming}
\label{sec:concurrent-constraint-programming}

Concurrent constraint programming%
\index{concurrent constraint programming} (CCP%
\index{CCP}) deals with the possibilities
of concurrent and parallel evaluation of constraints.  CCP programs
consist of sets of rules, and when executing constraint programs, one
of the applicable rules is executed arbitrarily.  This is in contrast
to other constraint languages, where all rules are evaluated one after
the other, if there is more than one possibility.  The non-determinism
in CCP programs is called ``don't care non-determinism''%
\index{don't care non-determinism}%
\index{non-determinism!don't care}, whereas the
other one is called ``don't know non-determinism''%
\index{don't know non-determinism}%
\index{non-determinism!don't know}, because the correct rule is not
known before all preceding rules have been evaluated.

Though this area could be quite interesting for constraint imperative
programming as well, we do not further discuss it in this work, since
the addition of concurrency is orthonogal to the issues considered
here.

For a definition of concurrent constraint programming, see Saraswat%
\index{Saraswat} \cite{saraswat93cc}.


% \section{General Constraint Literature}
% \label{sec:general-constraint-literature}

% Leler~\cite{leler87cp} writes about the specification of constraint
% languages.


\section{Constraint Language Implementations}
\label{sec:constraint-language-implementations}

The first system using constraints explicitly was the Sketchpad%
\index{Sketchpad} system by
I.~Sutherland%
\index{Sutherland} \cite{sutherland63sketchpad}.  It was not only the
first constraint system, but also the first system for interactively
manipulating graphical objects on computers.  Other early constraint
systems include a language for integer equations called Ref-Arf%
\index{Ref-Arf} by Fikes%
\index{Fikes}~\cite{fikes70refarf} and the Alice%
\index{Alice} system by
Lauriere%
\index{Lauriere}~\cite{lauriere78alice}, which is a solver for
combinatorial problems.

Constraint programs are already suited for formulating and solving a
lot of problems, but in practice, only few systems are in use which
provide constraints as the only means of program formulation.  These
are special-purpose systems, for example for graphical user interface
construction (such as
UltraViolet%
\index{UltraViolet}~\cite{borning95oti,borning98ultraviolet}) or
simulation (for example, ThingLab%
\index{ThingLab}~\cite{borning79thinglab} or
CONSTRAINTS%
\index{CONSTRAINTS}~\cite{sussman80constraints}).  Much more often it
is desirable to have access to functionality which can only be found
in other programming paradigms, for example when some parts of the
problem do not need the full power of constraints and more efficient
solution algorithms are known for these parts, or for tasks which
cannot easily be formulated in a declarative language, for example in-
and output.

Especially for interfacing with existing systems and applications, or
simply for interacting with the ``outside world'', the need for a
general-purpose programming language arises.

For this reason, constraint programming has been merged with languages
from all major programming paradigms.  To the user of such languages,
it is interesting how well they manage to keep the efficiency and
power of the different paradigms, so that their advantages are
preserved.  Some of the combinations fit very well together, mostly
because the components both have declarative semantics, whereas for
other paradigms, the semantic differences between the combined parts
are significant.  This leads to problems which need to be solved in
order to blend the paradigms in a useful way.  We will briefly survey
the existing approaches.


\subsection{Constraint Logic Programming}
\label{sec:constraint-logic-programming}

Logic programming%
\index{logic programming} languages like Prolog%
\index{Prolog} are the roots of constraint programming languages.  By
now they are even regarded as a subset of the constraint languages,
where the constraint domain is the set of Prolog terms.

The first definition of the name {\em constraint logic programming}%
\index{constraint logic programming}
was given by Jaffar%
\index{Jaffar} and Lassez%
\index{Lassez} \cite{jaffar87clp}, who use it as the name of a whole
class of programming languages.  They use the term
CLP(X)%
\index{CLP(X)}%
\index{CLP}, where X stands for the domain on which the constraints in
the language work.  The extension of logic programming with
constraints is theoretically elegant and practically straightforward.
Perhaps this is the reason why constraint logic programming systems
are the most widely used constraint language implementations today.
Jaffar%
\index{Jaffar} and Maher%
\index{Maher}~\cite{jaffar94clpsurvey} give a detailed survey of
constraint logic programming.  .

Not only have CLP languages been developed on the base of logic
programming languages, but classic logic languages have been extended
with constraints, for example Prolog~III%
\index{Prolog III}~\cite{colmerauer90prologIII}.  Other modern Prolog
systems have also been equipped with more or less advanced constraint
solving capabilities, such as the Ciao Prolog%
\index{Ciao Prolog} System~\cite{bueno02ciao} or GNU
Prolog%
\index{GNU Prolog}~\cite{gnuProlog}.


\subsection{Constraint Functional Programming}
\label{sec:constraint-functional-programming}

An example of {\em constraint functional programming}%
\index{constraint functional programming} languages is the language
\goffin{}%
\index{Goffin} \cite{chakravarty97goffin}, which combines higher-order
functions with concurrent constraint programming.

The research in this field is not as active as for constraint logic
languages, but recently the more general approach of combining all
three paradigms has been worked on \cite{Hofstedt_02B,
  KobayashiMarinIdaChe.WFLP02}.

Even though some aspects of functional programming will be discussed
in the context of constraint imperative languages, they are different
from those addressed by constraint functional programming.

\subsection{Constraint Imperative Programming}

The combination of constraint techniques and imperative (and
object-oriented) programming languages characterizes {\em constraint
  imperative programming}%
\index{constraint imperative programming}.  In constraint imperative
programs, constraints can be defined which relate object attributes
and/or program variables, and all language features known from
traditional imperative languages are also available.  In addition to
primitive constraints, which are implemented directly by the
integrated constraint solvers, the user can define higher-level
constraints which are mapped to primitive constraints by various
mechanisms.  In some approaches, imperative languages are also
extended with language elements from logic programming, such as
non-deterministic computations with logical variables and
backtracking.  The name ``constraint imperative programming'' was
first used by Borning%
\index{Borning} et
al.~\cite{lopez94kaleidoscope,freemanbenson90design}.  Apt%
\index{Apt} et al.~\cite{apt97search, apt98alma, apt98almaproject}
have worked on a ``imperative constraint language'', which also has
quite interesting concepts.  Both of these approaches and constraint
imperative programming in general are treated in detail in
Chapter~\ref{cha:constraint-imperative-programming}.


\section{Industrial Applications}

Constraint programming systems have been used in the ``real world'',
especially for tasks such as planning and scheduling, but also for the
optimal placing of mobile phone relay stations etc.  These and other
examples are given in~\cite{fruehwirthAnwendungen}
and~\cite{marriot98programmingwithconstraints}.

Another area where constraint techniques are successfully applied are
graphical user interfaces (GUI), where the positioning of graphical
objects on the computer screen is specified using constraints.  A
constraint solver built into the GUI toolkit manages the size and
position requirements of the individual objects.  An example of these
toolkits is described by Borning and
Freeman-Benson~\cite{borning95oti}.

% \cite{cras93review} presents reviews on the practical usage of
% constraint systems in the industry.


%%% Local Variables: 
%%% mode: latex
%%% TeX-master: "da.tex"
%%% End: 

%% End of constraint-programming.tex.



%% constraint-imperative.tex -- Introduction to CIP.
%%
%% Copyright (C) 2003 Martin Grabmueller <mgrabmue@cs.tu-berlin.de>

\chapter{Constraint Imperative Programming}
\label{cha:constraint-imperative-programming}
\index{constraint imperative programming}

Constraint imperative programming combines language elements of
traditional imperative programming languages with techniques from
constraint programming.

The advantages of constraint imperative programming will be introduced
by presenting an example adapted from literature \cite{lopez97phd,
  lopez94kaleidoscope}.  Fig.~\ref{pic:temperature} shows a
thermometer widget\footnote{widget = {\em wi}ndow ga{\em dget}, a
  graphical metaphor for human-machine interaction.}, whose ``mercury
column'' can be adjusted by dragging the mouse.  Input elements like
this thermometer are very common in graphical user interfaces (GUIs)
and lend themselves perfectly to demonstrating the advantages of
constraint support in interactive applications.  The upper end of the
column can be dragged up and down, but is of course restricted to stay
within the bounds of the scale.  Program~\ref{prog:imperative-example}
is a solution to this problem, presented in a language with typical
imperative features.  This example shows how closely connected the
program logic (reading the position of the mouse pointer and adjusting
the mercury column) and the maintenance of the program constraints
are.
%
\begin{figure}[h]
\begin{center}
\input{temperature.epic}
\end{center}
\caption{Thermometer widget}
\label{pic:temperature}
\end{figure}

%
\begin{Program}
\begin{ttlprog}
1\>\ttlWhile{} {\em mouse.pressed} \ttlDo{}\\
2\>\>\ttlVar{} {\em y}: int;\\
3\>\>{\em y} $\leftarrow$ {\em mouse.y};\`{\em read the position of the mouse pointer}\\
4\>\>\ttlIf{} {\em y} $>$ {\em scale.max} \ttlThen{}\`{\em restrict y-component to valid range}\\
5\>\>\>{\em y} $\leftarrow$ {\em scale.max};\\
6\>\>\ttlElse{}\\
7\>\>\>\ttlIf{} {\em y} $<$ {\em scale.min} \ttlThen{}\\
8\>\>\>\>{\em y} $\leftarrow$ {\em scale.min};\\
9\>\>\>\ttlEnd{};\\
10\>\>\ttlEnd{};\\
11\>\>{\em temp.max} $\leftarrow$ {\em y};\`{\em adjust the column}\\
12\>\ttlEnd{};
\end{ttlprog}
\caption{Imperative approach}
\label{prog:imperative-example}
\end{Program}
%% while mouse.pressed do
%%   var y: int;
%%   y := mouse.z;
%%   if y < scale.max then
%%     y := scale.max;
%%   else
%%     if y > scale.min then
%%       y := scale.min;
%%     end;
%%   end;
%%   temp.max := y;
%% end;
%
Program~\ref{prog:constrained-example} presents the constraint
imperative approach to this problem.  In this typical constraint
imperative language, the constraints are explicitly formulated and
their maintenance is delegated to the run-time constraint solver.
Since the constraints which limit the range of the column are
required, the equality constraint between the upper end of the mercury
column and the mouse pointer will only be enforced if the other two
constraints are not violated.
%
\begin{Program}
\begin{ttlprog}
1\>\ttlRequire{} {\em temp.max} $\leq$ {\em scale.max};\\
2\>\ttlRequire{} {\em temp.max} $\geq$ {\em scale.min};\\
3\>\ttlWhile{} {\em mouse.pressed} \ttlPrefer{} {\em temp.max} = {\em mouse.y};
\end{ttlprog}
\caption{Constraint imperative solution}
\label{prog:constrained-example}
\end{Program}
%% require temp.max <= scale.max;
%% require temp.max >= scale.min;
%% while mouse.pressed prefer temp.max = mouse.y;

The constraint imperative version is not only shorter and easier to
read and modify than the imperative one, but it is also easier to
check its correctness.  These are the main advantages of constraint
imperative programming.  This example shows how well constraints and
interactive, graphical applications interact, but as
Lopez~\cite{lopez97phd} noted, constraint imperative languages are
actually general-purpose languages, since they are a superset of
traditional programming languages.\footnote{Lopez writes about
  object-oriented programming languages, whereas this work is more
  targeted to traditional imperative languages without
  object-orientation.}

In this chapter, we will first review the existing literature about
constraint imperative programming.  Then we will identify the general
characteristics of constraint imperative programming languages and
summarize the main problems of this approach.

\section{Constraint Imperative Languages and Systems}

This section summarizes literature on constraint imperative
programming and on related topics like object-oriented programming
languages with constraint integration.  Constraint imperative systems
have been developed both as integrated programming languages where
constraints are added to some imperative base language, and as
programming libraries in various languages, which can be added to
existing programs without requiring the programmer to learn a new
language.

\subsection{\cool}
\index{COOL}
\index{Avesani}
\index{Perini}
\index{Ricci}
%
Avesani, Perini and Ricci~\cite{avesani90cool} describe the constraint
language \cool{} which is built on top of an object-oriented system.
Their goal was to develop an architecture for solving combinatorial
optimization problems.  In \cool{}, it is possible to create classes
and objects, and both of these can have slots.  Slots are used to
store values, and it is possible to declare slots as {\em
  constrainable}, what means constraints can be placed on slots, thus
constraining the values which can be stored into these slots.
Constraints are placed on slots by calling the built-in function {\em
  assert-constraint}.  The same function, called with different
parameters, is also used for removing constraints again. When all the
necessary constraints for modelling a problem have been placed on
slots, the built-in function {\em find} will calculate all solutions.
Interestingly, constraints can have an associated strength, similar to
the ones used in constraint hierarchies, but they are only used in
internal algorithms to the solver, and their semantics are not
precisely specified.

COOL is very tightly integrated with the underlying object-oriented
system and very specific to combinatorial problem solving.  It is not
well suited as a general purpose programming language.

\subsection{Alma-0: Embedded Search}
\index{Alma-0}
\index{Apt}

Alma-0 is the first language which was designed by Apt et al. for the
Alma project~\cite{apt98almaproject}.  The goal of this ongoing
project is the combination of imperative and declarative programming.
The designers decided to start with an imperative programming language
(\modula) and to add declarative features step by step.  First, only a
few language constructs for non-deterministic choice, succeeding and
failing statements and built-in backtracking were added to the
imperative base language, and the resulting language was called
Alma-0, showing that the language is only the first version of a
series of improving languages.

\index{trail}
%
With the embedded search capabilities, Alma-0 is already well suited
for solving many combinatorial problems and many illustrating examples
can be found in literature \cite{apt97search, apt98alma}.  The
implementation is that of traditional imperative languages with the
addition of a {\em trail} for storing variable assignments which might
need to be undone on backtracking.

The Alma project designers have written about the integration of
constraint into the language~\cite{apt98almaproject}, but there does
not exist a detailed description of these plans nor any implementation
yet.

The Alma-0 language could be called a logic imperative language,
rather than a constraint imperative language, because its semantics is
closer to logic programming languages than to constraint languages.
The planned extensions are more suited for constraint imperative
programming.  In~\cite{apt98almaproject}, the language is called an
``imperative constraint programming language'' and is introduced as
the natural extension of the concept of uninitialized variables as
they exist in Alma-0.  In Alma-0, variables can be uninitialized, and
when such a variable is used in an equality test, it gets initialized
with the other operand of the equality.  This concept will be
generalized for the planned constraint integration: variables will
then be initialized whenever they appear in constraint statements, of
which the equality test is a simple example.

\index{unknown}
\index{known, but varying}
%
In the planned language extension, variables are either ``unknowns''
or ``known, but varying''~\cite[p.~5]{apt98almaproject}.  The former
are variables in the mathematical sense, whereas the latter are
variables as in traditional imperative languages, where they stand for
storage locations which hold values.  Only ``unknowns'' can be
constrained in the proposed language.

The Alma project ``is a proposal for integrating constraints into ANY
imperative language''~\cite[p.~6]{apt98almaproject}, and some of the
proposed language elements will be used in the present thesis as well
(see Chapter~\ref{cha:turtle}).

\subsection{Kaleidoscope}
\index{Kaleidoscope}

\index{Lopez}
\index{Borning}
\index{Freeman-Benson}

Kaleidoscope is an object-oriented programming language which
integrates constraints as a fundamental concept.  It is described
in~\cite{lopez94kaleidoscope}, and the implementation is presented in
\cite{lopez94implementing}.  Benson et al. describe the principles
upon which it is based in~\cite{benson92int, benson91cip}.

\index{constraint!constructor}

Most of the imperative language constructs are modelled as
constraints, which are solved by a constraint solver integrated into
the run-time system.  The language provides primitive constraints, and
the user can define higher-level constraints which are then executed
in terms of the primitive ones by writing so-called {\em constraint
  constructors}.

A Kaleidoscope program consists of class definitions, which contain
attribute and method declarations.  In method bodies, all traditional
imperative statements like assignment, conditionals and loops can be
used.  Additionally, three different constraint statements are
available for placing constraints on object attributes.  The first one
enforces a constraint only once ({\em once}), the second for the rest
of the program execution ({\em always}) and the third as long as a
given condition remains true ({\em assert-during}).  The life-time of
constraints is also called {\em duration} (see
section~\ref{sec:constraints-in-an-imperative-environment}).

Constraints are tagged by {\em constraint strengths}, which indicate
the importance of the individual constraints.  Since Kaleidoscope
supports constraint hierarchies, weaker (less important) constraints
are not enforced when this is necessary to make stronger (more
important) constraints satisfiable.  For example, in the two
constraint statements,
%
\begin{ttlprog}
\>{\bf always} required {\em x} $>$ 0;\\
\>{\bf always} weak {\em x} = {\em y};
\end{ttlprog}
%
the second constraint will only have an effect on $x$ when the first
is not violated, that is, when $y$ has a value greater than $0$.  (The
words {\em required} and {\em weak} are symbolic names for constraint
strengths and can be defined by the programmer.  See also section
\ref{sec:constraints-in-an-imperative-environment}.)

The most interesting aspect of the Kaleidoscope approach is that the
semantics of imperative constructs such as assignment is modelled as
constraint statements.  This simplifies the language semantics as well
as the implementation, because fewer language constructs need to be
specified and implemented.  Variables in Kaleidoscope do not contain a
single value which varies, instead they contain an infinite sequence
of fixed values, where each element in this sequence corresponds to a
given point in time.  That means that a variable assignment as the
following
%
\begin{ttlprog}
\>$x$ $\leftarrow$ $x$ $+$ 1;
\end{ttlprog}
%
can be modelled as a constraint statement, where the indices indicate
the time points for which the variables stand:
%
\begin{ttlprog}
\>{\bf always} required $x_{i}$ $=$ $x_{i-1}$ $+$ 1.
\end{ttlprog}
%
The system automatically maintains the constraint
%
\begin{ttlprog}
\>{\bf always} weak $x_i = x_{i-1}$
\end{ttlprog}
%
for all variables, which ensures that each variable which is not
changed by an assignment between two time points has the same value as
it had before the last time step.

In constraint statements, data can flow in any direction between the
variables contained in constraints.  Sometimes, this is not what the
programmer wants, and Kaleidoscope provides {\em read-only}%
\index{read-only annotation}%
\index{?!read-only annotation} and {\em
  write-only annotations}%
\index{write-only annotation}%
\index{"!!write-only annotation}.  These specify which variables can
only be read but not modified, and which can only be stored into, but
not be used otherwise.  This is best shown by describing how an
assignment statement is translated from Kaleidoscope code
%
\begin{ttlprog}
\>$x \leftarrow y$;
\end{ttlprog}
%
into more basic constraint statements:
%
\begin{ttlprog}
\>{\bf always} required $tmp$ $=$ $y_{i-1}$?\\
\>{\bf always} required $x_i$ $=$ $tmp$?
\end{ttlprog}
%
The read-only annotations (written as question marks) specify that in
the first assignment, the information must flow from $y$ to $tmp$, and
in the second assignment from $tmp$ to $x$ because the variables on
the right can only be read, not written.  Write-only annotations are
written with an exclamation mark.

\index{constructor}
\index{constraint!constructor}
\index{splitting}
%
Besides the built-in primitive constraints, such as equality, identity
and linear arithmetic constraints over integers and real numbers,
Kaleidoscope supports the definition of higher-level constraints
(constraint constructors).  These can constrain variables of more
complex types, such as user-defined objects.  Whenever a constraint
constructor on complex data types appears in a constraint statement
and the statement is executed, the constructor is called in order to
place primitive or other user-defined constraints on the parts of the
complex object.  This evaluation scheme for constructors is called
{\em splitting} \cite{lopez97phd}, because it splits complex
constraints into simpler ones.  Figure~\ref{pic:splitting} shows how
splitting works for an equality constraint on two point objects, both
having attributes $x$ and $y$, standing for cartesian coordinates.
%
\begin{figure}[h]
\begin{center}
\input{splitting.epic}
\end{center}
\caption{Splitting}
\label{pic:splitting}
\end{figure}

It should be noted that splitting makes it possible to handle
constraints on complex data structures even when the constraint
solvers are not powerful enough for working on these data structures,
since they must only provide constraints on numbers.  On the other
hand, the splitting of complex data structures loses higher-level
information on the overall problem specification which could be useful
for more powerful constraint solvers.

The philosophy behind the Kaleidoscope design is quite different from
all the other systems presented in this section, and from the
\turtle{} language developed in this thesis.  In Kaleidoscope,
constraints are employed to automatically maintain invariants on the
relations between objects and the values of their attributes.  The
other constraint programming systems are more oriented towards problem
solving: they support the modelling of problems using constraints and
the search for solutions by global resolution algorithms.  They
concentrate on inferring values for variables or attributes and ignore
the relations between complex objects.

Kaleidoscope is designed to be a general purpose language, but only
research prototypes have been implemented, and it is not clear from
literature whether they are useful for real-life applications.

\subsection{JACK}
\index{JACK}
\index{Java Constraint Kit}
\index{Abdennadher}

\jack{} (\java{} Constraint Kit) \cite{abdennadher01:jack:inp,Jack2}
is a preprocessor and library which supports constraint programming in
\java{}.

Constraint problems are formulated by writing \java{} constraint
handling rules (JCHR), which are similar to the constraint handling
rules developed by Fr\"uhwirth \cite{fruehwirth98chr}. CHR programs
define rules for the propagation and simplification of constraints and
are well suited for writing constraint solvers, or for extending
existing constraint solvers.  In \jack{}, the programmer defines
handlers (also called constraint solvers) which include a set of rules
for handling constraints.  These CHR programs can be translated to
\java{} code and compiled by a \java{} compiler.

In addition to JCHR, \jack{} contains a tool for graphically
displaying constraint stores and the relations between the constraints
in the store.  This visualization can be used for debugging JCHR
programs and for improving their efficiency by investigating how the
various constraint handling rules are applied.  With this knowledge,
the rules can be re-written to make them more efficient.

\subsection{\djava{}: Declarative Java}

\djava{} is also an extension of the \java{} language with integration
of constraints~\cite{zhou98dj}.  In \djava{}, classes can include
component declarations, attribute declarations, constraints and
actions.  Components are the graphical components which are parts of
the class, and attributes are similar to member variables in \java{},
but their values are determined by the system as specified by the
constraints.  Actions are used to add behaviour to components.  As in
Kaleidoscope, user-defined constraints combine built-in and other
user-defined constraints to form more complex constraints on primitive
data types or classes.

\djava{} is implemented as a compiler which translates \djava{}
programs to \java{} code.  The emphasis of the \djava{} approach is on
the creation of graphical programs, but the language can also be used
for solving arbitrary combinatorial constraint problems.  The system
also includes facilities for specifying how the solutions to these
problems are displayed graphically.

The constraint capabilities of \djava{} are designed for graphical
user interface programming, so only the imperative part of the
language (which is essentially \java{}) is useful as a general purpose
language.

\subsection{Constraint Libraries}

\index{Koalog}
\index{ILOG}

There are several commercial constraint solving libraries available,
for example from the companies ILOG~\cite{ilogwww} and
Koalog~\cite{koalogwww}.  The ILOG Solver is a library for
\cplusplus{} \cite{ILOG}, whereas the Koalog Solver is for
\java{}~\cite{koalog}.

The ILOG Solver is a commercially successful library for constraint
programming which makes extensive use of language features of
\cplusplus{}, such as object-oriented programming and overloading.
Thus constraints can be created by simply writing arithmetic
expressions involving objects of the predefined classes for
representing variables.
 
The Koalog Solver consists of a library of classes for the two main
tasks in constraint programming: modelling and solving.  The first
task is supported by providing a set of classes for representing
variables and constraints, the second task by providing constraint
solvers and constraint optimizers for searching for any, all or
optimal solutions.  The system can be extended by creating new
subclasses of existing constraints with new behaviour and by
customizing and extending the solving algorithms.

Because of the integration of constraints into these imperative
languages, one could call these systems constraint imperative, but it
is important to note that this is not the kind of seamless integration
aimed at by the other systems, where constraints are an important
aspect of both the language syntax and semantics.


\subsection{Other Systems}

\index{K\"ok\'eny}
\index{YAFCRS}

K\"ok\'eny~\cite{ny94yet} describes an open architecture for embedding
constraints into object-oriented programs.  The system, called YAFCRS
(Yet Another Framebased Constraint Resolution System) mainly provides
representation methods for finite domain constraint problems and
solving algorithms.  It aims at being both usable for inexperienced
users by providing predefined algorithms for problem solvers, and at
being extensible and flexible for advanced users.  Therefore, there
are many built-in constraints to choose from, and there are mechanisms
for defining user-defined constraints.  These user-defined constraints
are also solved by splitting.

\index{Lamport}
\index{Schneider}
\index{aliasing}
\index{typing}

Lamport and Schneider~\cite{lamport84aliasing} invented a proof system
for a language in which typing and aliasing of variables can be
specified using constraints.  {\em Aliasing} means that two or more
variables share the same representation, so that a modification of one
variable also changes the others.  This system is based on Hoare
Triples and is intended for proving the correctness of imperative
programs.  Their system for handling aliasing is not restricted to
simple equality of variables, but can deal with arbitrary relations
between program variables, as long as they are expressible by
constraints.  Type declarations for variables are also treated as
constraints on these variables, although they permit the traditional
way of stating variable types by type annotations.

As an example (taken from~\cite{lamport84aliasing}), consider the
following variable declaration which defines the variables $f$ and $g$
(intended as temperatures in degrees Fahrenheit and degrees Celsius)
and relates them with constraints so that they always refer to the
same temperature.
%
\begin{ttlprog}
\>\ttlVar{} {\em f}, {\em c}: real {\bf constraints} $f=g\times c/5+32$ \ttlIn{} $S$
\end{ttlprog}
%
$S$ stands for an arbitrary statement and the constraint is enforced
as long as this statement executes.

Pachet and Roy~\cite{pachet95mixing} propose an extension of
\smalltalk{}~\cite{goldberg83smalltalk} with finite-domain constraint
satisfaction mechanisms.  In their system, constraints can be defined
over arbitrary \smalltalk{} objects, so that it is possible to re-use
existing components and combine them with constraints to form new
programs, instead of requiring that everything should be re-written
completely.

\section[Imperative and Declarative Language Characteristics]{Imperative and Declarative Language\\Characteristics}

There are various aspects which are important for understanding the
problems of integrating imperative languages with higher-level
features such as higher-order functions or constraints.

The interpretation of an imperative program relies on the notions of
state and time.  A program has an associated state (represented as the
values of the program variables) at a given time.  This state changes
over time, as assignments to the variables are being made.  Therefore,
it is necessary to take time into account when trying to find out what
an imperative program does.  Declarative languages do not have this
problem, as their computations are specified independent of time.

Related to this, imperative languages are statement-oriented, whereas
declarative languages are expression-oriented.  Imperative programs
contain a sequence of statements which are to be executed in order,
whereas declarative programs consist of expressions, where the order
of evaluation is not important as long as mathematical and otherwise
semantic restrictions do not interfere.  In declarative languages, it
does not even matter whether an expression is evaluated more than
once, because the evaluation has no effects except for producing a
value.  This freedom makes correctness proving much easier for
declarative programs, and the language implementation can benefit as
well since optimization is less difficult.

The main problem with integrating constraints and imperative languages
is the interaction between the imperative program flow, which alters
the state of the program, and the declarative interpretation of
constraints, which does not know anything about state or time.  One or
both of the two interpretations of computation need to be modified in
order to bring them together.


\section{Constraints in an Imperative Environment}
\label{sec:constraints-in-an-imperative-environment}

When constraints are used in an imperative environment, many problems
arise which do not exist in either of the two paradigms, as long as
these are treated in isolation.  Some of these problems appear
whenever declarative and imperative languages are combined, others are
specific to constraint imperative languages.

% Because of the notion of state and time in imperative systems, they
% have to be considered not only when specifying constraints, but when
% solving them, too.

\paragraph{Side Effects.}
\index{side effects}

As soon as imperative languages are part of a system, it is necessary
to define the semantics of side effects such as assignment statements
or imperative in-/output.  In a constraint imperative language, the
possibilities of interaction between side effects and constraint
statements must be examined, and rules for these interactions have to
be defined.  An example where this problem occurs is when a variable
is constrained by one or more constraints and the program assigns a
value to the variable which does violate the constraint(s).

\paragraph{Object Identity.}
\index{object identity}

In imperative languages, all objects created during a program run are
identified by their identity, so that more than one object which looks
the same can actually be different objects occupying different areas
of the computer's memory.  In higher-level languages, different
objects with the same contents are normally indistinguishable, thus
leading to more freedom in the implementation of those languages.

In non-imperative constraint languages, for example, an equality
constraint between two variables can simply be satisfied by assigning
the same object reference to both of them, or by copying the whole
object.  Since there is no way the programmer can notice the
difference, the effect is the same. In imperative languages
side-effects can later make the equality invalid in the latter case,
whereas in the former case, the equality is preserved because the
objects in both variables are simultaneously modified.

Lopez, Freeman-Benson and Borning have examined the implications of
object identity \cite{lopez94identity}, and the design of
Kaleidoscope was influenced by this so that it provides both {\em
  identity} constraints (written as ``=='') and {\em equality}
constraints (written as ``='').

\paragraph{Duration.}  
\index{duration}

The duration of a constraint is the time span in which the constraint
must be enforced.  This problem does not exist for declarative
constraint languages, because they specify constraints between
variables which have to hold as long as the variables exist.  For
imperative languages, where the flow of control over time is
explicitly expressed, it is necessary to state not only which
constraints need to hold, but also {\em when}.  On the one hand, this
adds much complexity to constraint imperative programs, on the other,
it makes them more flexible.

Kaleidoscope offers three possible durations for constraints: {\em
  once} means that a constraint is added to the store, the store is
re-solved and then the constraint is immediately retracted.  This is,
for example, used for implementing assignment with constraints.  {\em
  always} constraints are defined once and then the constraint is
enforced for the rest of the program run.  {\em assert-during}
constraints are active while a statement (which may contain loops, so
the duration is not known when asserting the constraint) is executed,
and then retracted.

% It is not clear from literature whether {\em always} constraints
% really exist forever, or whether their lifetime depends on the objects
% which are constrained by the constraints.


\paragraph{Strengths.}
\index{strength}
\index{stay constraint}
\index{default constraint}
\index{constraint!strength}
\index{constraint!default}
\index{constraint!stay}
%
Constraints need to be labelled with strengths when they shall form a
constraint hierarchy, in order to deal with under- and
over-constrained problems.  Although a constraint imperative system
without constraint hierarchies could be designed, its usefulness would
be drastically reduced, because it would be difficult to constrain
variables while the program dynamically adds or retracts constraints.
So-called {\em default constraints} or {\em stay constraints}
\cite{borning2000splitstay} are very useful, especially for graphical
applications, where the user expects the graphical objects to stay
where they are as long as no other (stronger) constraints are placed
on them, such as alignment constraints or movement with the mouse.

It should be noted that constraint hierarchies are useful in the
context of other constraint languages too, and they have been
integrated into constraint logic languages~\cite{wilson94hclp}.

In Kaleidoscope, at least one constraint strength called {\em
  required} does exist, and the user can define additional strengths
symbolically in a {\em strength declaration}, which lists all
available strengths for the program, strongest to weakest.  The
following example shows a strength declaration which declares five
strengths:
%
\begin{ttlprog}
\>{\bf strength} required, strong, medium, weak, weakest.
\end{ttlprog}
%
In all constraint statements following this declaration, the
constraint strengths {\em required}, {\em strong}, {\em medium}, {\em
  weak} and {\em weakest} can be used.

\subsubsection{Consequences}

Constraint imperative programming systems introduce a complete new set
of problems.  One could argue that they combine the problems of
declarative systems with the problems of imperative ones.  Existing
constraint imperative systems are slower than traditional imperative
systems, and they do not have the same clean semantics as the
declarative constraint programming languages for a mathematical
treatment for proving their correctness.  The first of these arguments
can be addressed by the fact that none of the existing systems had the
same development time as widespread imperative systems, and could be
well optimized to run faster.  For example, the Kaleidoscope'93
implementation was reported to run much faster than the first
implementation, Kaleidoscope'90.  Additionally, it must be said that
most of the systems run on top of other high-level systems, such as
Smalltalk or Lisp and are research projects.  Implementing them on
bare machines and optimizing them would improve their performance
further.

The argument about losing correctness is strong, but considering the
fact that the goal is to get the user to use more and more of the
declarative features of the language, the correctness is expected to
increase with the programmer's ability to use more powerful
techniques.


%%% Local Variables: 
%%% mode: latex
%%% TeX-master: "da.tex"
%%% End: 

%% End of constraint-imperative.tex.



%% turtle.tex -- Chapter on the programming language Turtle
%%
%% Copyright (C) 2003 Martin Grabmueller <mgrabmue@cs.tu-berlin.de>

%%%
%%% This chapter documents the design of the constraint imperative
%%% language Turtle and describes the language.  At the end, the
%%% formal operational semantics are given.
%%%

%%%%%%%%%%%%%%%%%%%%%%%%%%%%%%%%%%%%%%%%%%%%%%%%%%%%%%%%%%%%%%%%%%%%%%%%
\newcommand{\lsq}{{\normalfont [\![}}
\newcommand{\rsq}{{\normalfont ]\!]}}
\newcommand{\ST}[1]{{\mathcal{S}\lsq #1 \rsq}}
\newcommand{\TT}[1]{{\mathcal{T}\lsq #1 \rsq}}
\newcommand{\ET}[1]{{\mathcal{E}\lsq #1 \rsq}}
%%%%%%%%%%%%%%%%%%%%%%%%%%%%%%%%%%%%%%%%%%%%%%%%%%%%%%%%%%%%%%%%%%%%%%%%

\chapter[\turtle{} -- a Constraint Imperative Language]%
{\turtle{} -- a Higher-order Constraint Imperative Programming Language}
\label{cha:turtle}

As seen in the last chapter, there already exist several approaches to
the design and implementation of constraint imperative programming
languages.  But they all focus on other aspects of the constraint
imperative approach than the ones we are interested in.  The
combination of constraints and object-oriented programming has been
covered in detail by Lopez~\cite{lopez97phd} and we will not deal with
that topic at all.  The same applies to backtracking and search, which
are the main properties of the language Alma-0~\cite{apt97search,
  apt98alma, apt98almaproject}.  We are more interested in
investigating more fundamental properties of constraint imperative
languages, such as the interaction of side-effects and constraints or
constraints as a means of preserving program invariants. Additionally,
we want to integrate other properties of high-level languages, such as
higher-order functions and algebraic
data types%
\index{algebraic data type}.  Thus we have designed the
higher-order constraint imperative programming language \turtle{},
which combines functional, constraint and imperative programming
concepts into one language, hopefully benefitting from the advantages
of each of them.

This chapter presents the principles behind the design of the
constraint imperative language \turtle{}.
Chapter~\ref{cha:turtle-impl} describes how these principles have been
dealt with in the implementation of the language.


%%
%% List of features required in a proper constraint imperative 
%% language.
%%
\section{Requirements}
\label{sec:requirements}

It is important to recall the most important requirements one wishes
to address with a constraint imperative language, before going into
details of the language design.
%
\begin{itemize}
\item The use of high-level programming constructs should be optional,
  so that the unexperienced programmer can start with the features he
  or she is used to, and use the additional facilities when required.
  In the author's opinion, higher-level features will be used as soon
  as their usefulness (especially in conjunction with supporting
  libraries) is noted by programmers.

\item Constraint statements should fit smoothly into other imperative,
  block-structured statements.  For example, the scope of constraints
  should be similar to the scope of functions and variables.  This
  makes it easier to understand constraints in the context of the
  imperative program control flow.
  
\item Constraint imperative programs must be reasonably efficient.
  This means at least that constraint solving should not unnecessarily
  slow down normal program execution.  Programs which do not use
  constraint programming features should be as fast as programs
  written in a normal imperative language.
  
\item Even though imperative language features are supported, the
  language must be safe: this requires garbage collection and type
  checking. (\turtle{} uses static type checking, but dynamic type
  checking as known from languages in the \lisp{} family would work as
  well.)
  
\item The semantics of the constraint programming extensions should be
  clear and declarative, in order to minimize their impact on program
  complexity.
\end{itemize}
%
Freeman-Benson and Borning~\cite{benson92int} have also developed a
list of goals for their integration of constraints into imperative
languages, but this list is mostly oriented towards object-oriented
features, and could therefore easily be combined with our
requirements, should an integration of object-oriented programming be
needed.

%%
%% General arguments for and against integrating constraint into 
%% the base programming language.
%%
\section{Library or Language Integration}

Before starting to design and implement a new language, one should ask
whether the goals of this new language could be achieved by
alternative, maybe simpler or more effective methods.  For most
extensions of programming languages, a library of suitable data
structures and functions suffices to accomplish the desired effect.
Examples for this are constraint solving libraries like ILOG for
\cplusplus{}~\cite{ILOG} or Koalog for \java{}~\cite{koalog}.  The
library approach has a number of advantages:

\begin{itemize}
\item Compatibility with existing source code and programming
  environments.
  
\item Preserving investments (financial as well as intellectual).
  Existing libraries can often be combined with the language extension
  libraries, so they do not need be re-written to be used in new
  applications.  The learning curve for integrating libraries is not
  as steep as for a new language extension and mostly consists of
  learning the application programming interface (API) of the new
  library.
\end{itemize}

\noindent
On the other hand, a complete integration into a language (both
syntactically and semantically) also has advantages:

\begin{itemize}
\item Better optimization possibilities because the compiler knows
  about the semantics of certain constructs, such as constrainable
  variables or constraint statements.
  
\item Cleaner syntax because of smooth syntactic integration instead
  of an add-on library, where the functionality can be accessed only
  by data structure manipulations and function calls.  Both of these
  constructs obscure the meaning of the used programming constructs.
  For example, the use of a compound statement for expression
  synchronization as in \java{} or \ada{} is less error prone than
  several calls to different library functions, which must be done in
  the correct order.
  
\item Less semantic and syntactic restrictions than for library
  implementations.  These are restricted to the possibilities of the
  underlying language.  For example, it is relatively easy to write a
  library for emulating lazy evaluation in strict functional
  languages, but the opposite is significantly harder and probably
  results in a new implementation of a lazy language.  Another example
  is the addition of multithreading to the C language by using thread
  libraries. This extension is only possible because the library can
  side-step the language semantics using operating system calls.
  Integrating new language features into the base language lifts most
  of these restrictions.
\end{itemize}

\noindent
For this thesis, we take the approach of defining a new programming
language which incorporates many well-understood concepts with
constraint programming.  One more reason in addition to the arguments
listed above is the existence of many library implementations, whereas
only few languages have built-in support for constraints.  So it is
more interesting to study the latter, especially as it seems to be
harder to design and implement a tightly integrated system than to
build an add-on library.

%%
%% A short overview of the imperative base constructs of the Turtle
%%  language
%%
\section{The Base Language}
\label{sec:base-language}

Before looking at the language features typical for constraint
imperative languages, we describe the imperative language on which the
complete \turtle{} language is built.

The imperative subset of the \turtle{} language allows the programmer
to define variables, functions and data types.  In this language,
program variables are names for storage locations and their contents
can be changed by assignment.  Functions are sub-programs which take
parameters as input values and can return output values, like
functions in the \cee{}
%~\cite{fixme:c} 
or \cplusplus{}
%~\cite{fixme:cpp}
languages, methods in \java{}
%~\cite{fixme:java} 
or procedures in \modula{}~\cite{wirth85modula}.  Functions contain
code in the form of statement lists, consisting of assignment,
conditional and iteration statements.  All common arithmetic and
boolean expressions known from imperative languages are also allowed.

The execution model of the imperative base language is the traditional
model where statements are executed top-down, and each statement
executes in the state resulting from its predecessor.  When constraint
programming is added to the base language, these basic execution rules
will not be changed, in order to minimize the semantic changes
introduced by this extension.

\turtle{} supports the development process by providing a module
system, algebraic data types%
\index{algebraic data type} as known from functional languages and
support for overloading of functions and variables, also known as
ad-hoc polymorphism.  As already mentioned, functions can be
higher-order.  Functions can be passed as arguments to other
functions, they can be the results of functions and they can be stored
into variables and data structures.  These features will be described
in more detail in following sections.

\turtle{} does not have explicit memory management like \modula{}'s
{\tt new/dispose} or \cee{}'s {\tt malloc/free}.  Like most modern
languages, it uses automatic storage reclamation, also known as
garbage collection.  This makes the use of dynamic data structures
safe, because it avoids many sources of programming errors which
result in memory leaks or the use of already freed memory regions.

The language is statically typed and all variables and functions must
be declared with their types.  The compiler checks at compile time
whether the program is type correct, so when the program is run, it
cannot fail because of type errors.  This makes the language safer
than weakly typed languages like {\sc C}.

Finally, the \turtle{} base language has a simple exception model for
handling run-time errors.  Whenever a \turtle{} program fails, an
exception is raised and terminates the program.  This ensures that no
run-time error goes by unnoticed and leads to wrong results or worse
errors later in the program run.

%%
%% What is needed to grow a traditional imperative language to a
%% constraint imperative language.
%%
\section{Constraint Programming Extensions}
\label{sec:constraint-programming-extensions}

The step from an imperative to a constraint imperative language
involves the addition of several language constructs as well as the
adaption of existing features to the new programming style.  In this
section, we will investigate the constructs which were added to the
imperative base language.  Please note that the language modifications
suggested in this section are oriented towards a language design which
starts from a traditional imperative language and adds constraint
features to result in a constraint imperative language.  The
alternative way of adding imperative features to a constraint language
would require different modifications.

This section will introduce the various constraint programming
extensions which have been incorporated into the imperative base
language of \turtle{}.  Section~\ref{sec:language-description} will
describe the complete language informally, but in more detail.


\subsection{Constrainable Variables}
\label{sec:constrainable-vars}

\index{normal variables}
\index{constrainable variables}
%
Normal variables in imperative languages are names for storage
locations which hold values, and into which new values can be stored
with imperative assignment statements.  In addition, \turtle{} has
{\em constrainable variables}.  Their values are not modified by
assignment statements, but can be derived from sets of constraints.
It is the task of the {\em constraint solver} (see below) to manage
the constraints and to calculate the values of the constrainable
variables.

In \turtle{}, constrainable variables can be used like normal
variables most of the time.  It is possible to store values into
constrainable variables, to use them in expressions or to pass them as
parameters to functions.  When used in this way, they work like normal
variables.  Only when used in constraint statements, the two types of
variables differ: normal variables work like constants whereas
constrainable variables are treated like variables in the mathematical
sense, and the constraint solver can modify their values in order to
satisfy the constraints of the constraint statement.

\index{constrained type}

When declaring constrainable variables, these variables must be marked
as such by declaring them with a {\em constrained type}.  The
following program fragment illustrates this by defining two normal
variables $x$ and $y$ of type {\em int} and a constrainable variable
$z$ of type {\bf !} {\em int}.

\begin{ttlprog}
\>\ttlVar{} $x$: int, $y$: int;\quad\ttlVar{} $z$: {\bf!} int;
\end{ttlprog}
% var x: int;
% var y: ! real;
%
Constrained types must have a base type which is an integer or a real
type, and are declared with an exclamation mark preceding the base
type.

For a complete description on how constrainable variables are created
and used, see section~\ref{sec:turtle-constraints}.

%
% Which constraint statements are there, and how are they used.
%
\subsection{Constraint Statements}
\label{sec:constraint-statements}

The language must provide statements for maintaining the constraint
store and for examining the state of the store.  It must be possible
to add constraints to the store and to remove them when they are no
longer needed.  This is best done by providing a compound statement
which clearly describes how long each constraint will stay in effect.
This is similar to lexical scoping, where it is always syntactically
apparent where each variable is accessible and where not.  We
therefore propose a statement for enforcing constraints while a
sequence of statements is executed:
%
\begin{ttlprog}
\>\ttlRequire{} $z > 0$ \res{in}\\
\>\>$y \leftarrow 2$;\\
\>\>$x \leftarrow x$ * $y$;\\
\>\ttlEnd{};
\end{ttlprog}
%
This statement, called a {\bf require} statement, adds a constraint
($x>0$ in the previous example) to the store and then executes the
statements between the {\bf in} and {\bf end} keywords.  When the {\bf
  require} statement is left, the constraint is removed from the store
and does not constrain the involved variables ($x$ in the example) any
longer.

\index{constraint!strength}
%
Support for constraint hierarchies requires a mechanism for attaching
strengths to constraints.  For example, each constraint can have a
strength annotation in its definition:
%
\begin{ttlprog}
\>\ttlRequire{} $x > 0$ : {\em strong} \ttlAnd{}\\
\>\>\>\>\>$y = 0$ : {\em mandatory} \ttlIn{}\\
\>\>\dots\\
\>\ttlEnd{};
\end{ttlprog}
%
\index{constraint conjunction}
%
When a constraint is annotated with a strength, it is added to the
store with the given strength, otherwise with the strongest strength.
The strongest strength is {\em mandatory}, and was specified in the
previous example for clarity.  In \turtle{}, constraint strengths must
be integer constants, but it is possible to give symbolic names to
constants using a {\bf const} declaration.  The example makes use of
another feature of the {\bf require} statement: more than one
constraint can be enforced at the same time, by writing a {\em
  constraint
  conjunction}%
\index{constraint conjunction}%
\index{constraint!conjunction}%
\index{conjunction} instead of a single constraint.  A conjunction is
a set of constraints separated by the keyword {\bf and}.
It is not possible to specify {\em constraint disjunctions}%
\index{constraint disjunction}%
\index{constraint!disjunction}%
\index{disjunction} in \turtle{}, because disjunctions introduce
non-determinism and there are no means in \turtle{} to handle this.

A {\bf require} statement without a body ensures that the given
constraint will be enforced as long as the variables in the constraint
are alive.
%
\begin{ttlprog}
\>\ttlRequire{} $z > 1$;
\end{ttlprog}
% require z > 1;
%
The life time and scope of constraints are discussed in detail in
section~\ref{sec:turtle-constraints}.

% Additionally, we do not only want to support constraints which are
% enforced for the time a statement is executed, but also constraints
% which live as long as the data structures they are placed on.  For
% example, given the following data structure which represents an
% abstract syntax tree for typed arithmetic expressions ({\em operator}
% is the type of all binary operators and {\em typeset} is the type of
% all sets of types):

% \begin{ttlprog}
% 1\>\ttlDatatype{} expr = rnumber (value: real, typ: !typeset) \ttlOr{}\\
% 2\>\>\>\>\>\>\>\>\>inumber (value: int, typ: !typeset) \ttlOr{}\\
% 3\>\>\>\>\>\>\>\>\>bin (op: operator, l: expr, r: expr, typ: !typeset);\\
% \end{ttlprog}
% % datatype expr = rnumber (value: real, typ: !typeset) or
% %                 inumber (value: int, typ: !typeset) or
% %                 bin (op: operator, l: expr, r: expr, typ: !typeset);

% It would be nice to be able to place constraints on the types of the
% individual abstract syntax node.\footnote{In this example, we suppose
%   that the primitive constraint solver supported set constraints.}
% Then we would like to place the constraint that the type of a binary
% node is the intersection of the types of the operand nodes:

% \begin{ttlprog}
% 1\>\ttlVar{} b: expr $\leftarrow$ bin (left, right, fulltypeset ());\\
% 2\>\ttlRequire{} typ (b) $\subseteq$ typ (left) $\cap$ typ (right);\\
% \end{ttlprog}

% % var b: expr := bin (left, right, emptyset ());
% % require typ (b) $\subseteq$ typ (left) $\cup$ typ (right);

% By initialising the type of the binary node with the set of all types
% and then placing a constraint on this set, we can narrow this set
% until it contains exactly one type.  If the set becomes empty a type
% error can be diagnosed and if the set contains more than one element,
% the expression is ambiguous and should be rejected by the type
% checker, too.

% To be more precise, the constraint on the binary node should be weaker
% than the constraints on the leaves (which are normally constants of
% declared variables), and weaker than the constraint placed on the
% result type of an expression by the context in which the expression
% appears.  For example, if an expression appears on the right-hand side
% of an assignment, the type of the expression should be constrained to
% be in the type(s) found for the l-value on the left-hand side.


\subsection{User-defined Constraints}

\index{user-defined constraint}

Besides the constraints provided by the language, it must be possible
to define additional, more complex constraints.  These so-called {\em
  user-defined constraints} are means of abstraction over constraints
in the same way as functions are abstractions over statements.  For
illustration, consider the following constraint statement, which
constrains three variables to be pairwise distinct:

\begin{ttlprog}
\>\ttlRequire{} {\em x} $\neq$ {\em y} \ttlAnd{} {\em x} $\neq$ {\em z} \ttlAnd{} {\em y} $\neq$ {\em z};
\end{ttlprog}
% require x <> y and x <> z and y <> z;

\noindent
The use of user-defined constraints makes it possible to write this
constraint statement more cleanly, by first defining a constraint for
pair-wise inequality of variables, and then using this user-defined
constraint in the constraint statement.

\begin{ttlprog}
\>\ttlConstraint{} distinct ({\em x}: {\bf!} int, {\em y}: {\bf!} int, {\em z}: {\bf!} int)\\
\>\>\ttlRequire{} {\em x} $\neq$ {\em y} \ttlAnd{} {\em x} $\neq$ {\em z} \ttlAnd{} {\em y} $\neq$ {\em z};\\
\>\ttlEnd{};\\
\>\ttlRequire{} distinct ({\em x}, {\em y}, {\em z});
\end{ttlprog}
% constraint distinct(x: ! int, y: ! int, z: ! int)
%   require x <> y and x <> z and y <> z;
% end;
% require distinct (x, y, z);

\noindent
Of course, for this small program fragment, the advantage is not so
clear, but for large programs with many such constraints, it is much
harder to write down correctly the inequalities and not forgetting a
constraint on a pair of variables.  Using the user-defined constraint
{\em distinct} will solve this problem by avoiding the need to write
down the same constraints over and over again.

\subsection{Constraint Store and Solver}
\label{sec:store-and-solver}

Whereas the addition of constrainable variables, constraint statements
and user-defined constraints is directly visible in the syntax of a
programming language, the addition of a {\em constraint store} and a
{\em constraint solver} is more related to the language semantics.

Usage of constrainable variables and constraint statements requires a
mechanism for deducing their values from constraints.  This mechanism
consists of a constraint store which is necessary to store constraints
as long as they are in effect, and of a constraint solver which
derives values for constrainable variables from the constraints.

The constraint store holds constraints in some symbolic representation
which allows the solver to efficiently access the properties of
constraints and constrainable variables included in constraints.
Therefore it is organized differently than the normal store known from
imperative machines which is just an array of memory cells.


% \section{Syntax}

% A programming language's syntax requires some thought.  Not only
% technical, but also psychological aspects have to be considered, since
% many programmers are very used to the languages they have known since
% a long time, and no technical advantage would get them to use another
% programming language if they did not like its appearance.  There are a
% lot of advocates both for a very terse syntax and a redundant syntax,
% for various reasons, and normally the arguments for either of them are
% not technical as well.  Writing and reading are very subjective
% activities, not only as far as the outer form is concerned, but also
% for the contents.

% The syntax chosen for \turtle{} tries to make a compromise between too
% terse (and sometimes cryptic) and too long (and sometimes tiresome)
% syntax.  Appendix~\ref{cha:turtle-grammar} contains the formal
% definition of the grammar, and appendix~\ref{cha:example-modules}
% contains some complete programs for illustration.  As an example, a
% short program has been written in program~\ref{prog:turtle-syntax}.

% \begin{Program}
% \begin{ttlprog}
% 1\>\ttlFun{} fib($i$: int): int\\
% 2\>\>\ttlIf{} $i$ $\leq$ 1 \ttlThen{}\\
% 3\>\>\>\ttlReturn{} 1;\\
% 4\>\>\ttlElse{}\\
% 5\>\>\> \ttlReturn{} fib($i$ $-$ 1) $+$ fib($i$ $-$ 2);\\
% 6\>\>\ttlEnd{};\\
% 7\>\ttlEnd{};
% \end{ttlprog}
% \caption{Syntax for \turtle{}}
% \label{prog:turtle-syntax}
% \end{Program}

% It would be possible to provide other syntaxes for \turtle{}-programs
% and let the programmer decide which syntax suits her best.  This could
% be implemented by adding different parsers for each different input
% syntax to the compiler, which would translate the program text into a
% single, internal representation for further processing.

% A {\sc Lisp}-like syntax would have to be extended with type
% declarations, because the language semantics require these.  In the
% example in program~\ref{prog:lisp-syntax} we extended the syntax of
% Scheme~\cite{kelsey98r5rs} with type declarations for functions and
% function parameters.

% \begin{Program}
% \begin{ttlprog}
% 1\>({\bf define} ({\bf int} fib ({\bf int} $i$))\\
% 2\>\>({\bf if} ($\leq$ $i$ 1)\\
% 3\>\>\>1\\
% 4\>\>\>(+ (fib ($-$ $i$ 1)) (fib ($-$ $i$ 2)))))
% \end{ttlprog}
% \caption{{\sc Lisp}-Syntax}
% \label{prog:lisp-syntax}
% \end{Program}

% A syntax similar to that of {\sc C, C++} or {\sc Java} could be
% adapted without changes, but of course the semantics would be
% restricted to the abilities of \turtle{}.  A fibonacci-function in
% this syntax can be found in program~~\ref{prog:c-syntax}.

% \begin{Program}
% \begin{ttlprog}
% 1\>{\bf int} fib({\bf int} $i$) \{\\
% 2\>\>{\bf if} ($i$ $\leq$ 1) {\bf return} 1;\\
% 3\>\>{\bf else} {\bf return} fib($i$ $-$ 1) $+$ fib($i$ $-$ 2);\\
% 4\>\}
% \end{ttlprog}
% \caption{{\sc C, C++} or {\sc Java}-Syntax}
% \label{prog:c-syntax}
% \end{Program}

% Modula-syntax is much more verbose (see
% program~\ref{prog:modula-syntax}), and contains much redundancy.  This
% could help the compiler to deliver better diagnostics, but requires
% the programmer to type and read much more text.

% \begin{Program}
% \begin{ttlprog}
% 1\>{\bf PROCEDURE} fib($i$: {\bf INTEGER}): {\bf INTEGER}\\
% 2\>{\bf BEGIN}\\
% 3\>\>{\bf IF} $i$ $\leq$ 1 {\bf THEN}\\
% 4\>\>\>{\bf RETURN} 1\\
% 5\>\>{\bf ELSE}\\
% 6\>\>\>{\bf RETURN} fib($i$ $-$ 1) $+$ fib($i$ $-$ 2)\\
% 7\>\>{\bf END}\\
% 8\>{\bf END} fib;
% \end{ttlprog}
% \caption{Modula-syntax}
% \label{prog:modula-syntax}
% \end{Program}

% On the other extreme, there is the syntax of many functinal languages
% (e.g. Haskell, Clean, Standard ML or Opal), which omits all
% superfluous language elements of the other examples.  This leads to
% very concise programs and is extended by language features like
% pattern matching, how it was used in program~\ref{prog:fp-syntax} to
% implement a case distinction.

% \begin{Program}
% \begin{ttlprog}
% 1\>fib 0 = 1\\
% 2\>fib 1 = 1\\
% 3\>fib $i$ = (fib $i - 1$) $+$ (fib $i - 2$)
% \end{ttlprog}
% \caption{Syntax of functional programs}
% \label{prog:fp-syntax}
% \end{Program} 

% The examples were shown to illustrate how difficult it is to choose
% from any of these possibilities when designing a new language.  The
% concrete syntax chosen for \turtle{} will have to be investigated in
% the future and could be replaced in the future if if proves to be
% inconvenient.


\section{Language Description for \turtle{}}
\label{sec:language-description}

Now that the imperative base language as well as the constraint
extensions of \turtle{} have been introduced, this section describes
all concepts of the complete language.

\turtle{} is a higher-order constraint imperative programming
language.  It combines concepts from functional and imperative
programming languages with constraint programming by introducing few
but powerful programming constructs.

\index{module}
\index{module!main}
\index{main module}
\index{public}
\index{private}
%
A \turtle{} program consists of a set of one or more {\em modules},
where one module is the {\em main module} which is the first to get
control when the program is started.  Each module is a compilation
unit which encapsulates a set of functions, constraints, constants,
variables and data type definitions.  These definitions can be {\em
  public}, which means that other modules can {\em import} and use
these definitions.  Non-public definitions are {\em private} to the
module and there is no access from the outside.

Modules are used to organize subsystems of a program into pieces which
can be separately developed, tested and maintained, and are therefore
important for software engineering purposes.  Additionally, modules in
\turtle{} can be parametrized by data types, so that algorithms can be
expressed independently of the actual data types they work on.

Functions are first-class objects in \turtle{}, which can be passed as
parameters or function results and can be stored in data structures.
\turtle{} is lexically scoped, so the scope of every variable,
function and user-defined constraint is defined by its syntactic
position in the source code.

\turtle{} has a variety of built-in primitive data types such as
integers, floating-point reals, characters, strings, booleans and
built-in type constructors for arrays and lists, as well as the
ability to define recursive algebraic data types%
\index{algebraic data type}.

Last but not least, \turtle{} supports constraint programming by
providing syntax for specifying constraint hierarchies over arbitrary
domains with built-in constraint solvers and control over the life
time of constraints, both smoothly integrated into the higher-order
imperative base language.  Constraints can be used to determine values
for global and local variables as well as for constraining the fields
of {\em compound data types}, %
\index{compound data type}%
such as arrays, lists or algebraic data types%
\index{algebraic data type}.  User-defined constraints can be
specified which define higher-level constraints in terms of the
primitive constraints provided by the built-in constraint solvers.

The language features will be described in more detail in the
following subsections.  For the formal syntax of \turtle{}, consult
Appendix~\ref{cha:turtle-grammar}.

\subsection{Variables and Functions}

\index{variables}
\index{variables!local}
\index{variables!global}
\index{global variables}
\index{local variables}

Variables in \turtle{} can be global or local.  Global variables are
accessible from all functions and user-defined constraints in the module
where the variables are defined.  When the variables are declared {\em
  public}, they are also accessible to the functions and user-defined
constraints in all modules of the same program which imported the
variables' defining module.

Local variables can only be accessed from code in the functions or
user-defined constraints they are defined in, or from local functions
or user-defined constraints defined in the body of the function.

\index{functions}
%
Functions are used like functions or procedures in other imperative
programming languages.  They expect parameters as input values and
return values to the calling context.  The main difference to some
other imperative languages is the fact that functions in \turtle{} may
return compound data types like arrays, lists, tuples and user-defined
data structures.

The calling convention for functions is call-by-value, parameters are
evaluated before they are passed to a function.  Compound data types
are reference types, so that the evaluated form of such values are
references, and these references are then passed to functions.  A
function which receives a reference type as a parameter can thus
modify the passed value, similar to many object-oriented languages
like \java{}
%~\cite{java:fixme}
or \smalltalk{}%
%~\cite{smalltalk:fixme}
.

User-defined constraints are similar to functions and the same rules
apply for local variables, functions, exporting and importing of
constraints etc., but there are different calling conventions for
user-defined constraints (the constraint store is not re-solved while
user-defined constraints are executed).  User-defined constraints are
described in more detail in section~\ref{sec:turtle-constraints}

\subsubsection{Overloading}

Functions and variables in \turtle{} may have the same names, as long
as it is possible to determine unambiguously which function or
variable is used at each occurence of its name.  Whenever variables
and functions of the same name have different types, this can be done,
and data types can have the same name as variables or functions
because they only appear in type expressions.

Overloading in \turtle{} is similar to that in \ada{}
%~\cite{ada:fixme} 
and is more powerful than for example in \java{} or \cplusplus{},
because functions can not only be overloaded in their parameter types
but also in their return types.

\subsubsection{Constant Variables}

Variables can be declared with the {\bf const} keyword instead of the
{\bf var} keyword, then the storage location they stand for cannot be
changed by assignment after the variable has been initialized.  The
value stored in the variable, if it is a compound value, can be
modified, so it is not possible to define constant data structures in
\turtle{}.

\subsubsection{Constrainable Variables}

See section~\ref{sec:turtle-constraints} for a description of
constraints and constrainable variables in \turtle{}.


\subsubsection{Higher-order Functions}

Functions can be the return values and parameters of other functions,
and can even be stored in data structures.  Functions which accept
functions as parameters or return them are called higher-order
functions, and are a central concept of functional programming
languages.  Since it is not difficult to combine them with imperative
languages and make the language considerably more powerful, they have
been integrated into \turtle{}.\footnote{\turtle{} could therefore be
  called a {\em functional constraint imperative}
\index{functional constraint imperative}%
language, but that is not the focus of this work.  See
Chapter~\ref{cha:summary} for possible future work in this direction.}

Especially for handling data structures such as lists or arrays, and
for building reusable software libraries, higher-order functions are
very useful.

Local variables declared in a function exist as long as the
surrounding function lives, and whenever the function is invoked, it
can operate on the variables in its scope.  A nice example for this is
a function which creates another function with the functionality of
counting the number of times it was called.
Program~\ref{prog:counter} defines a function which creates such a
counter function.  Whenever the returned function is executed, it
yields the next integer in the increasing sequence starting from 1.
This is possible because the returned function maintains a copy of the
variable $x$ which exists as long as the function lives, and maintains
its state across function invocations.  Function and user-defined
constraint values are created by function/constraint expressions as
shown in the example program~\ref{prog:counter} in lines 3--6.

\begin{Program}
\begin{ttlprog}
1\>\ttlFun{} make\_counter (): \ttlFun{} (): int\\
2\>\>\ttlVar{} {\em x}: int $\leftarrow$ 0;\\
3\>\>\ttlReturn{} \ttlFun{} (): int\\
4\>\>\>\>\>\> {\em x} $\leftarrow$ {\em x} $+$ 1;\\
5\>\>\>\>\>\> \ttlReturn{} {\em x};\\
6\>\>\>\>\> \ttlEnd{};\\
7\>\ttlEnd{};
\end{ttlprog}
\caption{Counter function example}
\label{prog:counter}
\end{Program}
% fun make_counter (): fun (): int
%   var x: int := 0;
%   return fun (): int
%            x := x + 1;
%            return x;
%          end;
% end;

\subsection{Statements and Expressions}

The bodies of functions and user-defined constraints contain sequences
of statements.  When a function or user-defined constraint is
executed, the statements in its body are evaluated in order, and each
statement executes in the program state which was left by the previous
statement.

Statements may contain expressions as some component, for example the
condition inside a conditional statement.  The difference between
statements and expressions is that expressions produce values and
statements do not.

\turtle{} provides all basic imperative statement kinds
and expression forms, which will be documented in the rest of this
section.

\subsubsection{Conditionals}

Conditionals are expressed as {\bf if-then-else} statements of the
form
%
\begin{ttlprog}
\>\ttlIf{} {\em condition} \ttlThen{}\\
\>\>{\em stmt1};\\
\>\ttlElse{}\\
\>\>{\em stmt2};\\
\>\ttlEnd{};
\end{ttlprog}
%
For the common idiom where one conditional is nested inside the {\em
  else}-part of another conditional, \turtle{} provides the shorter
syntax:
%
\begin{ttlprog}
\>\ttlIf{} {\em condition1} \ttlThen{}\\
\>\>{\em stmt1};\\
\>\ttlElsif{} {\em condition2} \ttlThen{}\\
\>\>{\em stmt2};\\
\>\ttlElse{}\\
\>\>{\em stmt3};\\
\>\ttlEnd{};
\end{ttlprog}

\subsubsection{Loops}

Only one kind of loop statement is provided by \turtle{}, the {\bf
  while} loop:
%
\begin{ttlprog}
\>\ttlWhile{} {\em condition} \ttlDo{}\\
\>\>{\em stmt};\\
\>\ttlEnd{};
\end{ttlprog}

\subsubsection{Function Calls and Return}

A call to a function which has the predefined return type {\bf void}
is also a statement.  Calls to functions with other return types are
not allowed as statements and can only be used inside of expressions.
A function call is written by giving the name of the function
(possibly qualified, see section~\ref{sec:module-system}), followed by
the list of parameters in parentheses.  When there are no parameters,
empty paratheses must be written.  The following example writes the
string ``Good morning, Sir!'' \!\!and goes to the beginning of the next
line:
%
\begin{ttlprog}
  \>io.put ("Good morning, Sir!");\\
  \>io.nl ();
\end{ttlprog}
%
The last statement which is executed in a function with a return type
other than {\bf void} must be a {\bf return} statement, which
specifies the value which is returned by the currently active function
call.  The following program fragment returns the value of the
variable $x$ to the calling function.
%
\begin{ttlprog}
\>\ttlReturn{} $x$;
\end{ttlprog}

\subsubsection{Expressions}

Expressions allowed in \turtle{} are arithmetic and boolean
expressions as well as all other expressions which result in a value
which is not of type {\bf void}.  A few examples for expressions are:
%
\begin{ttlprog}
\>{\em x} $+$ {\em y}\\
\>3 * 4 $+$ ({\em a} $-$ 2)\\
\>\ttlTrue{} \ttlAnd{} \ttlNot{} {\em a}\\
\>io.get({\em f})\\
\>lists.map(\ttlFun{} ({\em a}: int): real \ttlReturn{} reals.from\_int ({\em a}); \ttlEnd{}, [2, 3, 1])\\
\>[2.0, 3.0, 1.0]
\end{ttlprog}

\subsection{Data Types}
\index{data types}

\index{integer}
\index{type!integer}
\index{floating point}
\index{type!real}
\index{characters}
\index{type!character}
\index{strings}
\index{type!string}
\index{algebraic data type}
\index{type!algebraic}

For practical programming, a rich set of data types is necessary to
model objects of the problem domain conveniently.  Therefore,
\turtle{} supports a variety of predefined data types ({\em primitive} %
\index{primitive type}%
\index{type!primitive}%
and
{\em structured}%
\index{structured type}%
\index{type!structured}%
) and a method for defining new data types.

Predefined data types in \turtle{} are integers, floating point
numbers, characters, strings, booleans, lists, arrays, tuples and
function types.  Moreover, the user can define her or his own data
types, in the form of recursive algebraic types.

\index{primitive type}

The number, character, string and boolean types are primitive types.
Strings could also be considered structured, because they consist of
element values.  They are listed with the primitive types because they
cannot contain values of arbitrary element types, but only characters.

\index{array}
\index{type!array}
\index{list}
\index{type!list}
\index{tuple}
\index{type!tuple}
\index{user-defined type}
\index{type!user-defined}

Structured data types are {\em arrays}, {\em lists}, {\em tuples} and
{\em user-defined types}. Arrays are vectors of fixed length and
uniform element types, lists are ordered collections of varying length
and uniform element types and tuples are ordered collections of fixed
length and heterogeneous element types.  User-defined data types are
also structured data types.

The following description lists the available data types.  See
Appendix~\ref{cha:turtle-grammar} for the syntax of type expressions
and values of the various types.

\begin{description}
  
\item[int] The data type {\em int} is a finite set of (possibly
  negative) integer numbers, where the minimum and maximum
  representable values are implementation-specific.  In the reference
  implementation, this type ranges from -536870912 to 536870911.

\item[long] Similar to {\em int}, but the minimum and maximum values
  of this type are guaranteed to range at least from -2147483648 to
  2147483647.  This is the range of 32-bit two-complement numbers and
  is provided for easy interfacing with the operating system.
  
\item[real] Approximations of real numbers, implemented as floating
  point numbers of type {\bf double} in the underlying C
  implementation.  Nowadays, these are IEEE~754 64-bit floating point
  numbers on most machines.
  
\item[bool] The type {\em bool} consists of the two constants {\bf
    true} and {\bf false}.
  
\item[char] This is the set of characters, such as 'a', 'Y' or '!'.
  The encoding for characters is the 16-bit Unicode encoding.
  Currently, the reference implementation only supports input and
  output of extended 8-bit ASCII characters, although it is capable of
  handling the full 16-bit character set internally.
  
\item[void] The void type, written as {\bf ()} is only used as the
  result type of functions which do not return anything (similar to
  procedures in \modula{} or void functions in \cee{}).

\item[string] A finite sequence of characters.
  
\item[Lists] The type {\bf list of} $\tau$ is the type of finite lists
  of values of type $\tau$, of varying length.  The access to
  individual elements of a list is only possible in linear time, but
  prepending an element needs only constant time.
  
\item[Arrays] The type {\bf array of} $\tau$ is the type of finite
  arrays of values of type $\tau$, where the length is fixed on array
  value creation.  Access to individual array elements is possible in
  constant time.
  
\item[Tuples] Tuple types are written $(\tau_1, \dots, \tau_n)$, and
  denote the type of $n$-tuples of the given element types.  Tuple
  values are written as the tuple elements in parentheses, separated
  by commas: ("hello", 3.14) has the type ({\bf string}, {\em real}).
  
\item[User-defined types] Users can define arbitrary algebraic types,
  which are similar to record or union types in other languages.

\end{description}


\subsubsection{User-defined data types}

\index{discriminated record}
\index{tagged union}
%
The user-defined data types are similar to the data types in
functional languages like \haskell{}~\cite{peytonjones98haskell98} or
\mllanguage{}~\cite{milner97sml}.  In the context of imperative
languages these data types could be considered as {\em discriminated
  records} in \pascal{} or \modula{} or {\em tagged unions} in \cee{}.
A variable of a user-defined type can be assigned different variants
of the same type.  When reading a field from the value stored in the
variable it is checked whether the variant stored in the variable does
have the field, and if not, an exception is raised.  These explicit
checks make user-defined types much safer than records or unions in
other imperative languages.

Besides the automatic checking for correctness, the programmer has the
possibility to examine the actual variant of a value and to adjust the
program flow accordingly.

As an example we will show a data type for representing binary trees
in Program~\ref{prog:tree-definition}.  The data type {\em tree} has
two variants: the variant {\em leaf} which represents a leaf of the
tree with a value, and the variant {\em node} which represents an
inner node with a search key.

\begin{Program}
\begin{ttlprog}
1\>\ttlDatatype{} tree = \>\>\>\>\>\>\>\>leaf({\em value}: int) \ttlOr{}\\
2\>\>\>\>\>\>\>\>\>node({\em left}: tree, {\em right}: tree, {\em key}: int);
\end{ttlprog}
\caption{Tree data type definition}
\label{prog:tree-definition}
\end{Program}

Using this data type declaration, the \turtle{} compiler automatically
generates a set of functions for creating instances of the type, for
accessing the fields and for examining the variant of a given value of
the type.  Program~\ref{prog:induced-signature} shows the names and
types of these generated functions.  The constructor functions receive
the values which will be stored into the fields of the value as
parameters and create either a leaf or a node value.  The disciminator
functions can be used to determine the variant of a given tree value,
and the selectors return the values stored in the corresponding
fields.  The mutators can be used to modify the fields by storing new
values into the appropriate storage locations.

The user-defined data types in \turtle{} are inspired by the algebraic
data types in the strict purely functional programming language
\opal{}~\cite{Pepper.Opal}.  Mutator functions have been added to
allow imperative programming with data structures constructed from
user-defined data types.

\begin{Program}
\begin{ttlprog}
1\>// {\em Constructors}\\
2\>\ttlFun{} leaf ({\em value}: int): tree\\
3\>\ttlFun{} node ({\em left}: tree, {\em right}: tree, {\em key}: int): tree\\
4\>// {\em Discriminators}\\
5\>\ttlFun{} leaf? ($t$: tree): bool\\
6\>\ttlFun{} node? ($t$: tree): bool\\
7\>// {\em Selectors}\\
8\>\ttlFun{} value ($t$: tree): int\\
9\>\ttlFun{} left ($t$: tree): tree\\
10\>\ttlFun{} right ($t$: tree): tree\\
11\>\ttlFun{} key ($t$: tree): int\\
12\>// {\em Mutators}\\
13\>\ttlFun{} value{!} ($t$: tree, {\em value}: int)\\
14\>\ttlFun{} left{!} ($t$: tree, {\em left}: tree)\\
15\>\ttlFun{} right{!} ($t$: tree, {\em right}: tree)\\
16\>\ttlFun{} key{!} ($t$: tree, {\em key}: int)
\end{ttlprog}
\caption{Induced signature for the {\em tree} data type}
\label{prog:induced-signature}
\end{Program}
% // Constructors
% fun leaf (value: int): tree
% fun node (left: tree, right: tree, key: int): tree
% // Discriminators
% fun leaf? (t: tree): bool
% fun node? (t: tree): bool
% // Selectors
% fun value (t: tree): int
% fun left (t: tree): tree
% fun right (t: tree): tree
% fun key (t: tree): int
% // Mutators
% fun value! (t: tree, value: int)
% fun left! (t: tree, left: tree)
% fun right! (t: tree, right: tree)
% fun key! (t: tree, key: int)

By using the automatically generated discriminator and selector
functions {\em leaf?}, {\em node?}, {\em left} and {\em right} a
function can traverse a tree made up of objects of this type.  The
function {\em leaves} in program~\ref{prog:tree-leaves} makes use of these
functions for calculating the number of leaves in a given tree.

\begin{Program}
\begin{ttlprog}
1\>\ttlFun{} leaves($t$: tree): int\\
2\>\>\ttlIf{} leaf?($t$) \ttlThen{}\\
3\>\>\>\ttlReturn{} $1$;\\
4\>\>\ttlElse{}\\
5\>\>\>\ttlReturn{} leaves(left($t$)) + leaves(right($t$));\\
6\>\>\ttlEnd{};\\
7\>\ttlEnd{};
\end{ttlprog}
\caption{Function {\em leaves}}
\label{prog:tree-leaves}
\end{Program}

The automatically generated functions can be used like any other
\turtle{} functions, they can be passed to higher-order functions or
stored in data structures.

\turtle{} user-defined data types can be parametrized.  In conjunction
with the parametric modules described in
section~\ref{sec:module-system}, polymorphic data types can be
programmed.  Consider for example the following data type, whose only
purpose is to encapsulate (``box'') another type:

\begin{ttlprog}
\>\ttlDatatype{} box$<a>$ = box ($value$: $a$);
\end{ttlprog}
% datatype box<a> = box (value: a);

Because the data type is parametrized with a data type, it can be used
for storing values of any type by instantiating it with an actual data
type:

\begin{ttlprog}
\>\ttlVar{} b: box$<$int$>$, c: box$<$\ttlArray{} \ttlOf{} char$>$;
\end{ttlprog}
% var b: box<int>, c: box<array of char>;

\subsubsection{Type Aliases}

Type expressions can be abbreviated in \turtle{} by defining type
aliases%
\index{type alias}%
\index{alias!type}%
\index{type!alias}.  This is done by giving a name to a type
expression in a {\bf type} declaration:
%
\begin{ttlprog}
\>\ttlType{} b = box$<$\ttlArray{} \ttlOf{} char$>$
\end{ttlprog}
%
In declarations following such a type declaration, the name $b$ can be
used instead of the longer type expression it stands for.  The
compiler expands type aliases when necessary to check for type
equality.  For example, in the following program fragment the types
$a$ and $b$ are equal as far as the type checker is concerned.
%
\begin{ttlprog}
\>\ttlType{} a = \ttlArray{} \ttlOf{} \ttlString{};\\
\>\ttlType{} b = \ttlArray{} \ttlOf{} \ttlString{};
\end{ttlprog}

\subsection{The Module System}
\label{sec:module-system}

\turtle{} has a module system which provides private name-spaces and
parametric modules.

Modules are a very important component when building larger software
systems, because parts of the system can be developed independently
from others.  By separating the signature (the set of externally
visible entities) from the implementation (e.g. the implementation
of an abstract data structure, such as a stack), the system remains
flexible despite its size, because the implementation can change over
time without affecting the components which use it.  Only when the
signature changes the using modules need to be modified.

\begin{Program}[htb]
\begin{ttlprog}
1\>\ttlModule{} a$<$$\alpha$$>$;\\
2\>\ttlExport{} f${}_a$;\\
3\>\ttlFun{} f${}_a$($x$: $\alpha$) \dots{} \ttlEnd{};\\
\\
4\>\ttlModule{} b$<$$\alpha$$>$;\\
5\>\ttlExport{} f${}_b$;\\
6\>\ttlFun{} f${}_b$($x$: $\alpha$) \dots{} \ttlEnd{};\\
\\
7\>\ttlModule{} c;\\
8\>\ttlImport{} a$<$int$>$, b$<$\ttlString{}$>$;\\
9\>\ttlFun{} f${}_c$($x$: int) \dots{} \ttlEnd{};
\end{ttlprog}
\caption{Example modules}
\label{prog:modules}
\end{Program}

In order to illustrate the working of the module system we will
discuss an example program.  The program consists of three modules,
which are shown in Program~\ref{prog:modules}.  Module $a$ has a
module parameter, $\alpha$, and exports the functions f${}_a$, which
is defined using the module parameter.  Module $b$ exports f${}_b$,
which is parametrized, too.  Module $c$ is the main program and
imports modules $a$ and $b$, where the modules are instantiated with
the data types {\em int} and {\bf string}, respectively.

In module $c$, the functions imported from $a$ and $b$ are made
available with the types which are created by replacing the module
parameters with the actual types.  That means, that module $c$ sees a
function f${}_a$ with type {\bf fun}({\em int}) and function f${}_b$
with type {\bf fun}({\bf string}).  In combination with overloading,
this becomes a very powerful tool for structuring programs.  The
\turtle{} standard library makes extensive use of parametric modules,
for example with a module which exports generic list manipulation
functions.  By importing the module with different parameter types
multiple times, lists of different element types can be manipulated
using the same functions.

In addition to fully qualified access to module members, such as using
{\tt io.nl()} in order to call the function {\tt nl} from module {\tt
  io}, the user can specify a list of identifiers which should be
imported unqualified, that means which can be used by just writing the
identifier, omitting the module name.  This is especially useful when
an imported entity is often used, but of course it increases the
possibility of name clashes with identifiers defined in the current
module.  The following example illustrates the import of the {\em
  lists} module, parametrized with the type {\bf string}. The
identifier {\em reverse} can be used without giving its module name,
as shown in the second line.

\begin{ttlprog}
\>\ttlImport{} lists$<$\ttlString{}$>$(reverse);\\
\>\ttlVar{} {\em s}: \ttlList{} \ttlOf{} \ttlString{} $\leftarrow$ reverse (["a", "b", "c"]);\\
\end{ttlprog}


\subsection{Constraints}
\label{sec:turtle-constraints}

Constraints in \turtle{} are used only for determining the values of
program variables.  This is in contrast to other constraint imperative
languages and especially to constraint logic languages.  In these
languages, either imperative or logic constructs are built on top of
constraint features, whereas in \turtle{}, constraints are added to an
imperative language solely for the purpose of calculating the values
for (explicitly declared) constrainable variables. In Kaleidoscope,
for example, constraints can be placed on any variable, and there is
no need to explicitly declare variables as being constrainable.  The
same applies to constraint logic programming languages, where any
variable may appear in a constraint and can be determined by
constraints if it is unbound.  \turtle{}, on the other hand, does not
have unbound variables.

The usage of constrainable variables shall be demonstrated by some
examples.

First, an example for a simple constraint imperative program will be
discussed (see Program~\ref{prog:maximum-two}).  The function {\em
  max} calculates the maximum of two nonnegative integer values by
declaring a local constrainable variable $mx$ and requiring that this
variable must be greater than or equal to the two parameters.  By
requesting (but not requiring) that $mx$ should be equal to zero, the
constraint solver will calculate the smallest value which is both
greater or equal to $x$ and $y$.  The expression {\bf var} 0 creates a
constrainable variable with initial value 0 and {\bf !}{\em mx}
retrieves the value of {\em mx} as calculated by the constraint
solver.

\begin{Program}
%% fun max(x: int, y: int): int
%%   var mx: ! int := var 0;
%%   require mx = 0 : strong and
%%     mx >= x and
%%     mx >= y in
%%     return !mx;
%%   end;
%% end;
\begin{ttlprog}
1\>\ttlFun{} max($x$: int, $y$: int): int\\
2\>\>\ttlVar{} $mx$: {\bf!} int $\leftarrow$ \ttlVar{} 0;\`{\em declare constrainable variable}\\
3\>\>\ttlRequire{} $mx$ = 0 : strong \ttlAnd{}\`{\em require minimal value}\\
4\>\>\>$mx$ $\geq$ $x$ \ttlAnd{}\`{\em require value greater or equal to parameters\dots}\\
5\>\>\>$mx$ $\geq$ $y$ \ttlIn{}\\
6\>\>\>\ttlReturn{} {\bf !}$mx$;\`{\em return the maximum}\\
7\>\>\ttlEnd{};\\
8\>\ttlEnd{};
\end{ttlprog}
\caption{Maximum of two nonnegative integers}
\label{prog:maximum-two}
\end{Program}

Program~\ref{prog:maximum} defines an extended variant of function
{\em max}, which illustrates an additional feature of constraint
imperative programming in \turtle{}: user-defined constraints.  The
program calculates the maximum element of a nonempty list of
nonnegative integers.  The function defines a variable $mx$, which has
the same role as in the previous example.  Now the program requires
the user-defined constraint {\em greatereq} for every element of the
list.  This constraint requires its first parameter to be greater or
equal than the second.  Since only the first parameter of the
constraint {\em greatereq} is marked as constrainable, only this
variable will be adjusted to satisfy the constraint.

It can easily be seen that the function {\em max} calculates the
maximum element of its arguments in both examples.  Of course an
imperative solution to this problem would be shorter, but the purpose
of the examples was to show as many aspects of constraint programming
in \turtle{} as possible, not to show the shortest possible solution.
The introductory example of
Chapter~\ref{cha:constraint-imperative-programming} is more suitable
for illustrating the advantages of constraint imperative programming.

\begin{Program}
%% constraint greatereq(x: ! int, y: int)
%%   require x >= y;
%% end;
%% fun max(l: list of int): int
%%   var mx: ! int := var 0;
%%   require mx = 0 : strong in
%%     while l <> null do
%%       require greatereq(mx, hd l);
%%       l := tl l;
%%     end;
%%     return !mx;
%%   end;
%% end;
\begin{ttlprog}
1\>\ttlConstraint{} greatereq ({\em x}: {\bf!} int, {\em y}: int)\\
2\>\>\ttlRequire{} {\em x} $\geq$ {\em y};\`{\em {\em x} must be at least as large as {\em y}}\\
3\>\ttlEnd{};\\
4\>\ttlFun{} max ({\em }l: \ttlList{} \ttlOf{} int): int\\
5\>\>\ttlVar{} {\em mx}: {\bf!} int $\leftarrow$ \ttlVar{} 0;\\
6\>\>\ttlRequire{} {\em mx} = 0 : strong \ttlIn{}\\
7\>\>\>\ttlWhile{} {\em l} $\neq$ \ttlNull{} \ttlDo{}\`{\em for every element of the list\dots}\\
8\>\>\>\>\ttlRequire{} greatereq ({\em mx}, \ttlHd{} {\em l});\`{\em the constraint {\em greatereq} is required}\\
9\>\>\>\>{\em l} $\leftarrow$ \ttlTl{} {\em l};\\
10\>\>\>\ttlEnd{};\\
11\>\>\>\ttlReturn{} {\bf!}{\em mx};\`{\em now {\em mx} holds the maximum}\\
12\>\>\ttlEnd{};\\
13\>\ttlEnd{};
\end{ttlprog}
\caption{Maximum of a list of nonnegative integers}
\label{prog:maximum}
\end{Program}


\subsubsection{Definitions of Notation}
\label{sec:definitions}

Before describing the constraint extensions of \turtle{} in detail,
some notation needs to be defined.  It will be used in subsequent
sections.

\index{primitive constraint}
\index{constraint!primitive}
\index{user-defined constraint}
\index{constraint!user-defined}
%
A constraint is either a {\em primitive constraint}, which can be
handled directly by the built-in constraint solver, or a {\em
  user-defined constraint}.  Primitive constraints are boolean
expressions, for example $x > y$ or $2x-y\leq 3$.  User-defined
constraints are similar to functions, but are defined with the keyword
{\bf constraint}.  Constraint definitions are independent of the
program execution, until they are {\em instantiated}, e.g.~until they
are invoked with actual parameters in constraint statements.

The constraint conjunctions given in the constraint statement {\bf
  require} may consist of primitive and user-defined constraints.
Both kinds of constraints can be annotated with {\em constraint
  strengths} when they appear in constraint statements.

\subsubsection{Constraint Strengths}

\index{constraint!strength}
\index{constraint strength}
\index{strength}
%
When constraints are added to the constraint store by using them in
constraint statements, each constraint can have an associated strength
which indicates its importance.  A constraint strength is written
after the constraint, separated by a colon, like in the following
example:
%
\begin{ttlprog}
\>\ttlRequire{} $x > 1$ : strong;
\end{ttlprog}
%
Each constraint strength must be either an integer constant or a name
which is defined as an integer constant.  The strongest constraint
strength has the value 0, and numerically greater values indicate
weaker strengths.  When no strength annotation is given for a
constraint, the strongest strength is assumed.

The \turtle{} language does not define the semantics of constraint
strengths, except that when a solver respects the associated strengths
for constraints, it is required to satisfy all mandatory constraints
(with strength 0), whenever possible, even if that requires the
violation of weaker constraints.

The \turtle{} compiler and run-time system (see
Chapter~\ref{cha:turtle-impl}) do not interpret constraints in any
way, they are just translated into a symbolic representation and
passed to one of the constraint solvers.  The representation currently
used only supports linear equations and inequalities, so the compiler
only accepts expressions of this form in constraint statements.
Lifting this limitation is an important task for future work.

\subsubsection{Constrainable Variables}

\index{constrainable}
\index{variable!constrainable}
%
Variables whose values can be determined (and modified) by constraints
are declared by annotating their data type.  They are called {\em
  constrainable variables}.  These variables must be explicitly
declared as such and be initialized by constructing a variable object
and storing it into the constrainable variable.  The variable is
marked as being constrainable by writing an exclamation mark before
the type of the values which can be stored into the variable.  The
program fragment
%
\begin{ttlprog}
\>{\bf var} $x$: {\bf!} int := \ttlVar{} 0;
\end{ttlprog}
%
defines and initializes a variable $x$ of type {\em int}, which can be
determined by constraints.  The expression
%
\begin{ttlprog}
\>\ttlVar{} 0
\end{ttlprog}
%
in the previous declaration of $x$ creates the variable object which
holds the value of the variable (0 in the example).

The use of variable objects for representing constrainable variable
requires an additional indirection for fetching values from the
variables or storing into them.

In places other than in constraints, constrainable variables can be
used like normal variables.  However, the programmer must keep in mind
that the value of such a variable might change when constraints (which
can be directly or indirectly related to the variable) are added or
removed from the store, if there is a connection between the variable
and one of the variables in these constraints.

Normal variables are just names for storage locations.  They may
appear in constraints, too, but they are treated as constants whose
values are determined by the time the constraints are instantiated.
Constrainable variables are more complicated, because they are
actually data objects which contain values and additional information
needed by the constraint solvers for handling the variable whenever
they appear in primitive constraints.

The following example declares both a normal and a constrainable
variable.  The variable $x$ is declared as a normal variable of type
{\em int} and initialized with the value $1$.  The variable $y$ is
declared as a constrainable variable (note the exclamation mark in the
type), and is initialized with a variable object holding the value
$2$.  The expression {\bf var} 2 creates this object.
%
\begin{ttlprog}
\>\ttlVar{} {\em x}: int $\leftarrow$ 1;\\
\>\ttlVar{} {\em y}: {\bf!} int $\leftarrow$ \ttlVar{} 2;
\end{ttlprog}
%
Figure~\ref{pic:constrainable-vars} illustrates the situation after
the two variables have been declared and initialized.

\begin{figure}[htp]
\begin{center}
\input{constrainable-vars.epic}
\end{center}
\caption{Normal and constrainable variables}
\label{pic:constrainable-vars}
\end{figure}

\index{"!!operator}
%
Since the value of a constrainable variable is not directly stored in
the location reserved for that variable, but in the variable object, a
special operator for retrieving the stored value must be used in
\turtle{}.  This operator is the exclamation mark, used as a prefix
operator.  In the following example, a constrainable variable $z$
containing the value 3 is declared.  Another variable $a$ is declared
in the second line and initialized with the value fetched from the
variable object of $z$.
%
\begin{ttlprog}
\>\ttlVar{} $z$: {\bf!} int $\leftarrow$ \ttlVar{} 3;\\
\>\ttlVar{} $a$: int $\leftarrow$ {\bf!}$z$;
\end{ttlprog}
%
The explicit handling of constrainable variables---creation of
variable objects and fetching their values---makes it possible to
share constrainable variables in different data structures, e.g.~for
placing several constraints on all the elements of a list of
constrainable variables.  It is not possible to change the value of
stored in the variable object of a constrainable variable by
assignment, it is only possible to create a new variable object and
let the constrainable variable contain that.  When the value in a
variable object shall be changed, constraint statements must be used.

Note that the exclamation mark in the type of variable $z$ marks the
type as constrainable, whereas the exclamation mark in the expression
{\bf !}$z$ fetches a value from the variable object.

Constrainable variables in \turtle{} are similar to reference values
in languages like Standard ML~\cite{milner97sml}, where they work like
a box into which a single value may be stored.  The main difference is
that the compiler knows about the special status of constrainable
variables and that they hold additional information for use with the
constraint solver.


\subsubsection{Liveness and Scope}
\label{sec:liveness-scoping}

Constraints have limited liveness ranges and scopes, similar to
variables.  A constraint lives---and is enforced---as long as the
body of the constraint statement is executing, or, for constraint
statements without a body, as long as the constrainable variables in
the constraints are alive.  A constrainable variable lives until the
program terminates (if it is a global variable), as long as their
defining environment (for local variables) or the containing data
structure (for constrainable fields of compound data types) lives.  An
environment lives as long as the function invocation for which it was
created is active, or as long as the closure lives in which it is
captured.  Compound data structures live as long as a reference to
them exists, which again is held in (global or local) variables.


\subsubsection{Constraint Translation}
\label{sec:constraint-translation}

\index{constraint!translation}

The expressions which appear in {\em require} statements must be
translated to machine code in another way than normal boolean
expressions as they appear in imperative statements.  After type
checking the constraint expression (in the same way as for normal
expressions) the constraints are translated to a symbolic
representation.  During run-time, these symbolic representations are
passed to the constraint solver, which can add them to its store to
check for consistency and for determining values for the variables.

\index{constraint!user-defined}

User-defined constraints cannot be passed to the constraint solver
directly, since the solver does not know anything about them and does
not have the necessary knowledge to work on general constraints.  When
a user-defined constraint appears in a constraint statement, the
compiler generates a call to the code generated for the constraint
definition, similar to normal function calls.  The definition contains
the constraint statements which implement the user-defined constraint;
if they are primitive constraints, they are added to the store,
otherwise, they are called again, until the code reaches a primitive
constraint.


\subsubsection{Constraint Solver Interface}

\turtle{} requires at least one constraint solver in the run-time
system which has to determine the values of constrainable variables
appearing in constraint statements.  Constraint statements must be
able to add constraints to the constraint store and to remove them
again, when they are no longer needed.  The interface between the
\turtle{} code and the constraint solvers is designed to consist of
only three primitive operations: constraint creation, addition and
removal.  The creation of constraints requires a symbolic
representation of constraints being built during run-time.  This
representation is solver-independent and contains only the information
necessary to deduce the relations between contained constrainable
variables and constants.  Constraints are then added to the constraint
solvers, and the solvers are allowed to add additional information to
the symbolic representation, for example upper and lower bounds for
the variable's value.  Whenever an added constraint cannot be
satisfied, the responsible constraint solver is expected to raise an
exception.  The last operation is constraint removal, and tells the
constraint solver which owns that constraint to remove it from the
store.

When a constraint solver has determined that a newly added constraint
is satisfiable, it must assign values to all contained constrainable
variables so that the values form a solution for the current
constraint store.  Because the symbolic representation of the
constraints contains the constrainable variables, the solver can
directly assign the calculated values to the value slots of the
variables.


% \subsubsection{Constraint Representation}
% \label{sec:constraint-representation}

% The semantics of constraints, constaint stores and constraint solvers
% in \turtle{} has already been discussed.  We will now have a closer
% look at the implementation of constraint integration into the
% imperative base language.  The following section will explain how
% primitive constraints are translated into a symbolic representation
% suitable for processing in the constraint solvers.

% Program~\ref{prog:constraint-example} declares two variables $x$ and
% $y$, where $x$ is constrainable.  Two constraints are defined on these
% variables.  Fig.~\ref{pic:constraints} shows the representation of the
% variables and the constraints as an object graph.

% \begin{Program}
% \begin{ttlprog}
% % fun ex()
% %   var x: ! int;
% %   var y: int;
% %   y := 1;
% %   prefer x = 0;
% %   require x < y;
% % end;
% 1\>\ttlFun{} ex()\\
% 2\>\>\ttlVar{} $x$: {\bf!} int;\\
% 3\>\>\ttlVar{} $y$: int;\\
% 4\>\>y $\leftarrow$ 1;\\
% 5\>\>\ttlPrefer{} $x = 0$;\\
% 6\>\>\ttlRequire{} $x < y$;\\
% 7\>\ttlEnd{};
% \end{ttlprog}
% \caption{Constraint example}
% \label{prog:constraint-example}
% \end{Program}

% The constrainable variable $x$ is implemented as a reference to an
% object, which holds references to all constraints this variable
% appears in, and the actual value of the variable.  The environment, in
% which the variables are declared holds references to all constraints
% in which local variables appear.  The constraints hold references to
% the memory locations into which the values of the variables must be
% stored.

% \begin{figure}[htp]
% \begin{center}
% \input{constraint.epic}
% \end{center}
% \caption{Constraint representation}
% \label{pic:constraints}
% \end{figure}

% Constrainable variables have a mark which determines whether the value
% of the variable is currently ``fixed'' or not.  The value of a
% variable is fixed in assignment statements, so that it will not be
% changed immediately after assignment when other constraints are
% satisfied.  This enforces the effect of assignment and breaks possible
% cycles in the constraint graph.

% For each global or local constrainable variable, a variable object is
% created as soon as the location for the variable is created (at
% startup for global variables and when an environment is created for
% local variables).  This variable object is then stored into the
% variable's location and remains there until the variable dies.  This
% means that in order to access constrainable variables, an extra
% indirection is necessary.

% For user-defined data types which contain constrainable fields, the
% same principle applies.  The constructor for the data type stores
% newly created variable objects into the constrainable fields when
% creating the data object.  The accessor and mutator functions for
% constrainable fields are changed accordingly.

\section{Operational Semantics for \turtle{}}
\label{operational-semantics}

This section provides a formal semantics for a subset of \turtle{}
called $\mu$\turtle{}.  All \turtle{} programs can be transformed into
$\mu$\turtle{} by syntactic transformation (see
section~\ref{sec:syntactic-transformation}), therefore, this semantics
specifies the complete \turtle{} language.

We present the semantics by defining an abstract machine and showing
how \turtle{} programs are translated to instructions for this
machine.

\subsection{The \turtle{} Abstract Machine}

The \turtle{} Abstract Machine (TAM) maintains a machine state, which
is repeatedly modified as specified below, until a terminating state
is reached.  While it is running, the TAM modifies the store and the
constraint store by interpreting TAM instructions.  The TAM is
basically a standard stack machine for imperative languages equipped
with registers for maintaining the computation environment, and
extended with instructions for managing the constraint store.

The constraint store and the constraint solvers work mostly
independent of the rest of the machine, except for the TAM
instructions which modify the store, and the fact that the constraint
solvers are allowed to modify the contents of constrainable variables.

Some of the ideas for the TAM were taken from abstract machines
described by Abelson et al.~\cite{abelson96sicp} and
Wilson~\cite{wilson03schintro}.

\subsubsection{Register Set}

\begin{table}
\begin{center}
\begin{tabular}{|ll|}
\hline
{\em acc} & accumulator\\
{\em sp} & evaluation stack pointer\\
{\em cont} & continuation\\
{\em pc} & program counter\\
{\em env} & environment\\
{\em ex} & exception handler\\
\hline
\end{tabular}
\end{center}
\caption{TAM registers}
\label{tab:tam-registers}
\end{table}

The TAM works on the registers shown in Table~\ref{tab:tam-registers}.
The accumulator {\em acc} stores intermediate values calculated during
program execution.  Function calls, as well as all primitive
operations, leave their results in this register.  The stack pointer
{\em sp} points to the top of the evaluation stack.  This stack is
only used for evaluation of nested expressions, function return points
are kept in a chain of {\em continuation}%
\index{continuation}%
\index{continuation record}%
\index{continuation!record} records, which is pointed to
by the continuation register%
\index{continuation register}%
\index{continuation!register} {\em cont}%
\index{cont@{\em cont} (register)}.  The program counter%
\index{program counter} {\em pc}%
\index{pc@{\em pc} (register)}
always points to the next TAM instruction to be executed and is
incremented after each instruction, or set to specific code addresses
by the call and branching instructions.  The environment register%
\index{environment register}, {\em env}%
\index{env@{\em env} (register)}, always points to the environment
which holds the local variables of the currently executing function as
well as a reference to the enclosing environment, which is needed to
access free variables.  The
last register, {\em ex}%
\index{ex@{\em ex} (register)}, holds a list of exception handlers.  An
exception handler consists of a code address at which execution should
be resumed when an exception occurs, and a continuation to use for
this code.


\subsubsection{Memory Model}

\begin{table}
\begin{center}
\begin{tabular}{|ll|}
\hline
{\em Code} & function and user-defined constraint store\\
{\em Stack} & evaluation stack\\
{\em Store} & dynamic data storage\\
{\em CStore} & constraint store\\
\hline
\end{tabular}
\end{center}
\caption{TAM memory}
\label{tab:memory-model}
\end{table}

The TAM operates on four memory areas.  The {\em Code}\/ area contains
the machine instructions of the executing program.  Each machine
instruction uses exactly one cell of the code area. All intermediate
values which are produced during evaluation of expressions and
function calls are stored on the {\em Stack}.  The {\em Store} holds
all dynamic data structures which are created during execution,
including environments, closures, continuation records and the chain
of exception handlers.  These data structures are necessary for the
execution of the TAM and are described below.  {\em CStore} is the
constraint store which is maintained by the integrated constraint
solvers.  Table~\ref{tab:memory-model} summarizes these memory
segments.

The {\em Code} segment has a fixed size and contains only the program
code, whereas the {\em Stack} and {\em Store} areas are conceptionally
of unlimited size and are arrays of equal-sized memory cells.  In a
practical implementation, they are of course finite in size and some
kind of memory management is required to maintain the illusion of
unbounded memory.  Section~\ref{sec:memory-management} describes how
this works in the \turtle{} reference implementation.  The
organization of the constraint store is undefined in order to make the
implementation more flexible.  The integrated constraint solvers are
responsible for maintaining some representation of the constraints in
the store, and the TAM accesses the constraint store only by abstract
machine instructions.

In the following sections, the notation $A[X]$ refers to the element
of the array $A$ at index $X$ and will be used to indicate locations
in the $Store$ and $Stack$ arrays.

\subsubsection{Instruction Set}

\begin{table}
\begin{center}
\begin{tabular}{|ll|}
\hline
{\em Load and store instructions}&\\
\hline
{\tt load-constant} $c$& load a constant into {\em acc}\\
{\tt load-variable} $l$& load from a memory location\\
{\tt load-array} & load indexed\\
{\tt store-variable} $l$& store into a memory location\\
{\tt store-array} & store indexed\\
{\tt fetch} & fetch from constrainable variable\\
%{\tt store} & store into constrainable variable\\
{\tt push} & push the content of the {\em acc} register\\
\hline
{\em Closure and environment creation}&\\
\hline
{\tt make-closure} $l$& create a closure\\
{\tt make-environment} $n$& create an environment\\
\hline
{\em Function call and return}&\\
\hline
{\tt call} & call a closure\\
{\tt save-continuation} $l$& set up return location\\
{\tt restore-continuation} & go to return location\\
\hline
{\em Branching instructions}&\\
\hline
{\tt jump} $l$& unconditional jump\\
{\tt jump-if-true} $l$& conditional jump\\
{\tt jump-if-false} $l$& conditional jump\\
\hline
{\em Constraint handling}&\\
\hline
{\tt make-constraint} $c, s$& create a constraint object\\
{\tt add-constraints} $l_1,\dots,l_n$& add to the constraint store\\
{\tt remove-constraints} $l_1,\dots,l_n$& remove from the constraint store\\
\hline
{\em Exception handling}&\\
\hline
{\tt handle} $l$& set up an exception handler\\
{\tt unhandle} & remove an exception handler\\
{\tt raise} & raise an exception\\
\hline
{\em Machine control}&\\
\hline
{\tt halt} & halt the machine\\
\hline
\end{tabular}
\end{center}
\caption{TAM instructions}
\label{tab:tam-instructions}
\end{table}

Table~\ref{tab:tam-instructions} lists the TAM instructions.  When not
otherwise noted, the instructions which calculate some values leave
their results in the register {\em acc}.

The various load and store instructions fetch values from memory
locations or constants to the accumulator or store the accumulator
value to memory.  The array instructions expect the array address and
the offset of the indexed array slot on the evaluation stack.  The
presence of constrainable variables makes the instruction {\tt fetch}
%and {\tt store} 
necessary, which performs the additional indirection
for constrainable variables. {\tt push} saves the value in the
accumulator to the stack, incrementing the stack pointer.

Closures are created by the {\tt make-closure} $l$ instruction, which
stores the code pointer $l$ and the current environment into a closure
record.  {\tt make-environment} $n$ creates a new environment with
space for holding $n$ local variables. The content of the {\em env}
register is saved to the environment, and all values on the evaluation
stack (which are the arguments passed to the function) are stored to
their corresponding environment slots.  The address of the newly
created environment record is then placed in the {\em env} register.

{\tt call} jumps to the procedure whose closure record is in the {\em
  acc} register.  The {\tt save-con\-tinuation} instruction saves the
current state of the machine (that is, the contents of the registers
and the evaluation stack) to a continuation record.  The {\tt
  restore-continuation} instruction restores the machine state from
the state pointed to by the continuation register.

The branching instructions control the program flow either
unconditionally ({\tt jump}) or depending on the value in the {\em
  acc} register: {\tt jump-if-false} jumps to a different location
when {\em acc} contains the constant {\bf false} and {\tt
  jump-if-true} jumps when {\em acc} contains {\bf true}.

The machine has several instructions for constraint and constraint
store management.  The instruction {\tt make-constraint} $c, s$
creates a constraint record from the constraint specification $c$ and
associates with it the strength $s$.  This instruction does not modify
the constraint store, it simply creates the run-time representation
for a constraint and stores a reference to that representation in the
accumulator.  The instructions {\tt add-constraints} $l_1,\dots,l_n$
and {\tt remove-constraints} $l_1,\dots,l_n$ manage the constraint
store by adding or removing $n$ constraints $l_1,\dots,l_n$ at once.
The constraint instruction for adding constraints automatically causes
the constraint solver to check whether the constraint store together
with the newly added constraints can be satisfied.  In that case, the
store is modified accordingly, otherwise, an exception is raised.
When constraints are removed, the solver re-solves its store, too, but
in this case the store cannot get unsatisfied, so no exception can be
raised.

Three instructions are used for handling exceptions.  The {\tt handle}
instruction adds an entry to the exception handler chain in register
{\em ex} by storing a pair consisting of a label and the current
continuation to the chain.  {\tt unhandle} removes the first entry
from the chain, which acts as a last-in-first-out (LIFO) list.  {\tt
  raise} raises an exception by taking the continuation and exception
handler from the first element of the exception list and installing
them in the registers {\em cont} and {\em pc}.

The {\tt halt} instruction stops machine execution.

\subsubsection{Environments, Closures and Continuations}

The TAM maintains three kinds of data structures: environments for
storing local variables, closures for handling higher-order functions
and continuations for function calls.

Environments are records which hold one pointer to the environment of
the surrounding function (parent environment), and zero or more slots
for variables, depending on the number of local variables of the
defining function.

Closures are two-element records which store one environment pointer
and the address of a function.  The address is necessary for
transferring control to the function's code and the environment must
be stored in the closure because the function must execute in the
environment in which it was defined, even if that defining function
was already left (for example when a function is returned by a
defining function).

Continuations are used for storing the machine state.  This is
necessary for function calls, because the called function must return
to the caller when it has executed all its statements.  This return is
done by simply copying back the machine state from a continuation.
Continuations combined with environments are similar to the stack
frames of traditional imperative languages which hold both local
variables and return addresses, but because of the separation of the
two concepts, continuations can be used in more powerful ways, for
example for implementing the function {\em
  call-with-current-continuation}%
\index{call-with-current-continuation} as in Scheme%
\index{Scheme}~\cite{kelsey98r5rs} and
some dialects of Standard ML%
\index{Standard ML}~\cite{milner97sml}.

\subsubsection{Program Execution}

Execution of a $\mu$\turtle{} program requires first the preparation
of the machine, and then the interpretation of the program's TAM
instructions until a {\tt halt} instruction is executed.  The machine
is prepared by storing all constants and functions into the store of
the TAM, pushing the program parameters (input of the program, a list
of strings which is given on the command line when the machine
interpreter is started) onto the evaluation stack, storing a function
containing only the {\tt halt} instruction into the initial
continuation record and placing the address of the program's main
function into {\em pc}.  By using a special function for halting the
machine, the main function can be treated like any other function,
except for using it as the program's entry point.

The following rules specify how the TAM instructions operate on the
TAM state.  The function $new(n)$, which appears in the evalutation
rules below, returns the index of the first of $n$ unused cells and
marks the cells as being in use.  

\index{E@$\mathcal{E}$ (evaluation function)}
%
The evaluation rules for the TAM instructions are of the form 
%
$$\ET{i}\rightarrow m$$
%
where $i$ is the instruction to be evaluated and $m$ is a sequence of
register-transfer expressions%
\index{register-transfer expression}%
\index{RTE} (RTE) of the form
%
$$d\leftarrow s.$$
%
$d$ is either a register, a {\em Stack} or a {\em Store} location and
$s$ is a register, a {\em Stack} location, a {\em Store} location or a
constant.  The notation $d\leftarrow s$ means that the contents of the
location $s$ is transferred to the location $d$.  By applying the
rules to each instruction of a program in turn, the registers and
store contents will be changed and eventually the program will halt,
with the contents of the registers and stores as the program's
results.

When interpreting TAM instructions with these rules, the register {\em
  pc} is incremented after each instruction except for the jump
instructions {\tt jump}, {\tt jump-if-true} and {\tt jump-if- false}
and the function call/return instructions {\tt call} and {\tt
  restore-continuation}.  

The evaluation function $\mathcal{E}$ has the following functionality
($TAM$ stands for the set of TAM instructions):
%
$$\mathcal{E}: TAM\rightarrow RTE\,\,\,list$$
%
The first group of instructions are the load and store instructions.
The load instructions work by fetching values from either the
instruction stream (as for {\tt load-constant}) or from the {\em
  Store} and placing them in the register {\em acc}.  The store
instructions take values from either the {\em acc} register or the
{\em Stack} and transfer them to the {\em Store}.  {\tt load-array}
and {\tt store-array} require more than one operand, so that one
operand is taken from {\em acc} and the other(s) from the {\em Stack}.
{\tt fetch} loads the value pointed to by {\em acc}.

\begin{tabbing}
\qquad \= \quad \kill
$\ET{\text{{\tt load-constant }} c} \rightarrow$\\
\>$acc \leftarrow c$
\end{tabbing}

\begin{tabbing}
\qquad \= \quad \kill
$\ET{\text{{\tt load-variable }} l} \rightarrow$\\
\>$acc \leftarrow Store[l]$
\end{tabbing}

\begin{tabbing}
\qquad \= \quad \kill
$\ET{\text{{\tt load-array}}} \rightarrow$\\
\>$acc \leftarrow Store[acc+Stack[sp-1]]$\\
\>$sp \leftarrow sp-1$
\end{tabbing}

\begin{tabbing}
\qquad \= \quad \kill
$\ET{\text{{\tt store-variable }} l} \rightarrow$\\
\>$Store[l] \leftarrow acc$
\end{tabbing}

\begin{tabbing}
\qquad \= \quad \kill
$\ET{\text{{\tt store-array}}} \rightarrow$\\
\>$Store[acc+Stack[sp-1]] \leftarrow Stack[sp-2]$\\
\>$sp \leftarrow sp-2$
\end{tabbing}

\begin{tabbing}
\qquad \= \quad \kill
$\ET{\text{{\tt fetch}}} \rightarrow$\\
\>$acc \leftarrow Store[acc]$
\end{tabbing}
%
% \begin{tabbing}
% \qquad \= \quad \kill
% $\ET{\text{{\tt store}}} \rightarrow$\\
% \>$Store[acc] \leftarrow Stack[sp-1]$\\
% \>$sp \leftarrow sp-1$
% \end{tabbing}
%
{\tt push} is the only instruction which explicitly manipulates the
stack by pushing the contents of the {\em acc} register onto the {\em
  Stack}.
%
\begin{tabbing}
\qquad \= \quad \kill
$\ET{\text{{\tt push}}} \rightarrow$\\
\>$Stack[sp] \leftarrow acc$\\
\>$sp \leftarrow sp+1$
\end{tabbing}
%
The instructions for creating closures and environment records first
allocate unused storage cells and then initialize them.  The {\tt
  make-closure} instruction stores the label of the closure's code and
the current environment register into the closure record, and the
environment creation instruction copies the contents of the {\em
  Stack} and the {\em env} register into the environment record.  Then
the saved contents of the {\em Stack} is removed from the stack, and
the address of the environment record is stored into the {\em env}
register.
%
\begin{tabbing}
\qquad \= \quad \kill
$\ET{\text{{\tt make-closure }} l} \rightarrow$\\
\>$acc \leftarrow new(2)$\\
\>$Store[acc+0] \leftarrow l$\\
\>$Store[acc+1] \leftarrow env$
\end{tabbing}
%
\begin{tabbing}
\qquad \= \quad \kill
$\ET{\text{{\tt make-environment }} n} \rightarrow$\\
\>$acc \leftarrow new(n+1)$\\
\>$Store[acc+0] \leftarrow env$\\
\>$Store[acc+1\dots acc+n] \leftarrow Stack[sp-n\dots sp-1]$\\
\>$sp \leftarrow sp - n$\\
\>$env \leftarrow acc$
\end{tabbing}
%
Control transfer on function calls is accomplished by simply fetching
the code location and the environment from the closure record
currently in the {\em acc} register.

\begin{tabbing}
\qquad \= \quad \kill
$\ET{\text{{\tt call}}} \rightarrow$\\
\>$pc \leftarrow Store[acc+0]$\\
\>$env \leftarrow Store[acc+1]$
\end{tabbing}
%
Functions are called by jumping directly to the machine instructions
of their function bodies.  Since it is normally necessary to return to
some code location after the call instruction, the current state of
the machine must be saved to the {\em Store} before performing the
control transfer to the called function.  This is done by the {\tt
  save-continuation} instruction which stores all machine registers
(except for {\em acc}, which holds function results and must therefore
not be saved across function calls) and the contents of the {\em
  Stack} to the continuation record in the {\em Store}.  The result of
the instruction is stored in the {\em cont} register, which points to
the chain of active continuation records, chained through the field at
index zero of the records.

\begin{tabbing}
\qquad \= \quad \kill
$\ET{\text{{\tt save-continuation }} l} \rightarrow$\\
\>$acc \leftarrow new(4+sp)$\\
\>$Store[acc+0] \leftarrow cont$\\
\>$Store[acc+1] \leftarrow l$\\
\>$Store[acc+2] \leftarrow sp$\\
\>$Store[acc+3] \leftarrow env$\\
\>$Store[acc+4\dots acc+4+sp-1] \leftarrow Stack[0\dots sp-1]$\\
\>$sp \leftarrow 0$\\
\>$cont \leftarrow acc$
\end{tabbing}
%
{\tt restore-continuation} is the complementary instruction to {\tt
  save-continuation} and restores the machine state (except for {\em
  acc} and {\em ex}) from the continuation register {\em cont}.

\begin{tabbing}
\qquad \= \quad \kill
$\ET{\text{{\tt restore-continuation}}} \rightarrow$\\
\>$pc \leftarrow Store[cont+1]$\\
\>$sp \leftarrow Store[cont+2]$\\
\>$env \leftarrow Store[cont+3]$\\
\>$Stack[0\dots sp-1] \leftarrow Store[cont+4\dots cont+4+sp-1]$\\
\>$cont \leftarrow Store[cont+0]$
\end{tabbing}
%
The branching instructions control the flow of execution by
transferring control to other code locations.  The {\tt jump}
instruction transfers control directly to some code location, whereas
the {\tt jump-if-true} and {\tt jump-if-false} instructions change the
flow of control only if the content of the {\em acc} register is {\bf
  true} or {\bf false}, respectively.

\begin{tabbing}
\qquad \= \quad \kill
$\ET{\text{{\tt jump }} l} \rightarrow$\\
\>$pc \leftarrow l$
\end{tabbing}

\begin{tabbing}
\qquad \= \quad \kill
$\ET{\text{{\tt jump-if-true }} l} \rightarrow$\\
\>{\bf if} $acc$ {\bf then} $pc \leftarrow l$ {\bf end}
\end{tabbing}

\begin{tabbing}
\qquad \= \quad \kill
$\ET{\text{{\tt jump-if-false }} l} \rightarrow$\\
\>{\bf if not} $acc$ {\bf then} $pc \leftarrow l$ {\bf end}
\end{tabbing}
%
The three constraint instructions are responsible for the creation of
constraints and for their addition to and removal from the store.  The
instruction {\tt make-constraint} creates the symbolic representation
of a constraint $c$, annotated by the constraint strength $s$.  How
this representation is designed depends on the implementation of both
the used constraint solver and the run-time interface, so the
operations are defined as the abstract operations $create$, $add$ and
$remove$.  $create$ allocates memory for the symbolic representation
and fills it with the constrainable variables, constants and strength
of the constraint and leaves a reference to the representation in the
{\em acc} register.  $add$ adds constraints to the constraint store
and raises an exception if the store cannot be satisfied with the new
constraints, $remove$ removes the constraints from the constraint
store again.
%
\begin{tabbing}
\qquad \= \quad \kill
$\ET{\text{{\tt make-constraint }} c,s} \rightarrow$\\
\>$create(c, s)$
\end{tabbing}

\begin{tabbing}
\qquad \= \quad \kill
$\ET{\text{{\tt add-constraints }} l_1,\dots,l_n} \rightarrow$\\
\>$add (CStore, l_1,\dots,l_n)$
\end{tabbing}

\begin{tabbing}
\qquad \= \quad \kill
$\ET{\text{{\tt remove-constraints }} l_1,\dots,l_n} \rightarrow$\\
\>$remove (CStore, l_1,\dots,l_n)$
\end{tabbing}
%
The {\tt handle} instruction sets up an exception handler by adding an
exception record to the exception handler list.  The current
continuation and the address of an exception handler are stored into
this record.
%
\begin{tabbing}
\qquad \= \quad \kill
$\ET{\text{{\tt handle }} l} \rightarrow$\\
\>$acc \leftarrow new(2)$\\
\>$Store[acc+0]\leftarrow new(2)$\\
\>$Store[Store[acc+0]+0]\leftarrow cont$\\
\>$Store[Store[acc+0]+1]\leftarrow l$\\
\>$Store[acc+1]\leftarrow ex$\\
\>$ex\leftarrow acc$
\end{tabbing}
%
The {\tt unhandle} instruction, which removes an exception handler,
simply drops the first element from the exception handler list.
%
\begin{tabbing}
\qquad \= \quad \kill
$\ET{\text{{\tt unhandle}}} \rightarrow$\\
\>$ex\leftarrow Store[ex+1]$
\end{tabbing}
%
Raising an exception is done by loading the continuation and code
address for the first exception handler from the list in register {\em
  ex}.

\begin{tabbing}
\qquad \= \quad \kill
$\ET{\text{{\tt raise}}} \rightarrow$\\
\>$cont \leftarrow Store[Store[ex+0]+0]$\\
\>$pc \leftarrow Store[Store[ex+0]+1]$\\
\>$ex \leftarrow Store[ex+1]$
\end{tabbing}
%
The {\tt halt} instruction stops the execution of the machine.  The
final state of the machine represents the final state of the program
to be executed, and the result of the {\em main} function which was
first called when starting the machine is located in the {\em acc}
register.

\begin{tabbing}
\qquad \= \quad \kill
$\ET{\text{{\tt halt}}} \rightarrow$\\
\>Halt the machine.
\end{tabbing}

\subsection{Syntactic Transformation}
\label{sec:syntactic-transformation}

\begin{table}
\begin{center}
\begin{tabular}{lll}
$e$ & $::= c\; |\;  v\;  |\;  f(e_1, \dots,e_n)\;|\;e_1 := e_2\;|\;e_1[e_2]\;|\;\text{\bf !}e$\\
& $|\quad  \text{\bf fun}\;(x_1,\dots,x_n)\;  e;\dots\; \text{\bf end}$\\
& $|\quad  \text{\bf constraint}\;(x_1,\dots,x_n)\;  e;\dots\; \text{\bf end}$\\
& $| \quad\text{\bf require}\; c_1:s_1\wedge\dots\wedge c_n:s_n\;\text{\bf in}\; e;\dots\;\text{\bf end}$\\
& $| \quad\text{\bf while}\; c\;\text{\bf do}\; e;\dots\;\text{\bf end}$\\
& $| \quad\text{\bf if}\; c\;\text{\bf then}\; e_1;\dots\;\text{\bf else}\; e_2;\dots\;\text{\bf end}$\\
& $|\quad  \text{\bf return} \;e$
\end{tabular}
\end{center}
\caption{$\mu$\turtle{} Syntax}
\label{tab:mu-turtle}
\end{table}

Before translating full \turtle{} source code to TAM code, the source
program must be type-checked and all type annotations must be removed.
Additionally, the storage locations for global and local variables
must be reserved in the code segment or in the environment frames of
the defining functions.\footnote{Of course, some features of the full
  \turtle{} syntax (such as array, list and string constructors) must
  be transformed so that appropriate run-time support functions are
  called.  Strings are treated as arrays of characters at this level.}
Then the simplification function $\mathcal{S}$ is to be applied to the
program, yielding a program in $\mu$\turtle{} syntax. This syntax
(shown in Table~\ref{tab:mu-turtle}) has constants, variables,
function applications, assignments, array accesses, the constrainable
variable dereferencing operator, function and user-defined constraint
expressions and constraint, loop, conditional and return statements.
%
\index{S@$\mathcal{S}$ (syntactic transformation)}
$$\mathcal{S}: Expr \rightarrow Expr$$
%
Function and constraint definitions are translated to function- and
constraint-valued variable assignments. The compiler is expected to
declare the automatically introduced variables during semantic
analysis. The initialization expressions of variable and constant
definitions are converted to assignments, too, and simplified
recursively.
%
\begin{tabbing}
\qquad \= \quad \kill
$\ST{\text{\bf fun}\; n\;(x_1,\dots x_n)\;e;\dots\text{\bf end}} \rightarrow$\\
\>$n :=\text{\bf fun}\; (x_1,\dots x_n)\;\ST{e};\dots\text{\bf end}$
\end{tabbing}
%
\begin{tabbing}
\qquad \= \quad \kill
$\ST{\text{\bf constraint}\; n\;(x_1,\dots x_n)\;e;\dots\text{\bf end}} \rightarrow$\\
\>$n :=\text{\bf constraint}\; (x_1,\dots x_n)\;\ST{e};\dots\text{\bf end}$
\end{tabbing}
%
\begin{tabbing}
\qquad \= \quad \kill
$\ST{\text{\bf var}\; n\; := e} \rightarrow$\\
\>$n := \ST{e}$
\end{tabbing}
%
\begin{tabbing}
\qquad \= \quad \kill
$\ST{\text{\bf const}\; n\; := e} \rightarrow$\\
\>$n := \ST{e}$
\end{tabbing}
%
One-armed {\bf if} statements are normalized by adding an empty {\bf
  else} part, other compound statements are recursively simplified.
%
\begin{tabbing}
\qquad \= \quad \kill
$\ST{\text{\bf if}\; c\;\text{\bf then}\; e;\dots\;\text{\bf end}} \rightarrow$\\
\>$\text{\bf if}\; \ST{c}\;\text{\bf then}\; \ST{e};\dots\;\text{\bf else}\;\text{\bf end}$
\end{tabbing}
%
\begin{tabbing}
\qquad \= \quad \kill
$\ST{\text{\bf if}\; c\;\text{\bf then}\; e_1;\dots\;\text{\bf else}\;e_2;\dots\;\text{\bf end}} \rightarrow$\\
\>$\text{\bf if}\; \ST{c}\;\text{\bf then}\; \ST{e_1};\dots\;\text{\bf else}\; \ST{e_2};\dots\;\text{\bf end}$
\end{tabbing}
%
\begin{tabbing}
\qquad \= \quad \kill
$\ST{\text{\bf while}\; c\;\text{\bf do}\; e;\dots\;\text{\bf end}} \rightarrow$\\
\>$\text{\bf while}\; \ST{c}\;\text{\bf do}\; \ST{e};\dots\;\text{\bf end}$
\end{tabbing}
%
Tuple assignment is translated to multiple single-assignments with
temporary variables.
%
\begin{tabbing}
\qquad \= \quad \kill
$\ST{l_1,\dots,l_n := r_1,\dots,r_n} \rightarrow$\\
\>$t_1 := \ST{r_1};\dots t_n := \ST{r_n};$\\
\>$\ST{l_1} := t_1;\dots \ST{l_n} := t_n;\qquad \text{where}\; t_1,\dots,t_n\;\text{are generated variables}$
\end{tabbing}
%
The fetch operator {\bf !} %and the variable object constructor {\bf
%  var} are 
is treated by simplifying its operand.
%
\begin{tabbing}
\qquad \= \quad \kill
$\ST{\text{\bf !}e} \rightarrow$\\
\>{\bf !}$\ST{e}$
\end{tabbing}
%
% \begin{tabbing}
% \qquad \= \quad \kill
% $\ST{\text{\bf var}\;e} \rightarrow$\\
% \>{\bf var} $\ST{e}$
% \end{tabbing}
%
Binary and unary operators (including the {\bf var} operator for
constructing constrainable variables) are eliminated by replacing
their uses with applications of functions performing their respective
operations, such as addition or negation.
%
\begin{tabbing}
\qquad \= \quad \kill
$\ST{e_1 \;op\; e_2} \rightarrow$\\
\>$f_{op}(\ST{e_1}, \ST{e_2})\quad\text{where}\; f_{op} \;\text{is the function for}\; op$
\end{tabbing}
%
\begin{tabbing}
\qquad \= \quad \kill
$\ST{op\; e} \rightarrow$\\
\>$f_{op}(\ST{e})\quad\text{where}\;f_{op}\;\text{is the function for}\; op$
\end{tabbing}
%
\begin{tabbing}
\qquad \= \quad \kill
$\ST{e_1[e_2]} \rightarrow$\\
\>$\ST{e_1}[\ST{e_2}]$
\end{tabbing}
%
{\bf require} statements are recursively simplified, similar to the
other compound statements.  Note that the strengths are not
simplified, because only constants (or names of constants) are allowed
as strengths.
%
\begin{tabbing}
\qquad \= \quad \kill
$\ST{\text{\bf require}\; c_1 : s_1 \wedge\dots\wedge c_n : s_n} \rightarrow$\\
\>$\text{\bf require}\; \ST{c_1} : s_1 \wedge\dots\wedge \ST{c_n} : s_n\;$
\end{tabbing}
%
\begin{tabbing}
\qquad \= \quad \kill
$\ST{\text{\bf require}\; c_1 : s_1 \wedge\dots\wedge c_n : s_n\;\text{\bf in}\;e;\dots;\text{\bf end}} \rightarrow$\\
\>$\text{\bf require}\; \ST{c_1} : s_1 \wedge\dots\wedge \ST{c_n} : s_n\;\text{\bf in }\;\ST{e};\dots\;\text{\bf end}$
\end{tabbing}
%
Return statements without an expression are provided with a constant
expression.  It does not matter which one, because the value will not
be used, it is just added to reduce the number of cases to be
considered in later stages of the transformation.
%
\begin{tabbing}
\qquad \= \quad \kill
$\ST{\text{\bf return}} \rightarrow$\\
\>$\text{\bf return}\; 0$
\end{tabbing}
%
\begin{tabbing}
\qquad \= \quad \kill
$\ST{\text{\bf return}\; e} \rightarrow$\\
\>$\text{\bf return}\; \ST{e}$
\end{tabbing}

\subsection{Translating $\mu$\turtle{} to the \turtle{} Abstract Machine}

The function $\mathcal{T}$ translates $\mu$\turtle{} code to machine
code for the TAM.  The following definition of this function expects
its input to be simplified with the function $\mathcal{S}$ given
above.  $\mathcal{T}$ takes a $\mu$\turtle{} expression as its input
and produces a list of TAM instructions:
%
\index{T@$\mathcal{T}$ (transformation function)}
$$\mathcal{T}: Expr \rightarrow TAM \; list$$
%
$\mathcal{T}$ requires that variables are represented as a location
descriptor which tells the TAM how to access the location allocated
for the variables.  The function $\mathcal{L}$, used in the definition
of $\mathcal{T}$ below, transforms variables to location descriptors
at compile time.  A location descriptor $l\in
Location=\text{I\!N}\times \text{I\!N}$ consists of a back-pointer,
which tells how many environment links need to be followed to reach
the environment in which a variable $v\in Var$ is stored, and an index
which tells the position in this environment.  Global variables are
defined in the topmost environment, which encloses all locally defined
environments.
%
\index{L@$\mathcal{L}$ (location assignment)}
$$\mathcal{L}: Var \rightarrow Location$$
%
For the operational semantics of \turtle{}, the formal definition of
$\mathcal{L}$ is not necessary, therefore it has been omitted.
Informally, the definition of this function simply searches for the
given variable in the compile-time environment in which all declared
variables are held, and computes the back-pointer and the index during
the search.

Constants and variables are simply loaded into the accumulator,
assignments are translated to store instructions.  Indexed references
and assignments require the array as well as the index expression to
be evaluated before performing an indexed load or store operation.

\begin{tabbing}
\qquad \= \quad \kill
$\TT{c} \rightarrow$\\
\>{\tt load-constant }$c$
\end{tabbing}

\begin{tabbing}
\qquad \= \quad \kill
$\TT{v} \rightarrow$\\
\>{\tt load-variable }$\mathcal{L}\lsq{}v\rsq{}$
\end{tabbing}

% \begin{tabbing}
% \qquad \= \quad \kill
% $\TT{v} \rightarrow$\\
% \>{\tt load-variable }$\mathcal{L}\lsq{}v\rsq{}$\\
% \>{\tt fetch}$\qquad\qquad\text{iff}\;v\;\text{is a constrainable variable}$
% \end{tabbing}

\begin{tabbing}
\qquad \= \quad \kill
$\TT{e_1[e_2]} \rightarrow$\\
\>$\TT{e_2}$\\
\>{\tt push}\\
\>$\TT{e_1}$\\
\>{\tt load-array}
\end{tabbing}

% \begin{tabbing}
% \qquad \= \quad \kill
% $\TT{e_1[e_2]} \rightarrow$\\
% \>$\TT{e_2}$\\
% \>{\tt push}\\
% \>$\TT{e_1}$\\
% \>{\tt load-array}\\
% \>{\tt fetch}$\qquad\qquad\text{iff}\;e_1[e_2]\;\text{is a constrainable field}$
% \end{tabbing}

\begin{tabbing}
\qquad \= \quad \kill
$\TT{v := e} \rightarrow$\\
\>$\TT{e}$\\
\>{\tt store-variable }$\mathcal{L}\lsq v\rsq$
\end{tabbing}

% \begin{tabbing}
% \qquad \= \quad \kill
% $\TT{v := e} \rightarrow$\\
% \>$\TT{e}$\\
% \>{\tt push}\\
% \>{\tt load-variable }$\mathcal{L}\lsq{}v\rsq{}$\\
% \>{\tt store}$\qquad\qquad\text{iff}\;v\;\text{is a constrainable variable}$
% \end{tabbing}

\begin{tabbing}
\qquad \= \quad \kill
$\TT{e_1[e_2] := e_3} \rightarrow$\\
\>$\TT{e_3}$\\
\>{\tt push}\\
\>$\TT{e_2}$\\
\>{\tt push}\\
\>$\TT{e_1}$\\
\>{\tt store-array }
\end{tabbing}
%
% \begin{tabbing}
% \qquad \= \quad \kill
% $\TT{e_1[e_2] := e_3} \rightarrow$\\
% \>$\TT{e_3}$\\
% \>{\tt push}\\
% \>$\TT{e_2}$\\
% \>{\tt push}\\
% \>$\TT{e_1}$\\
% \>{\tt load-array}\\
% \>{\tt store}$\qquad\qquad\text{iff}\;e_1[e_2]\;\text{is a constrainable field}$
% \end{tabbing}
The constrainable variable dereferencing operator is directly
translated to a {\tt fetch} instruction.
%
\begin{tabbing}
\qquad \= \quad \kill
$\TT{\text{\bf !} e} \rightarrow$\\
\>$\TT{e}$\\
\>{\tt fetch}
\end{tabbing}
%
Function calls are translated to code which first saves the return
point in the continuation chain, then evaluates and pushes all
arguments onto the stack, evaluates the function expression and calls
the code from the resulting closure.
%
\begin{tabbing}
\qquad \= \quad \kill
$\TT{f(a_1,\dots,a_n)} \rightarrow$\\
\>{\tt save-continuation} $l$\\
\>$\TT{a_1}$\\
\>{\tt push}\\
\>\dots{}\\
\>$\TT{a_n}$\\
\>{\tt push}\\
\>$\TT{f}$\\
\>{\tt call }\\
$l:$
\end{tabbing}
%
Function calls in {\em tail position} (that is, function calls which
are the last thing a function does before it returns) do not set up a
continuation for returning, but reuse the current continuation, which
was set up by the calling function. That means that these calls simply
omit the {\tt save-continuation} instruction necessary for non-tail
calls.
%
\begin{tabbing}
\qquad \= \quad \kill
$\TT{f(a_1,\dots,a_n)} \rightarrow$\\
\>$\TT{a_1}$\\
\>{\tt push}\\
\>\dots{}\\
\>$\TT{a_n}$\\
\>{\tt push}\\
\>$\TT{f}$\\
\>{\tt call }\\
\end{tabbing}
%
Function expressions are translated to the code for the body of the
function, enclosed in function prologue and epilogue code.  Then a
closure is constructed which references this code.  The code for the
function body is simply placed at the current position of the
generated instruction stream, so the jump around the body is
necessary.  In the actual implementation, the function bodies are
collected during translation and placed at the end of the
instructions.  This saves one jump instruction per function creation.
%
\begin{tabbing}
\qquad \= \quad \kill
$\TT{\text{\bf fun}\;(x_1,\dots,x_n)\; e;\dots\;\text{\bf end}} \rightarrow$\\
\>{\tt jump} $l_2$\\
$l_1:$\>{\tt make-environment} $n$\\
\>$\TT{e;\dots}$\\
\>{\tt restore-continuation}\\
$l_2:$\>{\tt make-closure} $l_1$
\end{tabbing}
%
The translation of constraint expressions is nearly identical to that
of function expressions.  The only difference is that all constraint
statements inside of the user-defined constraint need not re-solve the
constraint store after adding constraints.  Re-solving can be avoided
because user-defined constraints can only be invoked from constraint
statements which will re-solve the constraint store anyway.
%
\begin{tabbing}
\qquad \= \quad \kill
$\TT{\text{\bf constraint}\;(x_1,\dots,x_n)\; e;\dots\;\text{\bf end}} \rightarrow$\\
\>{\tt jump} $l_2$\\
$l_1:$\>{\tt make-environment} $n$\\
\>$\TT{e;\dots}$\\
\>{\tt restore-continuation}\\
$l_2:$\>{\tt make-closure} $l_1$
\end{tabbing}
%
{\bf require} statements with body statements are the most complicated
statements to compile, because they require the management of the
constraint store.  Before entering the body of the statement, the
constraints are created and added to the store, and upon exit from the
body, they are removed.  An additional complication arises from the
fact that the constraints must be removed from the store not only when
the body finishes normally, but also when an exception is raised
during the execution of the body.  The following translation
accomplishes this by setting up an exception handler before entering
the body which removes the constraints before re-raising the
exception.

\begin{tabbing}
\qquad \= \quad \kill
$\TT{\text{\bf require}\; c_1:s_1\wedge\dots\wedge c_n:s_n\;\text{\bf in}\; e;\dots\;\text{\bf end}} \rightarrow$\\
\>{\tt make-constraint} $c_1, s_1$\\
\>{\tt store-variable} $\mathcal{L}\lsq t_1\rsq$\\
\>\dots\\
\>{\tt make-constraint} $c_n, s_n$\\
\>{\tt store-variable} $\mathcal{L}\lsq t_n\rsq$\\
\>{\tt add-constraints} $\mathcal{L}\lsq t_1\rsq,\dots,\mathcal{L}\lsq t_n\rsq$\\
\>{\tt handle} $l_1$\\
\>$\TT{e;\dots}$\\
\>{\tt remove-constraints} $\mathcal{L}\lsq t_1\rsq,\dots,\mathcal{L}\lsq t_n\rsq$\\
\>{\tt unhandle}\\
\>{\tt jump} $l_2$\\
$l_1:$\>{\tt remove-constraints} $\mathcal{L}\lsq t_1\rsq,\dots,\mathcal{L}\lsq t_n\rsq$\\
\>{\tt raise}\\
$l_2:\qquad\qquad\qquad\qquad\qquad\qquad\qquad\qquad \text{where}\; t_1,\dots,t_n\;\text{are generated variables}$\>
\end{tabbing}
%
When {\bf require} statements do not have a body, the creation of an
exception handler is not necessary, so the transformation consists
only of the creation of the symbolic constraint representation and the
addition of the constraints to the store.
%
\begin{tabbing}
\qquad \= \quad \kill
$\TT{\text{\bf require}\; c_1:s_1\wedge\dots\wedge c_n:s_n\;} \rightarrow$\\
\>{\tt make-constraint} $c_1, s_1$\\
\>{\tt store-variable} $\mathcal{L}\lsq t_1\rsq$\\
\>\dots\\
\>{\tt make-constraint} $c_n, s_n$\\
\>{\tt store-variable} $\mathcal{L}\lsq t_n\rsq$\\
\>{\tt add-constraints} $\mathcal{L}\lsq t_1\rsq,\dots,\mathcal{L}\lsq t_n\rsq\qquad \text{where}\; t_1,\dots,t_n\;\text{are generated variables}$
\end{tabbing}
%
{\bf while} and {\bf if} statements are translated straightforwardly
using conditional and unconditional jump instructions.
%
\begin{tabbing}
\qquad \= \quad \kill
$\TT{\text{\bf while}\; c\;\text{\bf do}\; e;\dots\;\text{\bf end}} \rightarrow$\\
\>{\tt jump} $l_2$\\
$l_1:$\>$\TT{e;\dots}$\\
$l_2:$\>$\TT{c}$\\
\>{\tt jump-if-true} $l_1$
\end{tabbing}

\begin{tabbing}
\qquad \= \quad \kill
$\TT{\text{\bf if}\; c\;\text{\bf then}\; e_1;\dots\;\text{\bf else}\; e_2;\dots\;\text{\bf end}} \rightarrow$\\
\>$\TT{c}$\\
\>{\tt jump-if-false} $l_1$\\
\>$\TT{e_1;\dots}$\\
\>{\tt jump} $l_2$\\
$l_1:$\>$\TT{e_2;\dots}$\\
$l_2:$
\end{tabbing}
%
The {\bf return} statement simply calculates the function result (left
in the register {\em acc}, where the calling function expects it) and
restores the continuation saved by the caller.
%
\begin{tabbing}
\qquad \= \quad \kill
$\TT{\text{\bf return}\; e} \rightarrow$\\
\>$\TT{e}$\\
\>{\tt restore-continuation}
\end{tabbing}

%%% Local Variables: 
%%% mode: latex
%%% TeX-master: "da.tex"
%%% End: 

%% End of turtle.tex.


%% turtle-impl.tex -- Chapter on the implementation of Turtle
%%
%% Copyright (C) 2003 Martin Grabmueller <mgrabmue@cs.tu-berlin.de>

\chapter{The Implementation of \turtle{}}
\label{cha:turtle-impl}

This chapter describes the implementation of the \turtle{} compiler
and the \turtle{} run-time system.  First we will give an introduction
to the compilation process, then the various modules of the compiler
are briefly described.  We will not go into detail for most parts of
the compiler, because they are fairly straightforward and well treated
in literature~\cite{appel98moderncompiler,grune00compilerdesign}.
Only the more interesting algorithms and concepts will be covered
explicitly.  Then we will present the structure of the run-time system
and document various aspects of the implementation, such as memory
management algorithms and constraint handling.  The chapter concludes
with a discussion of the implementation techniques and presents some
statistical data about the generated code and performance of \turtle{}
programs.


\section{The \turtle{} Compiler}
\label{sec:turtle-compiler}

In this section the \turtle{} compiler---as currently implemented---is
described.  Its task is the translation of \turtle{} source code into
executable programs.  As for most modern high-level languages, the
generated programs require substantial run-time support (for memory
management etc.) from a library which gets linked into the final
program (either statically or via a run-time link loader).  The
run-time system will be described in the next section, but it is very
important that the compiler and the run-time system agree on the
run-time in\-ter\-face (e.g.~data representation and function call
conventions) because the compiler must generate code which works with
the run-time system when executing the program.


\subsection{Compiler Description}
\label{sec:compiler-description}

The \turtle{} compiler is implemented as a library which exports a
single function.  This function expects as input the name of the input
file and a data structure containing the compilation options, such as
optimization switches, and paths to module directories.  The function
translates the input file to either an object file (for library
modules) or to an executable (for main programs) and returns a success
indicator to the caller.  The library can be linked into any
application, so it could be used for implementing an integrated
development environment.  Currently only a command line compiler is
available, which allows to set the compilation options on the command
line.

\begin{figure}[htp]
\begin{center}
\input{compiler.epic}
\end{center}
\caption{\turtle{} compiler and internal representations}
\label{pic:compiler}
\end{figure}

\index{scanner}
\index{token}
\index{lexical analysis}
%
Figure~\ref{pic:compiler} illustrates the architecture of the
\turtle{} compiler, which can be divided into an analysis and a
synthesis part.  The first component of the analysis, the {\em
  scanner}, reads the source code from an external file and groups the
individual characters into lexical entities, so-called {\em tokens}.
Layout and comments are removed and keywords are separated from other
identifiers.  The scanner recognizes the regular part of the \turtle{}
syntax and attaches source code locations to the individual tokens.
This information is important to provide precise error messages in the
parsing and semantic checking phases and for generating debugging
information to be used with the final program during run-time, for
example for displaying back-traces on exceptions.  The following
phases of the compiler maintain the source code location information
in their respective program representations.

\index{recursive descent}
\index{abstract syntax tree}
\index{AST}
%
The token stream produced by the scanner is then processed by the
parser, which recognizes the context-free structure of the syntax and
creates an {\em abstract syntax tree} (AST).  This abstract syntax
tree is a 1-to-1 mapping from source code into a tree structure, and
only very few syntactic transformations are performed in this part of
the compiler.\footnote{Currently, only {\bf elsif} constructs are
  expanded into nested {\bf if}-statements.}  The parser used is a
hand-written {\em recursive descent} parser.

\index{semantic analysis}
\index{overloading resolution}
%
The parsing phase is followed by the semantic analysis.  This phase
takes the AST as its input and checks the context-dependent
correctness of the program and relates the identifiers in the program
to the entities (variables, functions, etc.) they refer to.  Because
of the overloading provided by the language, these relations depend on
the types of identifiers and expressions.  Furthermore, the module
concept of \turtle{} requires that imported modules are made available
in the global name-space of the module to be compiled.  In this step,
also the instantiation of module parameters is performed as described
in section~\ref{sec:overload-resolution}.

\index{name mangling}
%
When the analysis is finished, the compiler has an internal
representation of the source program which is syntactically and
semantically correct\footnote{Of course only as far as the static
  semantics is concerned, the dynamic semantics can only be checked
  when the program is actually run.} and in which every identifier is
unambiguously related to the entity it stands for.  All nodes in the
program graph are annotated with the types of the expressions they
represent.  For all identifiers, a unique name has been generated,
either by {\em name mangling}\footnote{Generating a name which
  includes the original name and an alphanumeric representation of its
  type.} for global variables, functions or constraints, or by
appending a unique number to the name for local variables, functions
and constraints.


\index{HIL}
\index{high-level intermediate language}

This intermediate tree-like representation of the program text is
called {\em high-level intermediate language} (HIL).  It is the input
to the synthesis phase of the compiler.

Before being passed on, the HIL could be optimized by a high-level
optimizer module which could perform procedure integration, common
subexpression elimination and other optimizations.  Since this is
currently not implemented, this module is represented by a dotted box.

\index{LIL}
\index{low-level intermediate language}
%
The synthesis phase is responsible for translating the HIL to the
target code, which in the case of the current \turtle{} implementation
is the programming language ANSI \cee{}.  This translation is
implemented in two steps.  First, the HIL is translated by the code
generator to a {\em low-level intermediate language}\footnote{The idea
  to use several different intermediate representations for
  compilation comes from~\cite{muchnick97advancedcompiler}, where
  three intermediate representations are used (HIR, MIR and LIR).}
(LIL) which has a form similar to assembler language, but is machine
independent.  Here are located some machine-independent optimizations,
known as {\em peep-hole optimization}, such as collecting heap
overflow checks at the start of basic blocks.

In the second step of the synthesis, the LIL is translated to \cee{}
code, which is then compiled to machine code by a \cee{} compiler.

\begin{figure}
\begin{center}
\input{compiling.epic}
\end{center}
\caption{Compilation of \turtle{} programs}
\label{pic:compiling}
\end{figure}

Figure ~\ref{pic:compiling} shows the various phases of the compiler
together with their in- and output files and describes how the
generated \cee{} code is processed further.  The input to the
compiler, the source code file, is shown on the upper left.  It is
read into the scanner and processed by the parser.  The semantic
analysis must check references to external identifiers, so the context
checker must know the interfaces of all imported modules.  These are
read from the interface files of the modules, shown on the left.

After the successful semantic analysis the interface file for the
currently compiled source module is written out, which includes all
module dependencies, the optional module parameters and the signature
of all exported types, functions, variables and constraints.  This
interface file is shown on the right of the context check/overload
resolution module.

After the optional optimization step, the HIL is given to the code
generator, producing optimized LIL code.  The optimized LIL code is
then passed to the code emitter, which writes out a \cee{} source code
file.

\index{C compiler}
\index{linker}
\index{object code}
\index{executable}

The \cee{} source code is translated to object code by a \cee{}
compiler, which is not part of the \turtle{} implementation.  Since
the \cee{} compiler needs to know about global variables and
functions, it is given the header files for the imported modules.  The
result of this compilation step is an object file.  When compiling a
program and not only a library module, the linker is invoked to link
the newly created object file, the object files of the imported
modules and the \turtle{} run-time library to form the executable
program. The linking step is not necessary when compiling a library
module and is therefore put into a dotted box in the figure.

For illustrating the various stages of the compiler, the intermediate
representations for the \turtle{} program~\ref{prog:mini-prog} will be
shown.

\begin{Program}
\begin{ttlprog}
1\>\ttlModule{} example;\\
\\
2\>\ttlPublic{} \ttlFun{} main(argv: \ttlList{} \ttlOf{} \ttlString{}): int\\
3\>\>\ttlVar{} x: int $\leftarrow$ 1;\\
4\>\>\ttlVar{} y: int $\leftarrow$ 2;\\
5\>\>x $\leftarrow$ y * x;\\
6\>\>\ttlReturn{} 0;\\
7\>\ttlEnd{};
\end{ttlprog}
\caption{Compilation example}
\label{prog:mini-prog}
% module example;

% public fun main(argv: list of string): int
%   var x: int := 1;
%   var y: int := 2;
%   x := y * x;
%   return 0;
% end;
\end{Program}

The HIL representation for this program is shown in
program~\ref{prog:mini-prog-hil}.  All type information has been made
explicit and all identifiers have been made unique by either name
mangling (for the function name {\em main}) or by appending unique
numbers (for the local variables)

\begin{Program}
\begin{ttlprog}
1\>.\ttlFun{} example\_main\_pF1pLpS\_pI\\
2\>\>.param argv\_0\\
3\>\>.local x\_1\\
4\>\>.local y\_2\\
\\
5\>\>x\_1:int $\leftarrow$ 1:int\\
6\>\>y\_2:int $\leftarrow$ 2:int\\
7\>\>x\_1:int $\leftarrow$ (y\_2:int * x\_1:int):int\\
8\>\>\ttlReturn{} 0
\end{ttlprog}
\caption{Compilation example -- HIL}
\label{prog:mini-prog-hil}
\end{Program}

Finally, Program~\ref{prog:mini-prog-lil} contains the LIL code for
the example program. In this representation all nested expressions
have been put into a form suitable for machine execution.  All
references to local variables have been translated to references to
the environment and the expressions have been unnested.  Additionally,
code for checking for heap overflow and environment creation has been
generated.  LIL is similar to the TAM machine language presented in
Chapter~\ref{cha:turtle}, but has more instructions so that generation
of efficient code is easier. Other instructions are necessary because
of the restriction of real machines. For example, the {\tt gc-check}
instruction in LIL checks for heap overflow, which is not necessary in
an abstract machine with conceptionally unbounded memory.

\begin{Program}
\begin{ttlprog}
1\>L0:\\
2\>\>gc-check  \#4\\
3\>\>make-env  \#1, \#2\\
4\>\>load-int   \#1\\
5\>\>store     env(0, 1)\\
6\>\>load-int  \#2\\
7\>\>store     env(0, 2)\\
8\>\>load      env(0, 2)\\
9\>\>push\\
10\>\>load      env(0, 1)\\
11\>\>mul\\
12\>\>store     env(0, 1)\\
13\>\>load-int  \#0\\
14\>\>restore
\end{ttlprog}
\caption{Compilation example -- LIL}
\label{prog:mini-prog-lil}
\end{Program}

\subsection{Overload resolution}
\label{sec:overload-resolution}

\index{overload resolution}
%
Identifiers in \turtle{} programs can be defined multiple times in the
same scope, as long as the applications of these identifiers can be
unambiguously assigned to their declarations.  This means that it must
be possible to either distinguish identifiers by context (type names
can only occur in type expressions; variables, functions or
constraints only in expressions) or by type.  It is not allowed to
have more than one data type with the same name in a scope, and
variables, functions and constraints must be distinguishable by their
type.

\index{Ada}
%
In contrast to other programming languages like \java{} or
\cplusplus{}, variables and the return types of functions can be
overloaded, as for example in \ada{}.  The implemented overload
resolution algorithm, a simplified variant of Baker's algorithm as
presented in~\cite{bilsonOverload} (originally described
in~\cite{baker82overload}), is capable of making the right assignments
by examining the context of the identifiers.  The basic idea is to
collect for each expression the set of possible interpretations (e.g.
for an identifier all variables or functions with that name) in a
bottom-up traversal of the syntax tree, and to eliminate as soon as
possible all interpretations which lead to wronlgy typed or otherwise
non-sensical interpretations (for example, assignment to the name of a
function).  If at the top-level of statement expressions more than one
interpretation arrives, the expression is ambiguous, and if the set of
interpretations is empty, a semantic error has been found, for example
the use of an undefined identifier or a type error.

Overloading is also used for type-checking parametrized modules.
\turtle{} requires the programmer to give actual types as parameters
to imported parametrized modules.  When adding the public declarations
of these modules to the scope of the importing program, all occurences
of type parameters in these declarations are replaced by their
corresponding actual parameters.  Thus, when a module is imported
multiple times with different actual parameters, the definitions of
this module are imported several times, with different types.

For example, consider the following {\bf import} statement:

%import listmap<int, string>, listmap<char, int>;
\begin{ttlprog}
\>\ttlImport{} listmap$<$int, \ttlString{}$>$, listmap$<$char, int$>$;
\end{ttlprog}
%
The importing module will see the following function declarations:
%
%public fun map(f: fun(A): B, l: list of A): list of B
%public fun map(f: fun(int): string, l: list of int): list of string
%public fun map(f: fun(char): int, l: list of char): list of int
\begin{ttlprog}
\>\ttlFun{} map(f: \ttlFun{}(int): \ttlString{}, l: \ttlList{} \ttlOf{} int): \ttlList{} \ttlOf{} \ttlString{}\\
\>\ttlFun{} map(f: \ttlFun{}(char): int, l: \ttlList{} \ttlOf{} char): \ttlList{} \ttlOf{} int
\end{ttlprog}
%
With these multiple declarations of function {\em map}, the normal
overload resolution algorithm can type-check applications of this
function to arguments of different types.  Thus, with a simple
substitution when importing module declarations, parametrized modules
can be type-checked with a semantic analyzer which does not know
anything about parametric polymorphism at all!

\subsection{Code Generation}
\label{sec:code-generation}

\index{code generation}

The translation from HIL to LIL is performed by the code generation
module.  While HIL is a tree-like representation, LIL is a flat,
assembler-like language which only supports simple stack or register
operations and labels for naming locations in the instruction
sequence.  The code generation algorithm is similar to that by Dybvig%
\index{Dybvig},
Hieb%
\index{Hieb} and Butler%
\index{Butler}~\cite{dybvig90destination}.  Their article shows how a
relatively simple compiler without data or control flow analysis can
generate reasonably efficient machine code.  They define a
transformation function, which receives a syntax tree, the location
for the expected result and the target of the control flow (or the two
targets, if it is a boolean expression used in a conditional) as
inputs and produces the generated code.  By these means, unnecessary
register moves and jumps to jump instructions can be avoided.  In many
cases, the need for an additional optimization phase for cleaning up 
the machine code (often called {\em peep-hole optimization}%
\index{peep-hole optimization}) does not arise when using this
algorithm.

Nevertheless, the \turtle{} compiler contains a simple peep-hole
optimizer, which for example combines all the heap-overflow checks in
basic blocks into one at the beginning of the block.  This
optimization could only be integrated with the code generation
algorithm if an additional analysis would be added which determines in
advance how much memory is allocated in any basic block, so that the
code generator knows the amount of memory when it emits the first
heap-overflow check of a basic block.  Our solution seems to be much
simpler, but better analysis could find more opportunities for
optimizing these overflow checks.

Some complications in the code generation for \turtle{} programs are
described below in section~\ref{sec:tail-recursion-elimination}.


\subsection{Closure Representation}
\label{sec:closure-representation}

\index{closure}
\index{free variable}

Since \turtle{} is a higher-order language where functions can be
arguments and results of functions and where functions can be stored
into arbitrary data structures, functions must be represented
internally in such a way that every function, when called, can access
its free variables.  Free variables can be global variables or
variables declared in lexically enclosing functions.  For global
variables, no special care must be taken because they can be accessed
by simply using their address in the data segment.  Local variables
(including parameters) need to be treated differently, because there
may be multiple copies of each local variable and because local
variables may live longer than the function invocation during which
they were created.  Therefore, for each definition of a local (named
or anonymous) function, a data structure is created which contains a
pointer to the machine code to execute when calling the function, and
a pointer to the environment in which the local variables are stored.
Since each environment contains a pointer to the enclosing
environment, all free variables can be accessed following this pointer
chain.

For global functions, the run-time representation is simpler, because
there are no enclosing environments for top-level function, and thus
the only free variables can be global variables.  Global functions are
simply represented by a pointer to their entry point.

\index{Appel}
%
The advantage of the representation of closures as two-element records
is the implementation simplicity.  A possible problem is that this
representation, combined with linked environments can lead to an
increase in asymptotic space complexity, as noted by
Appel~\cite{appel92compilingwithcontinuations}.

\subsection{Tail Recursion Elimination}
\label{sec:tail-recursion-elimination}

\index{tail recursion}
%
All functions of one source module are compiled to a single \cee{}
function. As an example, consider the compilation of the two mutually
recursive functions {\em odd} and {\em even} in
program~\ref{prog:odd-even}.  These functions determine whether an
integer is odd or even by repeatedly calling themselves until the
integer becomes zero.  The \cee{} code resulting from compilation with
the \turtle{} compiler is shown in figure~\ref{prog:c-odd-even}.  The
code was taken directly from the output of the compiler but has been
cleaned up for better presentation.\footnote{The labels L1, L4 and L6
  were used for interrupt checks and have been deleted.  These
  instructions are inserted by the compiler at each function and loop
  beginning and check for operating system signals.}

The generated \cee{} code illustrates that the two functions from the
\turtle{} program have been combined into the function {\em
  host\_procedure}.  When the function is called, a pointer (to the
descriptor for the \turtle{} function to be called) is passed in the
variable {\em pc}, and the {\bf switch} statement determines which
case of the statement should be executed.  In the example, case 0 has
been allocated to the function {\em odd} and case 5 for the function
{\em even}.  Because of the combination of two source functions into
one \cee{} function, the function {\em odd} can call the function {\em
  even} by simply pushing the parameter onto the evaluation stack and
jumping to label L5 (line 35).  The other way around works similarly
(line 58).

\begin{Program}
\begin{ttlprog}
1\>\ttlFun{} odd($i$: int): bool\\
2\>\>\ttlIf{} $i = 0$ \ttlThen{}\\
3\>\>\>\ttlReturn{} \ttlFalse{};\\
4\>\>\ttlElse{}\\
5\>\>\>\ttlReturn{} even ($i$ $-$ 1);\\
6\>\>\ttlEnd{};\\
7\>\ttlEnd{};\\
\\
8\>\ttlFun{} even ($i$: int): bool\\
9\>\>\ttlIf{} $i = 0$ \ttlThen{}\\
10\>\>\>\ttlReturn{} \ttlTrue{};\\
11\>\>\ttlElse{}\\
12\>\>\>\ttlReturn{} odd ($i$ $-$ 1);\\
13\>\>\ttlEnd{};\\
14\>\ttlEnd{};
\end{ttlprog}
\caption{Functions {\em odd} and {\em even} in \turtle{}}
\label{prog:odd-even}
% fun odd(i: int): bool
%   if i = 0 then
%     return false;
%   else
%     return even (i - 1);
%   end;
% end;

% fun even (i: int): bool
%   if i = 0 then
%     return true;
%   else
%     return odd (i - 1);
%   end;
% end;
\end{Program}

\begin{Program}
{\scriptsize
\begin{ttlprog}
1\>static int\\
2\>host\_procedure (void)\\
3\>\{\\
4\>\>ttl\_value acc;\\
5\>\>ttl\_value * sp;\\
6\>\>ttl\_value * alloc;\\
7\>\>ttl\_environment env;\\
8\>\>ttl\_descr pc;\\
9\>\>ttl\_closure self = NULL;\\
10\>\>TTL\_RESTORE\_REGISTERS;\\
11\> L\_jump:\\
12\>\>switch (pc $-$ descriptors)\\
13\>\>\>\{\\
14\>\>\>case 0:\\
15\>L0:\\
16\>\>\>\>\>env = NULL;\\
17\>\>\>\>\>TTL\_GC\_CHECK (10);\\
18\>\>\>\>\>TTL\_MAKE\_ENV (1, 0);\\
19\>\>\>\>\>env$-$$>$locals[0] = *($-$$-$sp);\\
20\>\>\>\>\>acc = env$-$$>$locals[0];\\
21\>\>\>\>\>TTL\_PUSH();\\
22\>\>\>\>\>acc = TTL\_INT\_TO\_VALUE (0);\\
23\>\>\>\>\>if (*($-$$-$sp) {\bf!}= acc) goto L3;\\
24\>L2:\\
25\>\>\>\>\>acc = TTL\_FALSE;\\
26\>\>\>\>\>TTL\_RESTORE\_CONT;\\
27\>\>\>\>\>break;\\
28\>L3:\\
29\>\>\>\>\>acc = env$-$$>$locals[0];\\
30\>\>\>\>\>TTL\_PUSH();\\
31\>\>\>\>\>acc = TTL\_INT\_TO\_VALUE (1);\\
32\>\>\>\>\>acc = (ttl\_value) ((((int)*($-$$-$sp)) $-$ ((int)acc)));\\
33\>\>\>\>\>TTL\_PUSH();\\
34\>\>\>\>\>env = TTL\_VALUE\_TO\_OBJ (ttl\_environment, env$-$$>$parent);\\
35\>\>\>\>\>goto L5;\\
36\>\>\>case 5:\\
37\>L5:\\
38\>\>\>\>\>env = NULL;\\
39\>\>\>\>\>TTL\_GC\_CHECK (10);\\
40\>\>\>\>\>TTL\_MAKE\_ENV (1, 0);\\
41\>\>\>\>\>env$-$$>$locals[0] = *($-$$-$sp);\\
42\>\>\>\>\>acc = env$-$$>$locals[0];\\
43\>\>\>\>\>TTL\_PUSH();\\
44\>\>\>\>\>acc = TTL\_INT\_TO\_VALUE (0);\\
45\>\>\>\>\>if (*($-$$-$sp) {\bf!}= acc) goto L8;\\
46\>L7:\\
47\>\>\>\>\>acc = TTL\_TRUE;\\
48\>\>\>\>\>TTL\_RESTORE\_CONT;\\
49\>\>\>\>\>break;\\
50\>L8:\\
51\>\>\>\>\>pc = descriptors $+$ 8;\\
52\>\>\>\>\>acc = env$-$$>$locals[0];\\
53\>\>\>\>\>TTL\_PUSH();\\
54\>\>\>\>\>acc = TTL\_INT\_TO\_VALUE (1);\\
55\>\>\>\>\>acc = (ttl\_value) ((((int)*($-$$-$sp)) $-$ ((int)acc)));\\
56\>\>\>\>\>TTL\_PUSH();\\
57\>\>\>\>\>env = TTL\_VALUE\_TO\_OBJ (ttl\_environment, env$-$$>$parent);\\
58\>\>\>\>\>goto L0;\\
59\>\>\>\}\\
60\>\>TTL\_SAVE\_REGISTERS;\\
61\>\}
\end{ttlprog}
}
\caption{Generated C code for {\em odd} and {\em even}}
\label{prog:c-odd-even}
\end{Program}

This compilation scheme makes it possible to compile calls to known
functions in the same module to direct jump instructions, yielding
extremely efficient function calls.  Unfortunately, this scheme cannot
be employed for function calls across module boundaries, because it is
not possible to compile portable \cee{} code with this property.

\index{Scheme}
\index{Gambit}
\index{Feeley}
%
In order to achieve proper tail-recursive function calls, even across
module boundaries, the \turtle{} compiler uses a compilation scheme
similar to the one used in the \gambit{} \scheme{} compiler for its
\cee{} back-end.  Feeley et al.  describe how to compile languages
with higher-order functions to portable
\cee{}~\cite{feeley97compilingtoc}.

\begin{figure}[htp]
\begin{center}
\input{dispatcher.epic}
\end{center}
\caption{Function call implementation}
\label{pic:dispatcher}
\end{figure}

Figure~\ref{pic:dispatcher} shows how this compilation scheme works
for non-local function calls.  On the left, the descriptor table which
is generated for each module is shown.  This table contains an entry
for each function entry point and each return location, and each table
entry points to the {\em host function}, that is the \cee{} function
which contains its code.  The variable {\em initial\_pc} points to the
first element of the table and indicates the entry point for the first
function to be called.  Below the descriptor table, a closure record
is shown.  The first word of a closure record also contains a pointer
to the host function.  The reason is that it is possible to fetch the
first word of a closure record or descriptor table entry and to get
the correct host function, so that global functions do not need
closure records but can instead point directly into the descriptor
table.

In the middle of the figure, the host function is shown.  It
calculates from the descriptor in the global variable {\em pc} the
index of the function which is to be called and then jumps to the
corresponding program code with a {\bf switch} statement.  When a
function calls another function, it first saves the descriptor of the
return point on a stack, loads the target function into {\em pc} and
returns from the host function.  When the called function returns, it
takes the saved {\em pc} from the stack and returns from its host
function, so that the dispatch loop will jump to the return point.

Closures are distinguished from normal descriptors by the fact that
their address does not point into the descriptor table.  This will
cause the {\bf switch} statement to jump to its {\bf default} branch.
Since the real descriptor address for the closure code is stored in
the closure together with the captured environment of the closure,
both can be fetched from the closure object and by jumping to the
beginning of the host function, the correct code will be executed.

This implementation of closures has the advantage that normal
functions and closures can be called in the same way, but at a
slightly higher cost of calling closures.

A host function runs as long as only function calls inside a module
are made.  Whenever a function from another module is to be called or
whenever a function must return to a function in another module, the
host function is left.  This ensures that no stack space is needed for
these function calls or returns, but it requires a means to call the
host function of the next \turtle{} module to be called.  This is the
task of the main loop.

The main loop (called {\em dispatcher} in Figure~\ref{pic:dispatcher})
initializes the program counter and calls the host function associated
with the program counter via the descriptor table.  When the host
function returns, it has placed the program counter to call next in
the variable {\em pc} which will be called in the next loop iteration.
This continues until the program terminates by calling an operating
system call for this purpose.

Feeley et al.~\cite{feeley97compilingtoc} report that the \cee{} code
generated in this way runs at half the speed compared to the machine
code generated directly by the native code back-end of the Gambit
compiler.  Slower code is a clear disadvantage of this approach, but
the generation of portable \cee{} has the obvious advantage that the
\turtle{} system can compile code for various architectures without
any work for porting.

Additionally, the machine code produced by the \cee{} compiler is
better than one would expect, since \cee{} compilers often include
sophisticated optimizations which would be hard to achieve when
implementing them again for the \turtle{} compiler.


\subsection{Compiling Constraint Statements}
\label{sec:compiling-constraints}
\index{compilation!constraints}
\index{constraint!compilation}
\index{constraint compilation}

The compiler distinguishes two kinds of constraints in constraint
statements: first, we have trivial constraints which do not contain
any references to constrainable variables.  These constraints are
translated like normal boolean expressions, followed by a test whether
the result was true or false.  If the result was false and the
constraint was required, an exception is raised.

For non-trivial constraints, compilation is more complicated.  Since
constraints are first-class objects (they have to remain accessible to
the constraint solver until the scope of the constraints is left), a
representation of the constraint must be built.  This representation
must contain references to the constrainable variables so that the
constraint solver can fetch the values of these variables and can
store new values into them.  Recall that references to normal
variables, function calls and constant subexpressions are treated as
constants, so that only the results of evaluating them have to be
stored in the constraint representation.  After building the
representation, the constraint is tagged with its strength (0 for
required constraints, and values greater than 0 for preferential
(non-required) constraints) and added to the constraint store.  Adding
the constraint will cause the constraint solver to re-solve the store.
If any required constraints in the store cannot be satisfied after
adding the new constraint, the new constraint will be removed and an
exception will be raised.  If any preferential constraints are not
satisfied, the solver tries to satisfy as many preferential
constraints as possible, but no exception will be raised.

For illustration of the translation of constraint statements, we will
translate the following code fragment step by step:

\begin{Program}
% var x: ! int := var 0;
% var y: int := 4;
% require 10 * x > 3 * y - 1;
\begin{ttlprog}
1\>\ttlVar{} x: {\bf!} int $\leftarrow$ \ttlVar{} 0;\\
2\>\ttlVar{} y: int $\leftarrow$ 4;\\
3\>\ttlRequire{} 10 * x + 10 $>$ 3 * y $-$ 1;
\end{ttlprog}
\caption{Constraint compilation example}
\label{prog:constraint-compilation-example}
\end{Program}

Line 1 declares the variable $x$ as a constrainable integer variable,
so the compiler will need to translate the {\bf require} statement as
a non-trivial constraint statement.  The translation proceeds by first
partitioning the terms of the constraint into constants (that includes
non-constrainable variables and function calls, which are evaluated
before the constraint is created) and constrainable variables,
together with their coefficients.  For our example, we have the
constant expression 
$$ -10+3*y - 1$$
and the constrainable variable with coefficient
$$10*x$$

The translation of the {\bf require} statement consists of first
pushing the constraint's strength onto the evaluation stack.  For our
example, since no strength was specified, 0 (the strongest strength)
is assumed.  After that, an indicator for the kind of constraint (the
inequality `$>$') is pushed, followed by the number of constrainable
variables (1 in the example). Then the value of the constant term
(which must be evaluated before the constraint is created) is pushed
onto the stack, followed by all constrainable variables with their
corresponding coefficients.
Program~\ref{prog:example-constraint-code} shows the generated code
for the example.

\begin{Program}
\begin{tabular}{ll}
load-constant & 0 \qquad{\em // constraint strength}\\
push&\\
load-constant & 3 \qquad{\em // constraint kind `$>$'}\\
push&\\
load-constant & 1 \qquad{\em // number of constrainable variables}\\
push&\\
load-variable & y \qquad{\em // calculate the constant term\dots}\\
push&\\
load-constant & 3\\
mul&\\
push&\\
load-constant & -11\\
add&\\
push&\\
load-variable & x \qquad{\em // load the constrainable variable object}\\
push\\
load-constant & 10\\
push\\
add-constraint & \qquad{\em // add the constraint to the store}
\end{tabular}
\caption{Generated code for the constraint example}
\label{prog:example-constraint-code}
\end{Program}

Note that difference between the variables $x$ and $y$. The value of
$y$ is used for calculating the constant term of the constraint,
whereas the value of $x$ (which is a variable object) is pushed onto
the stack so that the constraint solver responsible for the constraint
can access the variable.

Before translating the constraints into code, the compiler checks
whether the constraint is representable in the symbolic
representation.  Non-linear constraints or constraints with relations
not supported by the constraint solvers are rejected by the compiler.


\section{The Run-Time System}

The run-time system comprises the environment for executing the
imperative, functional and constraint code, for example memory
management, in-/output functions, the constraint store and constraint
solver.

Figure~\ref{pic:turtle-program-parts} shows the individual parts of a
\turtle{} program when it is executed.  The code section is the
compiled program code of the source program, the data section holds
the global variables of the \turtle{} program and the run-time system,
and a variable for each global function storing its function
descriptor.  The stack is needed for the run-time system, because it
is written in the \cee{} language.  A separate small stack for
evaluating expressions and holding intermediate results is contained
in the data section.  The code of the run-time system is shown as {\em
  \turtle{} run-time} in the figure.  The heap stores dynamically
created data structures, the environments and continuations of active
\turtle{} functions.  It is divided into two equal-sized parts,
because the garbage collector is a copying collector which moves the
heap data from one space to the other during collection.  The
constraints are held in one or more constraint stores, where they are
accessed and manipulated by the constraint solver component(s) only.

\begin{figure}[htp]
\begin{center}
\input{runtime-schema.epic}
\end{center}
\caption{\turtle{} program components}%
\label{pic:turtle-program-parts}
\end{figure}


\subsection{Memory Management}
\label{sec:memory-management}

\index{memory management}
\index{garbage collection}
\index{copying collection}
\index{stop\&copy collector}
\index{Cheney}
%
The dynamic memory of the \turtle{} system is maintained automatically
by the garbage collector included in the run-time system.  Memory for
dynamic data structures is allocated in the heap.  As soon as the heap
becomes full and an allocation request cannot be satisfied, the
garbage collector is invoked in order to remove all memory cells which
are no longer needed from the heap, thus making them available for new
allocation requests.  The collector employs a simple {\em stop\&copy}
algorithm, as described by Cheney in~\cite{cheney70compaction}
(reference given in~\cite{jones96gc}).

\index{roots}
\index{semi-space}
%
For this algorithm, the heap is divided into two areas of equal size
called semi-spaces.  One of the two halves is used for allocation at a
given time, while the other is unused between collections.  Since a
heap segment is continuous in memory, memory allocation can be
implemented very efficiently by simply incrementing a pointer by the
size of the object to allocate.  As soon as the allocation pointer
reaches the end of the semi-space,\footnote{This can be implemented
  with an explicit comparison between the allocation pointer and the
  end address of the semi-space, or by write-protecting a page at the
  end of the space and catching the trap which will be generated when
  the allocated object in the write-protected page is initialized.
  \turtle{} uses the former method, because it is much more portable,
  easier to implement and efficient enough for our purposes.} all heap
objects which are (transitively) reachable from some storage locations
called {\em garbage collection roots} (such as the stack and the
machine registers) are copied to the other semi-space and the roles of
the two spaces are switched.  All objects which remain in the old
allocation space have not been reachable from the roots and therefore
cannot influence the future program execution, they are called {\em
  garbage}.  Since the old allocation space will become allocation
space again after the next garbage collection, there is no need to
explicitly remove the garbage objects, they will simply be overwritten
by objects allocated in the future.

Cheney's algorithm is extended in the \turtle{} garbage collector to
check how full the allocation semi-space is after each collection, and
if the collection did not free enough memory, the heap is resized.
This ensures that garbage collections will not occur too frequently
and adapts the heap size to the memory requirements of the running
program.


\subsection{Data Representation}  
\label{sec:data-representation}

The data representation specifies how values are represented during a
\turtle{} program's run-time.  The representation scheme used is
similar to the one used in the Scheme~48 system
\cite{kelsey94tractable}, with an optimization suggested in the paper.

Turtle objects are denoted by so-called descriptors, which are either
immediate values, occupying only one word of storage or pointers to
two or more words in the heap.  Figure~\ref{pic:data-rep} shows the
four types of descriptors and how they are related to each other.

\begin{figure}[htp]
\begin{center}
\input{data-rep.epic}
\end{center}
\caption{Data representation}%
\label{pic:data-rep}
\end{figure}

\begin{figure}[htp]
\begin{center}
\input{data-header.epic}
\end{center}
\caption{Header field layout}%
\label{pic:data-header}
\end{figure}

Immediate values are distinguished by their two low bits, which must
be \#b00\footnote{This notation will be used in this section for
  binary numbers.}.  Heap-allocated objects come in two flavours:
pairs (also known as {\em cons cells}%
\index{cons cell} in other languages) are two-word objects (mainly
used for linked list structures) and have \#b10 in the low bits of
their descriptors.  All other heap-allocated objects have \#b01 in
their two lowest descriptor bits.  The descriptors for heap-allocated
objects point to the header field of their objects, when the two
lowest bits are masked out.  That means that all heap-allocated
objects must be aligned on a two-word boundary.  For non-pair cell
pointers, the data area is preceded by a header field, which is one
word in size and contains the tag bits \#b11 in the lowest bits, a
type code in the next 6 bits and a 24-bit unsigned size field in the
upper bits (see Figure~\ref{pic:data-header}).  The header field is
required by the garbage collector to find out which words in the
objects are pointers and how large the object is, so that the objects
can be traced correctly.

The special header tag \#b11 allows us to omit a header field for
pairs, thus reducing the storage requirements for lists by one word
per pair.  The reason for the special tag is that the garbage
collector can check whether a pointer into the heap is pointing to a
pair or another heap-allocated object by simply examining the two
lowest bits of the first word of the object.  If it is \#b11, it is
not a pair, otherwise, it is a pair.  Therefore it is an important
invariant of the run-time system that {\em no} descriptor in the
system which is not a header field is tagged \#b11.


\subsection{Hand-Coding}  
\label{sec:handcoding}

\index{hand-coded}
%
A lot of low-level functions cannot be implemented in Turtle, because
there is no way to access operating system features or the C library
directly.  Therefore, {\em hand-coded} modules have been added to the
Turtle compiler.

A hand-coded module contains function definitions, where the function
body is either left out entirely, or is replaced by a string.  In the
former case, the implementation of the function must then be given in
a separate file as a \cee{} macro which is copied into the \turtle{}
compiler's output, in the latter case, the string must name a \cee{}
function which is defined in the separate file.  In the \cee{} source
(either a macro or a function), calls to the \cee{} library, operating
system calls or calls to the \turtle{} run-time system can be made.
It is the responsibility of the programmer to make sure that no
invariants of the run-time environment are violated.

Another possibility to access the underlying system is the use of {\bf
  foreign} expressions.  The statement
%
\begin{ttlprog}
\>\ttlVar{} x: int $\leftarrow$ {\bf foreign} "TTL\_MAKE\_INTEGER (SIGINT)";
\end{ttlprog}
%
initializes the variable $x$ with the integer value of the \cee{}
preprocessor symbol {\tt SIGINT}.  The argument to {\bf foreign}
expression is simply copied to the generated code in the correct
place.

The usage of hand-coded modules can affect the safety of \turtle{}
programs, so they have to be written with special care and knowledge
of the internals of the run-time system.  In order to avoid program
errors because of hand-coded modules, the \turtle{} compiler must be
explicitly instructed to accept these modules.  When this is not done,
the compiler gives an error when a hand-coded module should be
compiled.

The same functionality could be achieved by providing a possibility to
link object files written in other languages into \turtle{} programs,
but the chosen approach has some advantages: the compiler will create
interface files for hand-coded modules, it will also handle the
dispatch loop and the initialization of variables etc.  All which is
left for the programmer writing such a module is the programming of
the function bodies.

The term ``hand-coding'' and the idea to include this mechanism was
inspired by the implementation of the functional language
\opal{}~\cite{Pepper.Opal}.

\section{Run-Time Constraint Solver}

The \turtle{} run-time system contains constraint solvers for solving
the constraints for which the compiler cannot generate more efficient
straight-line code.  Currently, two constraint solvers are included in
the implementation: a solver for cycle-free linear arithmetic equality
and inequality constraints over the real numbers and a finite-domain
solver ranging over the integers.  The architecture is open for the
inclusion of more powerful solvers since the interface between the
(imperative) run-time system and the constraint solvers is clearly
designed and flexible enough for a wide variety of solvers.

The real number constraint solver used implements the Indigo
algorithm.  Indigo is a local propagation solver which can handle
constraint hierarchies.  It works by narrowing intervals of real
numbers and was developed by Borning, Anderson and Freeman-Benson.
The implementation is based on the description
in~\cite{borning96indigo}.

The finite-domain constraint solver is a simple backtracking solver
which is only optimized to make use of equality and inequality
constraints with one variable and one constant to narrow the domains
of variables before performing an exhaustive search to find solutions.
Constraint hierarchies are currently not supported by the
finite-domain solver.

Both solvers are restricted to linear equations and inequalities over
their respective domains, where the Indigo solver deals with the
relations $=$, $\le$, $\ge$ and the backtracking solver with the
relations $=$, $<$, $>$, $\le$, $\ge$ and $\not=$.  The compiler
checks whether the constraints are syntactically correct and rejects
constraints which the built-in solvers cannot handle.  This knowledge
about the capabilites of the individual solvers is currently
hard-wired into the compiler and would need to be extended before
integrating other constraint solvers.

\subsection{Run-Time-Solver-Interface}

Communication between the imperative virtual machine and the run-time
constraint solver needs to perform several tasks:

\begin{itemize}
\item Create variables on which the solver(s) can operate.
\item Create constraints which tell the solver(s) how to operate on
  the variables.
\item Invoke the solver to obtain solutions.
\item Work with the constrainable variables by extracting their values.
% \item Mark solver-specific data structures as being used and inform
%   the solver(s) about garbage collection, so that unused memory can be
%   freed.
\end{itemize}

Constrainable variables are known to the compiler because of the
constrainable type annotations given by the programmer.  For every
constrainable variable, a slot in the data segment (for global
variables), in the environment of the defining function (for local
variables) or in the data object (for record fields) is reserved.  The
variable objects which must be stored into these slots must be explicitly
initialized by {\bf var} expressions (see
section~\ref{sec:constrainable-vars}).

In Figure~\ref{pic:variable-rep} (a) an environment frame is shown
which corresponds to the environment of a function with the following
local variables, just before the initializations are done:

\begin{ttlprog}
1\>\ttlVar{} x: {\bf!} int $\leftarrow$ \ttlVar{} 0;\\
2\>\ttlVar{} y: int;\\
3\>\ttlVar{} z: {\bf!} int $\leftarrow$ \ttlVar{} 2;
\end{ttlprog}

Figure~\ref{pic:variable-rep} (b) shows the same environment after
initialization has finished.  The variable $y$, which is a normal
variable, is initialized to the value 0, whereas the other variables
are initialized with constrainable variable objects.  These objects
contain a value slot, which is initialized by the operand of the {\bf
  var} expressions and a pointer to a solver-specific data structure,
which holds the information necessary for constraint solving.

\begin{figure}[htp]
\begin{center}
\input{variables.epic}
\end{center}
\caption{Representation of variables and variable objects}%
\label{pic:variable-rep}
\end{figure}

This layout of constrainable variables is the same for all domains and
for all solvers, and only the solver-specific pointer differs for the
various solvers.

In order to store the values obtained by the constraint solvers, one
additional requirement is necessary.  The solver-specific data
structure must reserve a pointer field at the beginning which holds a
pointer back to the constrainable variable object.  This pointer is
used by the garbage collector.

The garbage collector must be able to reclaim constrainable variable
objects and inform the various constraint solvers when their specific
data structures are no longer needed.  This is done by treating
constrainable variable objects like record objects, which are copied
during garbage collection, when they are reachable.  Additionally, the
solver-specific object (which can be found by following the pointer in
the constrainable variable object) is adjusted to point to the copy of
the constrainable variable.  After each garbage collection, the
individual solvers are informed that a collection was done, so that
they can free all variable and constraint structures which are not
marked.

% Until now, we have seen how the necessary data structures are
% maintained by the virtual machine, and how the necessary instructions
% for handling them can be produced by the compiler.  What remains to be
% discussed is how constraints are added to the solvers' constraint
% stores and how the solvers are invoked to check the constraints for
% satisfiability and to produce values for the variables.

Currently, there are two different instructions for constraint
creation, one for constraints over the real numbers and one for
constraints over the integers.  This constraint creation instruction
is then responsible for constructing a solver-specific representation
of the constraint which can then be added to the constraint store of
the solver.  After adding all the constraints of the conjunction to
the solver, an instruction for re-solving the store is emitted.  When
values for the individual variables are determined, the values are
stored into the constrainable variable objects, reachable through the
pointer in the solver-specific data structure mentioned above.

The interface between the run-time system and the constraint solvers
is very simple, nevertheless it is very flexible because of the
two-directional references between constrainable variables and
solver-specific variable representation.

The constraint programming model currently implemented in \turtle{}
only allows a single solution for the constraints, so that the solvers
only determine the first (not necessarily best) solution.  This could
be fixed by adding backtracking to the \turtle{} semantics, as with
Alma-0.  Unfortunately, this would complicate \turtle{}'s semantics
considerably, so it was left out of the language as described in this
work.

Additionally, under-constrained problems are not detected, so that for
constraints over the real numbers, intervals of reals can be found as
solutions, whereas only single real values can be stored into
\turtle{} variables.  When such a situation arises, one value from the
interval is chosen and stored into the variable.

The constraint hierarchy support in \turtle{} is restricted to the
syntax necessary for specifying them and the interface between the
run-time system and the constraint solvers which attaches strengths to
the symbolic constraint representation.  It is the task of the
constraint solvers to make use of the strength information (as the
Indigo solver does) or to ignore it (as in the implemented finite
domain solver).

\section{The Standard Library}

A library containing often-used data structures and functions is very
important if a language is intended to be used in real life.
Therefore, a standard library for \turtle{} has been designed and
implemented in the reference implementation.

The library provides a range of useful data types, such as trees and
hash tables, support modules for the built-in data types for list,
array, string and number manipulation and of course functions for
imperative in- and output.  Additionally, some low-level library
modules have been included for interfacing with the operating system,
such as for process management and network programming.

The library makes intensive use of \turtle{} features such as the
module system, pa\-ra\-me\-trized modules and higher-order functions.
Thus the library implementation was very useful in debugging the
compiler and testing the language design.

The structuring of the library was inspired by the design of the
``Bibliotheca Opalica,'' the standard library of the functional
language Opal~\cite{Pepper.Opal}.

Appendix~\ref{cha:turtle-library} briefly describes the modules of the
standard library. The complete documentation is included in the
\turtle{} implementation (see Appendix~\ref{cha:turtle-compiler} for
information where the reference implementation can be obtained).


% \section{Semantics}

% The following describes the translation scheme for constraint
% statements.  All kinds of {\em require} statements are based on the
% primitive operations {\em assert} and {\em retract} which add a
% (labelled) constraint conjunction to the store or remove a constraint
% conjunction, respectively.

% For all kinds of constraint statements, it is first checked whether
% there are any constrainable variables in any of the constraints.  If
% not, the constrains are simply translated as normal boolean
% expressions and if they are of strength {\em required} and not
% satisfied, an exception is raised.

% \begin{tabbing}
% \qquad \= \qquad \= \qquad \= \kill
% $\mathcal{T}\langle${\bf require} $c_1/l_1 \wedge\dots\wedge c_n/l_n, \mathcal{C}\rangle$ \qquad(if no constrainable variables in $c_n$)\\
% \>$\Rightarrow$\>$\mathcal{T}\langle \text{\bf if not}\, c_1\, \text{{\bf then} {\em raise exception} {\bf end}}\rangle$\\
% \>\>\dots\\
% \>\>$\mathcal{T}\langle \text{\bf if not}\, c_n\, \text{{\bf then} {\em raise exception} {\bf end}}\rangle$
% \end{tabbing}

% The following kind of the {\em require} statement implies an unlimited
% duration for the constraints, so that the constraints are added to the
% constraint store without keeping a reference for later removal.

% \begin{tabbing}
% \qquad \= \qquad \= \qquad \= \kill
% $\mathcal{T}\langle${\bf require} $c_1/l_1 \wedge\dots\wedge c_n/l_n, \mathcal{C}\rangle$\\
% \>$\Rightarrow$\>$assert(c_1/l_1 \wedge\dots\wedge c_n/l_n)$
% \end{tabbing}

% The {\em require-while} variant assert the constraints and keeps a
% reference to the added constraints so that they can be removed as soon
% as the loop is terminated.  But note that the translation of the loop
% must take the constraint reference into account, so that a statement
% which quits the loop prematurely can properly remove the constraints,
% for example when a {\em return} statement is executing in the loop, or
% when an exception is raised.\footnote{This could be implemented by
%   storing the references to active constraint conjunctions in the
%   currently active environment during run time and let the exception
%   mechanism walk the call chain, thereby removing (and retracting) all
%   active constraints between the point where the exception was raised
%   and the handler.}

% \begin{tabbing}
% \qquad \= \qquad \= \qquad \= \kill
% $\mathcal{T}\langle${\bf require} $c_1/l_1 \wedge\dots\wedge c_n/l_n \text{\em WhileStmt\/}, \mathcal{C}\rangle$\\
% \>$\Rightarrow$\>$tmp = assert(c_1/l_1 \wedge\dots\wedge c_n/l_n)$\\
% \>\>$\mathcal{T}\langle\text{\em WhileStmt\/}, \mathcal{C}\cup \{tmp\}\rangle$\\
% \>\>$retract(tmp)$
% \end{tabbing}

% The {\em require-in} statement is nearly the same as the {\em
%   require-while}, so no special actions are necessary.

% \begin{tabbing}
% \qquad \= \qquad \= \qquad \= \kill
% $\mathcal{T}\langle${\bf require} $c_1/l_1 \wedge\dots\wedge c_n/l_n \text{\em InStmt\/}, \mathcal{C}\rangle$\\
% \>$\Rightarrow$\>$tmp = assert(c_1/l_1 \wedge\dots\wedge c_n/l_n)$\\
% \>\>$\mathcal{T}\langle\text{\em InStmt\/}, \mathcal{C}\cup \{tmp\}\rangle$\\
% \>\>$retract(tmp)$
% \end{tabbing}

\section{Discussion}
\label{sec:impl-discussion}

The current compiler implementation is quite usable, as demonstrated
by the fact that we were able to write a web server and a \turtle{}
compiler frontend in \turtle{} (scanner, parser and abstract syntax
tree implementation).  The example programs run with reasonable speed
and memory consumption and provide enough debugging information (stack
backtraces on exceptions) to program comfortably.

User-defined data structures make the modeling of problems very
convenient, and the use of constructor functions and garbage
collection avoids a lot of programming errors due to manual memory
management.

One of the major drawbacks of the current implementation technique is
the slow compilation speed and the size of the resulting object
programs.  The \turtle{} compiler is quite fast, but it produces C
files which contain extremely large functions.  For example, a
\turtle{} parser written in \turtle{} is 900 lines long, and results
in a 12,300 line C file.  The use of a single function for a complete
\turtle{} module has the advantage that the C compiler can optimize it
very well, but the compilation speed of the C compiler is very slow.
The generation of large C functions and the necessary dispatch tables
results in large data and code sections in the resulting program.

\subsection{Benchmarks}

For illustration, we present timings for some of the \turtle{} example
programs and equivalent programs written in C.  The C compiler used
was GCC 2.95.2 and the example programs were run on an AMD Duron
processor (x86 architecture), clocked at 900MHz with 128MB main
memory.  Table~\ref{tab:benchmark} lists the results, all measured
times are in seconds.

The example programs are {\em `hanoi'}, {\em `tak'} and {\em `loops'}.
{\em `hanoi'} solves the Towers of Hanoi problem with 25 disks, {\em
  `tak'} runs the Takeuchi function 500 times and {\em `loop'} is an
empty loop which runs 500,000,000 times.  The columns named {\em
  `turtle'} show the times for the \turtle{} programs, the columns named
{\em `gcc'} show the C programs.  The columns with {\em `-O3'} show the
times for the programs optimized with the GCC command line option -O3.
For the \turtle{} programs, this means that the C output of the
compiler was compiled with -O3.  The last two columns show the ratio
between the execution times of \turtle{} and C programs, unoptimized
and optimized respectively.

The example programs run slower for a factor from 11.94--22.24 times
for the unoptimized case, and 7.7--13.3 for the optimized case.
Several reasons for that can be identified: \turtle{} programs
allocate all activation records on the heap, so they use a lot of
memory which must be repeatedly reclaimed by the garbage collector.
This is the reason why the {\em `loop'} program is much closer to its
C equivalent: it does not call any functions.  Tail-recursion
elimination does decrease performance further, as described in
section~\ref{sec:tail-recursion-elimination}.  Finally, \turtle{}
programs perform an interrupt check\footnote{The interrupt check tests
  whether asynchronous operating system signals have arrived and calls
  the appropriate signal handlers.} at each function entry and loop
header, which the C programs do not.

No timings have been made for the constraint example programs, because
the implemented constraint solvers are too slow to compare them with
any other (serious) constraint programming implementation.  The
solvers have been implemented to show that constraint imperative
programming is possible with \turtle{} and not as examples for
efficient constraint solving techniques.  Therefore they are not good
enough to be compared with other implementations.

\begin{table}
\begin{center}
\begin{tabular}{|l|r|r|r|r|r|r|}
\hline
Name & turtle & gcc & turtle -O3 & gcc -O3 & turtle/gcc & turtle -O3/gcc -O3 \\
\hline\hline
hanoi & 14.01 & 0.63 & 8.05 & 0.83 & 22.24 & 9.70 \\
\hline
tak & 12.65 & 0.65 & 8.51 & 0.64 & 19.46 & 13.30 \\
\hline
loop & 34.03 & 2.85 & 9.56 & 1.24 & 11.94 & 7.70 \\
\hline
\end{tabular}
\end{center}
\caption{Benchmark figures}
\label{tab:benchmark}
\end{table}

\subsection{Open Issues}

The implementation of \turtle{} is nearly complete with regard to the
language description developed in Chapter~\ref{cha:turtle}, but some
minor implementation issues need to be addressed to make it a
practically usable constraint imperative programming language.

The imperative part of the language is implemented completely as
documented in this work, but the constraint extensions need more work.
Translation and execution of constraint statements and the constraint
solvers work, but the garbage collector has not yet been extended to
properly collect constraints and constrainable variables when they are
no longer needed.  For serious usage of the programming system for
real applications, this would have to be implemented, of course, in
order to avoid memory space leaks.  Also the semantics concerning the
interaction of exceptions and constraints have not been implemented.
An exception occurring while a constraint statement is in effect does
not remove the constraints of the statements from the constraint
store, so constraints can remain in the store even after the body of
the corresponding constraint statement has been left.

Another problem is that the currently implemented constraint solvers
are too weak for serious constraint programming.  They can only serve
as a proof that the integration of constraints into an imperative base
language does work.

Finally, in the current implementation, the compiler is connected too
tightly to the integrated constraint solvers.  It must know which
solvers are available, which relations are defined for the individual
solvers and how the constraints in constraint statements need to be
assigned to the solvers which can handle them.  This knowledge, which
is currently implemented in the compiler, should be abstracted out
into some kind of database which describes each solver and can be
queried by the compiler when generating code for constraint
statements.

%%% Local Variables: 
%%% mode: latex
%%% TeX-master: "da.tex"
%%% End: 

%% End of turtle-impl.tex.



%% summary.tex -- Summary and evaluation.
%%
%% Copyright (C) 2003 Martin Grabmueller <mgrabmue@cs.tu-berlin.de>

\chapter{Summary and Evaluation}
\label{cha:summary}

In this work, we introduced constraint programming, examined the
research in the field of constraint integration into imperative
languages, defined the properties these combinations should have, and
finally presented the design and implementation of the higher-order
constraint imperative programming language \turtle{}.

This chapter discusses some alternatives to the design and
implementation decisions made for this work, gives a list of desirable
future works in the field of constraint imperative programming and
finally closes with a conclusion.

\section{Alternative Designs and Implementations}
\label{sec:alternatives}

The integration of backtracking into the language similar to Alma-0
would have made the interaction to constraint solving possible for
problems with multiple solutions.  On the other hand, the
implementation of arbitrary backtracking in the presence of assignment
and higher-order functions would have been much more difficult than it
is in the present solution.

Also, the design of \turtle{} differs significantly from that of
Kaleidoscope, since the \turtle{} design started with an imperative
language and added constrainable variables and constraint statements
later.  This results in a more efficient implementation, because
well-known techniques for implementing imperative languages can be
employed for programs (or program parts) which do not make use of
constraints.  In Kaleidoscope, even the simplest imperative
operations, such as assignment, are modelled in terms of constraints
and a sophisticated optimizer is necessary for discovering which
assignments can be implemented imperatively and thus more efficiently.
The problem of different treatment of object identity and object
equality does not appear in \turtle{}, since only identity is defined
on arbitrary objects.  Unfortunately, the limitation that only
constrainable variables can be determined by constraints and the fact
that only integer and real types may appear in constraint types reduce
the power of the constraint extensions considerably.  For example, it
is not possible in \turtle{} to model the relations between arbitrary
data structures, as it is in Kaleidoscope.

\section{Future Works}
\label{sec:future-works}

The present language design of \turtle{} and the state of the
implementation can not be regarded as a production-quality programming
system.  It is best viewed as a starting point for investigating the
possibilities and limitations of constraint imperative programming,
and for the more ambitious research field of multiparadigm programming
languages, combining not only imperative and constraint languages but
also the functional and relational (logic) programming styles.

\paragraph{Language enhancements.}
\index{language enhancements} Real polymorphism, combined with type
inference would reduce a lot of typing work when writing \turtle{}
programs.  The work on the standard library revealed some
short-comings, because some control structures known from other
languages would have been useful from time to time.  A more useful
loop construct like the {\bf for} loop in C, {\bf break} and {\bf
  continue} statements for exiting loops, or {\bf switch} statements
for multiway-branching were often missed.

Support for object-orientation was sometimes missed, too.  Given the
already available data structures for user-defined types and
higher-order functions, this could be implemented within the current
implementation merely by extending the syntax.

When writing code which handles user-defined data types with many
variants, dispatching on the type of variant is very tedious.  A
pattern-matching construct as known from functional languages would
ease that problem, and could also be implemented by syntax
transformations early in the compilation process.

Currently, the \turtle{} reference implementation uses exceptions to
handle run-time errors such as subscripting errors or uninitialized
variables only internally.  There is no way for the user to intercept
and handle exceptions, or to manually raise them.  This has not been
done for time reasons.

\paragraph{Constraint solvers.}
Constraint solvers for more domains and more powerful constraint
solvers should be integrated into the \turtle{} run-time system.
Finite domain solvers and linear arithmetic solvers capable of
handling cycles in the constraint graph could extend the applicability
of \turtle{} drastically.  An interesting idea is to remove all
knowledge about constraint solving from the compiler, except the
capability to construct a symbolic representation for arbitrary
expressions.  This representation could then be handed over to a Meta
Solver~\cite{hofstedt2001diss} which in turn processes the constraints
and delegates the solving process to one or more solvers which can
handle the constraints.

\paragraph{Standard library.}
To be generally useful, the standard library needs more abstract data
types (for example graphs), code for interoperatibility with existing
libraries (so-called glue code) and modules which implement generally
useful algorithms, e.g. graph algorithms.

\paragraph{Debugger.}
\index{debugger} A source-code debugger, which would not only support
\turtle{} programs, but also the visual presentation of constraints
and constraint networks, would ease debugging in constraint-imperative
languages. Meier~\cite{meier95debugging} describes a debugging
framework for constraint-logic programs, which could be adapted to the
needs of a constraint-imperative environment.

\paragraph{Optimizations.}
\index{optimizations}
%
A lot of well-known and established optimization techniques for
imperative languages could be applied to the imperative subset of
\turtle{}.  Inlining (sometimes referred to as {\em procedure
  integration}), inside of modules and across module boundaries,
constant propagation and folding, some control flow analysis and
optimizations would be beneficial.  Also, some of the optimizations
known from functional language implementations should be considered:
boxing analysis, escape analysis and code fusion.  Especially
important are optimizations which reduce heap allocation, such as
advanced closure representation optimizations, because closure and
environment allocation is an important source of inefficiency in the
current implementation.

A better garbage collector could improve the performance of \turtle{}
programs further.

\paragraph{Byte code compiler/interpreter.}
\index{byte code!compiler}
%
In addition to the C backend a byte code compiler could be
implemented.  This would speed up the development cycle.  Byte code
modules take up less storage and are easier to handle, for example for
dynamic loading.

Using the \java{} byte code as a target code comes to mind, because it
would provide interoperatibility with many applications, tools and
libraries.  On the other hand, direct generation of machine code could
improve the performance of the implementation a lot and should be
considered, too.

\paragraph{Logic Variables and Non-determinism.}
Alma-0 supports non-deterministic calculation and logic variables,
similar to logic programming languages.  The designers of Kaleidoscope
also thought about integrating them, and support for these language
features seems especially useful for solving constraint problems with
several solutions and optimization problems.

\paragraph{Further paradigm integration.}  
More programming paradigms, for example functional or logic
programming, could be integrated into the language, thus leading to a
true {\em multiparadigm programming language}
\index{multiparadigm!programming}%
in the sense of Budd et al.~\cite{buddGeneral}.

\section{Conclusion}

Starting with the literature on the topic of constraint imperative
programming, this thesis has identified the minimal properties a
language in this paradigm should have: constrainable variables,
constraint statements and a constraint solver.  Based on these key
features, the higher-order constraint imperative language \turtle{}
has been designed, starting with an imperative language with
higher-order functions and adding constraint programming extensions
later.  \turtle{} has been implemented for demonstrating the
possibilities of the language design and to explore its limitations.

The resulting language is practical, as has been verified by
implementing programs of varying sizes in \turtle{}.  The set of
example and test programs ranges from trivial ``Hello-World'' programs
to a functional web server (including built-in directory listing and
Wiki support~\cite{cunningham01wiki}) and a
hand-written parser for the \turtle{} language.  The integrated
constraint support was used in programs for solving crypto-arithmetic
puzzles and simple layout problems.  The major drawback when using
\turtle{} for practical programming is not the language design, which
is simple but flexible, but the implementation of the integrated
constraint solvers, which currently limits constraint imperative
programming to the solution of small problems.

The modeling of problems using \turtle{}'s language features,
especially algebraic data types and constraints, is very convenient.
High-level language features such as higher-order functions and
user-defined constraints lead to much cleaner programs than could be
written using traditional imperative languages.  The imperative
features of the language are available whenever a problem of
imperative nature is to be solved.  This flexibility of constraint
imperative programming is especially useful when solving problems with
varying requirements in one project and is worth considering in the
design of new programming languages and systems.

%%% Local Variables: 
%%% mode: latex
%%% TeX-master: "da.tex"
%%% End: 

%% End of summary.tex.



%% appendix.tex -- Appendix to the Diplomarbeit.
%%
%% Copyright (C) 2003 Martin Grabmueller <mgrabmue@cs.tu-berlin.de>


\begin{appendix}

%%%%%%%%%%%%%%%%%%%%%%%%%%%%%%%%%%%%%%%%%%%%%%%%%%%%%%%%%%%%%%%%%%%%%%

\newenvironment{grammar}%
  {\newcommand\produces[2]{##1 \> $\rightarrow$ \> ##2 \\}
   \newcommand\orproduces[1]{\> \> \makebox[0pt][r]{$|$ }##1 \\}
   \newcommand\res[1]{'{\bf ##1}'}
%   \newcommand\res[1]{\underline{\bf ##1}}
   \newcommand\emptyprod{{$\varepsilon$}}
   \newcommand\heading[1]{\rule{\linewidth}{1pt} \\{\bf ##1}\\[2ex]}
   \newcommand\separator{\rule{\linewidth}{1pt} \\}
   \begin{tabbing}
   \qquad\qquad\qquad\qquad\qquad \= \qquad \= \kill}
  {\end{tabbing}}

%%%%%%%%%%%%%%%%%%%%%%%%%%%%%%%%%%%%%%%%%%%%%%%%%%%%%%%%%%%%%%%%%%%%%%

\chapter{\turtle{} Grammar}
\label{cha:turtle-grammar}
\index{Turtle grammar}
\index{grammar!Turtle}
\index{syntax definition}

The lexical and grammatical structure of Turtle is described using
Extended Backus Naur Form (EBNF).  This notation is summarized below.
\index{EBNF}
\index{Extended Backus Naur Form}

\vskip0.5em
\noindent
\begin{tabular}{ll}
$A \rightarrow E$&
The non-terminal $A$ produces the form $E$.\\

$E | F$&
Alternative; either $E$ or $F$ is produced.\\

$\{ E \}$&
The form $E$ is repeated, possibly zero times.\\

$[E]$&
The form $E$ is optional, it may be omitted completely.\\

$(E)$&
For grouping, denotes $E$ itself.\\

$\varepsilon$& The empty string.\\
\end{tabular}
\vskip0.5em

\noindent
Nonterminals are printed in normal roman font, terminals appear bold
and in single quotes.

\section{Lexical Entities}
\index{lexical entities}
\index{token}

The lexical entities (also called {\em tokens}) represent the regular
part of the \turtle{} grammar.  They are recognized by the scanner
module and then processed by the parser, which recognizes the
context-free part of the syntax.  In addition to the token classes
defined below, there is a number of one- or two-character symbols.  We
do not define names for them in order to keep the grammar for the
context-free syntax readable.

\begin{grammar}
\produces{IntConst}{Digit \{Digit\}}
\produces{LongConst}{Digit \{Digit\} \res{L}}
\produces{RealConst}{Digit \{Digit\} \res{.} Digit \{Digit\} [ Exponent ]}
\produces{Exponent}{( \res{e} $|$ \res{E} )[ \res{$+$} $|$ \res{$-$} ] Digit \{ Digit \}}
\produces{StringConst}{\res{\dq} \{ StringChar \} \res{\dq}}
\produces{CharConst}{\res{'} StringChar \res{'}}
\produces{Ident}{( Letter $|$ \res{\_} ) \{ Letter $|$ Digit $|$ \res{\_} $|$ \res{?} $|$ \res{!} \}}
\produces{Letter}{\res{A} $|$ \dots $|$ \res{Z} $|$ \res{a} $|$ \dots $|$ 
\res{z}}
\produces{Digit}{\res{0} $|$ \dots $|$ \res{9}}

\produces{StringChar}{ {\em (any except \res{\,$\backslash$}, \res{\,\dq} and newline)} $|$ EscapeChar }
\produces{EscapeChar}{ \res{$\backslash\backslash$} $|$
 \res{$\backslash$\dq} $|$
 \res{$\backslash$'} $|$
 \res{$\backslash$n} $|$
 \res{$\backslash$r} $|$
 \res{$\backslash$t} $|$ 
 \res{$\backslash$b} $|$
 \res{$\backslash$v} $|$
 \res{$\backslash$a} $|$
 \res{$\backslash$f\,} }
\produces{Comment}{\res{//} \{ {\em (any except newline)} \}}
\orproduces{\res{/*} \{ {\em (any except '*/') \} \res{\,*/}}}
\end{grammar}

\index{whitespace}
Spaces, tab characters, newlines and comments serve as separators
between tokens and are otherwise ignored.

\index{string literals}
String literals are not allowed to cross line boundaries, so a literal
newline character inside a string is an error.

\index{reserved words}
Some of the tokens in the token class {\em Ident} are reserved and
have a special meaning in the \turtle{} syntax.  They may not be used
except where allowed in the context-free syntax.  The reserved words
in \turtle{} are:
%
\vskip1ex
\begin{tabular}{llllll}
and&
array&
const&
constraint&
datatype&
do\\
else&
elsif&
end&
export&
false&
foreign\\
fun&
hd&
if&
import&
in&
list\\
module&
not&
null&
of&
or&
public\\
require&
return&
sizeof&
string&
then&
tl\\
true&
type&
var&
while&&\\
\end{tabular}
\vskip1ex
\noindent
The following is a list of the non-alphanumeric symbols:
\index{non-alphanumeric symbols}
\vskip1em
\begin{tabular}{llllllllllll}
.&
,&
(&
)&
[&
]&
\{&
\}&
;&
!&
$+$&
$-$\\
*&
/&
\%&
$=$&
$<>$&
$<$&
$<=$&
$>$&
$>=$&
:&
::&
:=
\end{tabular}
\vskip1em

\section{Context-free Syntax}
\index{context-free syntax}
This is the context-free part of the \turtle{} grammar.

%\enlargethispage{-1.5cm}
\begin{grammar}
\heading{Module Definition\index{module definition syntax}}

\produces{CompilationUnit}{Module}

\produces{Module}{\res{module} QualIdent [ FormalTypeParams ] \res{;} \\\>\>ModDecls Declarations}

\produces{ModDecls}{[ ModImports ] [ ModExports ]}

\produces{ModImports}{\res{import} ImportIdent \{ \res{,} ImportIdent \} \res{;}}

\produces{ModExports}{\res{export} QualIdent \{ \res{,} QualIdent \} \res{;}}

\produces{ImportIdent}{QualIdent [ TypeParams ] [ OpenIdentList ] }
\produces{OpenIdentList}{\res{(} Ident \{\res{,} Ident\} \res{)}}

\heading{Declarations\index{declaration syntax}}

\produces{Declarations}{\{ Declaration \res{;} \}}

\produces{Declaration}{TypeDecl $|$ DatatypeDecl $|$ VarDecl $|$ ConstDecl}
\orproduces{FunDecl $|$ ConstraintDecl}

\produces{TypeDecl}{[ \res{public} ] \res{type} Ident \res{=} Type}

\produces{DatatypeDecl}{[ \res{public} ] \res{datatype} Ident [ FormalTypeParams ] \res{=} \\\>\> DatatypeVariant \{ \res{or} DatatypeVariant \}}

\produces{DatatypeVariant}{Ident [\res{(} FieldList \res{)}]}

\produces{FieldList}{Field \{\res{,} Field\}}
\produces{Field}{Ident \res{:} Type}

\produces{VarDecl}{[ \res{public} ] \res{var} VariableList}
\produces{ConstDecl}{[ \res{public} ] \res{const} VariableList}

\produces{VariableList}{Variable \{\res{,} Variable\}}
\produces{Variable}{Ident \res{:} Type [\res{:=} ConsExpression]}

\produces{FunDecl}{[ \res{public} ] \res{fun} Ident ParamList [\res{:} Type] SubrBody}

\produces{ConstraintDecl}{[ \res{public} ] \res{constraint} Ident ParamList SubrBody}
\produces{ParamList}{\res{(} [ Parameter \{\res{,} Parameter\} ] \res{)}}
\produces{Parameter}{Ident \res{:} Type}

\produces{SubrBody}{StmtList \res{end}}

\heading{Type expressions\index{type expression syntax}}

\produces{Type}{QualIdent [ TypeParams ]}
\orproduces{\res{!} Type $|$  \res{()} $|$ \res{(} Type \{\res{,} Type\} \res{)}}
\orproduces{\res{array} \res{of\,} Type
                 $|$  \res{list} \res{of\,} Type
                 $|$  \res{string}}
\orproduces{\res{fun} \res{(} [Type \{\res{,} Type\}] \res{)} [\res{:} Type]}

\heading{Statements\index{statement syntax}}

\produces{StmtList}{\{ Stmt \res{;} \}}

\produces{Stmt}{VarDecl $|$ ConstDecl $|$ FunDecl $|$ ConstraintDecl}
\orproduces{IfStmt $|$ WhileStmt $|$ ReturnStmt $|$ InStmt}
\orproduces{RequireStmt $|$ ExpressionStmt}

\produces{IfStmt}{\res{if\,} CompareExpr \res{then} StmtList \\\>\>\{ \res{elsif\,} CompareExpr \res{then} StmtList \} \\\>\>[\res{else} StmtList] \res{end}}

\produces{WhileStmt}{\res{while} CompareExpr \res{do} StmtList \res{end}}

\produces{InStmt}{\res{in} StmtList \res{end}}

\produces{ReturnStmt}{\res{return} [TupleExpr]}

\produces{ExpressionStmt}{Expression}

\heading{Constraints\index{constraint syntax}}

\produces{RequireStmt}{\res{require} ConstraintConj [ InStmt ]}
\produces{ConstraintConj}{Constraint [ \res{:} Strength ]\\\>\>\{ \res{and} Constraint [ \res{:} Strength ] \}}
\produces{Constraint}{CompareExpression}
\produces{Strength}{IntConst}

\heading{Expressions\index{expression syntax}}

\produces{Expression}{AssignExpr}

\produces{AssignExpr}{TupleExpr [ \res{:=} TupleExpr ]}

\produces{TupleExpr}{ConsExpr \{ \res{,} ConsExpr \}}

\produces{ConsExpr}{OrExpr [\res{::} ConsExpr]}

\produces{OrExpr}{AndExpr \{ \res{or} AndExpr \}}

\produces{AndExpr}{CompareExpr \{ \res{and} CompareExpr \}}

\produces{CompareExpr}{AddExpr [CompareOp AddExpr]}

\produces{AddExpr}{MulExpr \{ AddOp MulExpr \}}

\produces{MulExpr}{Factor \{ MulOp Factor \}}

\produces{Factor}{SimpleExpr $|$ UnOp Factor}

\produces{SimpleExpr}{AtomicExpr \{ ActualParameters $|$ Index \}}

\produces{ActualParameters}{\res{(} [ ConsExpr \{\res{,} ConsExpr \}] \res{)}}
\produces{Index}{\res{[} AddExpr \res{]}}

\produces{AtomicExpr}{QualIdent $|$ IntConst $|$ LongConst $|$ RealConst}
\orproduces{StringConst $|$ CharConst $|$ BoolConst}
\orproduces{ArrayExpr $|$ ListExpr $|$ \res{null}}
\orproduces{FunExpr $|$ ConstraintExpr}
\orproduces{\res{array} AddExpr \res{of\,} TupleExpr}
\orproduces{\res{list} AddExpr \res{of\,} TupleExpr}
\orproduces{\res{string} AddExpr \res{of\,} SimpleExpr}
\orproduces{\res{(} TupleExpr \res{)}}
\orproduces{\res{var} AtomicExpr $|$ \res{!} AtomicExpr}
\orproduces{\res{foreign} StringConst}

\produces{BoolConst}{\res{false} $|$ \res{true}}

\produces{FunExpr}{\res{fun} ParamList \res{:} Type SubrBody}
\produces{ConstraintExpr}{\res{constraint} ParamList SubrBody}

\produces{ArrayExpr}{\res{\{} ConsExpr \{\res{,} ConsExpr\} \res{\}}}

\produces{ListExpr}{\res{[} ConsExpr \{\res{,} ConsExpr\} \res{]}}

\heading{Operators\index{operators}}

\produces{CompareOp}{\res{$=$} $|$ \res{$<>$} $|$ \res{$<$} $|$ \res{$<=$}
$|$ \res{$>$} $|$ \res{$>=$}}

\produces{AddOp}{\res{$+$} $|$ \res{$-$}}

\produces{MulOp}{\res{$*$} $|$ \res{$/$} $|$ \res{\%}}

\produces{UnOp}{\res{$-$} $|$ \res{not} $|$ \res{hd} $|$ \res{tl} $|$ \res{sizeof\,}}

\heading{Miscellaneous}

\produces{QualIdent}{Ident \{ \res{.} Ident \}}
\produces{TypeParams}{\res{$<$} Type \{\res{,} Type\} \res{$>$}}
\produces{FormalTypeParams}{\res{$<$} Ident \{\res{,} Ident\} \res{$>$}}

\end{grammar}

\noindent
Note that some symbols have been replaced by more readable symbols in
the other chapters of this thesis (e.g., {\bf :=} has been replaced by
$\leftarrow$).  The compiler only accepts the symbols defined in this
appendix.

\chapter{Example modules}
\label{cha:example-modules}

The program listings in this appendix will serve as examples on how
complete programs written in Turtle look like.  The different examples
were chosen to demonstrate how Turtle can be used to program in
different programming paradigms.

The first is a module for list sorting, taken from the Turtle standard
library.  Because of its size, the module is split up into
programs~\ref{prog:example-module1} and~\ref{prog:example-module2}.
The module is written in pure functional style.  In lines 1 to 3 we
can see the module header, which consists of the {\bf module}
declaration, stating the name of the module and the name of the module
parameter $\alpha$.  Following, we see that the module imports a
module from the library called {\em lists}, which is instantiated with
the module parameter.  Only one name is exported, the sorting function
{\em sort}.  Line 4 contains the function header, which lists the
parameters and return types of the function.  Note that all types in
this example depend on the module parameter $\alpha$.

\index{quicksort example}
\index{examples!quicksort}
\begin{Program}
% module listsort<A>;
% import lists<A>;
% export sort;
% fun sort (a: list of A, cmp: fun (A, A): int): list of A
%   fun smaller (elem1: A): fun (A): bool
%     return fun (elem2: A): bool
%              return cmp (elem1, elem2) > 0;
%            end;
%   end;
%   fun greatereq (elem1: A): fun (A): bool
%     return fun (elem2: A): bool
%              return cmp (elem1, elem2) <= 0;
%            end;
%   end;
%   fun sort (a: list of A): list of A
%     if a = null then
%       return null;
%     else
%       if tl a = null then
%         return a;
%       else
%         if tl tl a = null then
%           if cmp (hd a, hd tl a) > 0 then
%             return hd tl a :: hd a :: null;
%           else
%             return a;
%           end;
%         else
%           var l1: list of A := lists.filter (smaller (hd a), tl a);
%           var l2: list of A := lists.filter (greatereq (hd a), tl a);
%           return lists.append (sort (l1), lists.append (hd a :: null, sort (l2)));
%         end;
%       end;
%     end;
%   end;
%   return sort (a);
% end;
\begin{ttlprog}
1\>\ttlModule{} listsort$<$$\alpha$$>$;\\
2\>\ttlImport{} lists$<$$\alpha$$>$;\\
3\>\ttlExport{} sort;\\
4\>\ttlFun{} sort (a: \ttlList{} \ttlOf{} $\alpha$, cmp: \ttlFun{} ($\alpha$, $\alpha$): int): \ttlList{} \ttlOf{} $\alpha$\\
5\>\>\ttlFun{} smaller (elem1: $\alpha$): \ttlFun{} ($\alpha$): bool\\
6\>\>\>\ttlReturn{} \ttlFun{} (elem2: $\alpha$): bool\\
7\>\>\>\>\>\>\> \ttlReturn{} cmp (elem1, elem2) $>$ 0;\\
8\>\>\>\>\>\> \ttlEnd{};\\
9\>\>\ttlEnd{};\\
10\>\>\ttlFun{} greatereq (elem1: $\alpha$): \ttlFun{} ($\alpha$): bool\\
11\>\>\>\ttlReturn{} \ttlFun{} (elem2: $\alpha$): bool\\
12\>\>\>\>\>\>\> \ttlReturn{} cmp (elem1, elem2) $\leq$ 0;\\
13\>\>\>\>\>\> \ttlEnd{};\\
14\>\>\ttlEnd{};
\end{ttlprog}
\caption{Quicksort on lists}
\label{prog:example-module1}
\end{Program}

Below, in lines 5 to 14, the internal helper functions {\em smaller}
and {\em greatereq} are defined.  These are higher-order functions
which return functions for determining whether their argument {\em
  elem2} is smaller or greater or equal to the outer argument {\em
  elem1}, respectively.

The longest part of the functions, from line 15 to line 35, consists
of the recursive sorting procedure, which implements a naive version
of Quicksort: empty or one-element lists are already sorted, and the
elements of lists of length two are swapped if they are not already in
order.  All lists longer than two elements are split into three parts:
the pivot element, which is the first element of the list for
simplicity, a list of all elements smaller than the pivot and a list
of all elements greater or equal to the pivot.  The two lists are
created using the {\em filter} function from module {\em lists}, are
sorted recursively and are finally appended to form the resulting
sorted list.

\begin{Program}
\begin{ttlprog}
15\>\>\ttlFun{} sort (a: \ttlList{} \ttlOf{} $\alpha$): \ttlList{} \ttlOf{} $\alpha$\\
16\>\>\>\ttlIf{} a = \ttlNull{} \ttlThen{}\\
17\>\>\>\>\ttlReturn{} \ttlNull{};\\
18\>\>\>\ttlElse{}\\
19\>\>\>\>\ttlIf{} \ttlTl{} a = \ttlNull{} \ttlThen{}\\
20\>\>\>\>\>\ttlReturn{} a;\\
21\>\>\>\>\ttlElse{}\\
22\>\>\>\>\>\ttlIf{} \ttlTl{} \ttlTl{} a = \ttlNull{} \ttlThen{}\\
23\>\>\>\>\>\>\ttlIf{} cmp (\ttlHd{} a, \ttlHd{} \ttlTl{} a) $>$ 0 \ttlThen{}\\
24\>\>\>\>\>\>\>\ttlReturn{} \ttlHd{} \ttlTl{} a :: \ttlHd{} a :: \ttlNull{};\\
25\>\>\>\>\>\>\ttlElse{}\\
26\>\>\>\>\>\>\>\ttlReturn{} a;\\
27\>\>\>\>\>\>\ttlEnd{};\\
28\>\>\>\>\>\ttlElse{}\\
29\>\>\>\>\>\>\ttlVar{} l1: \ttlList{} \ttlOf{} $\alpha$ $\leftarrow$ lists.filter (smaller (\ttlHd{} a), \ttlTl{} a);\\
30\>\>\>\>\>\>\ttlVar{} l2: \ttlList{} \ttlOf{} $\alpha$ $\leftarrow$ lists.filter (greatereq (\ttlHd{} a), \ttlTl{} a);\\
31\>\>\>\>\>\>\ttlReturn{} lists.append (sort (l1), lists.append (\ttlHd{} a :: \ttlNull{}, sort (l2)));\\
32\>\>\>\>\>\ttlEnd{};\\
33\>\>\>\>\ttlEnd{};\\
34\>\>\>\ttlEnd{};\\
35\>\>\ttlEnd{};\\
36\>\>\ttlReturn{} sort (a);\\
37\>\ttlEnd{};
\end{ttlprog}
\caption{Quicksort on lists (continued)}
\label{prog:example-module2}
\end{Program}

The next example will illustrate how imperative programming is done in
Turtle.  The program was adapted from the example program {\tt
  QUEENS.TIG} in~\cite{appel98moderncompiler} and solves the
$N$-queens problem, which is how to place $N$ queens on a chess board
of size $N\times N$ without any two queens attacking each other.

The first function {\em print\_board} simply prints the current board
configuration.  More interesting is the function {\em try}, which
tries to place a queen in column $c$.  If it succeeds, it places the
queen and calls itself recursively to place a queen in column $c + 1$,
and it backtracks on failure.  By trying all rows for each column, all
possibilities are eventually tried and all solutions are found.

\index{n-queens example}
\index{examples!n-queens}
% module queens;
% import io;
% var row: array of int, col: array of int;
% var diag1: array of int, diag2: array of int;
% var size: int;
% var solutions: int;
% fun print_board ()
%   var i: int, j: int;
%   i := 0;
%   while i < size do
%     j := 0;
%     while j < size do
%       if col[i] = j then
%       io.put (" o");
%       else
%       io.put (" .");
%       end;
%       j := j + 1;
%     end;
%     io.nl ();
%     i := i + 1;
%   end;
%   io.nl ();
% end;
% fun try (c: int)
%   if c = size then
%     print_board ();
%     solutions := solutions + 1;
%   else
%     var r: int;
%     r := 0;
%     while r < size do
%       if (row[r] = 0) and (diag1[r + c] = 0) and 
%       (diag2[r + (size - 1) - c] = 0) 
%       then
%       row[r] := 1;
%       diag1[r + c] := 1;
%       diag2[r + (size - 1) - c] := 1;
%       col[c] := r;
%       try (c + 1);
%       row[r] := 0;
%       diag1[r + c] := 0;
%       diag2[r + (size - 1) - c] := 0;
%       end;
%       r := r + 1;
%     end;
%   end;
% end;
% fun main (argv: list of string): int
%   size := 8;
%   solutions := 0;
%   row := array size of 0;
%   col := array size of 0;
%   diag1 := array size * 2 - 1 of 0;
%   diag2 := array size * 2 - 1 of 0;
%   try (0);
%   io.put (solutions);
%   io.put (" solutions found.\n");
%   return 0;
% end;
\begin{Program}
\begin{ttlprog}
1\>\ttlModule{} queens;\\
2\>\ttlImport{} io;\\
3\>\ttlVar{} row: \ttlArray{} \ttlOf{} int, col: \ttlArray{} \ttlOf{} int;\\
4\>\ttlVar{} diag1: \ttlArray{} \ttlOf{} int, diag2: \ttlArray{} \ttlOf{} int;\\
5\>\ttlVar{} size: int;\\
6\>\ttlVar{} solutions: int;\\
7\>\ttlFun{} print\_board ()\\
8\>\>\ttlVar{} i: int, j: int;\\
9\>\>i $\leftarrow$ 0;\\
10\>\>\ttlWhile{} i $<$ size \ttlDo{}\\
11\>\>\>j $\leftarrow$ 0;\\
12\>\>\>\ttlWhile{} j $<$ size \ttlDo{}\\
13\>\>\>\>\ttlIf{} col[i] = j \ttlThen{}\\
14\>\>\>\>\>io.put ("o");\\
15\>\>\>\>\ttlElse{}\\
16\>\>\>\>\>io.put (".");\\
17\>\>\>\>\ttlEnd{};\\
18\>\>\>\>j $\leftarrow$ j $+$ 1;\\
19\>\>\>\ttlEnd{};\\
20\>\>\>io.nl ();\\
21\>\>\>i $\leftarrow$ i $+$ 1;\\
22\>\>\ttlEnd{};\\
23\>\>io.nl ();\\
24\>\ttlEnd{};
\end{ttlprog}
\caption{N-queens problem}
\label{prog:example-queens1}
\end{Program}

\begin{Program}
\begin{ttlprog}
25\>\ttlFun{} try (c: int)\\
26\>\>\ttlIf{} c = size \ttlThen{}\\
27\>\>\>print\_board ();\\
28\>\>\>solutions $\leftarrow$ solutions $+$ 1;\\
29\>\>\ttlElse{}\\
30\>\>\>\ttlVar{} r: int;\\
31\>\>\>r $\leftarrow$ 0;\\
32\>\>\>\ttlWhile{} r $<$ size \ttlDo{}\\
33\>\>\>\>\ttlIf{} (row[r] = 0) \ttlAnd{} (diag1[r $+$ c] = 0) \ttlAnd{} \\
34\>\>\>\>\>(diag2[r $+$ (size $-$ 1) $-$ c] = 0) \\
35\>\>\>\>\ttlThen{}\\
36\>\>\>\>\>row[r] $\leftarrow$ 1;\\
37\>\>\>\>\>diag1[r $+$ c] $\leftarrow$ 1;\\
38\>\>\>\>\>diag2[r $+$ (size $-$ 1) $-$ c] $\leftarrow$ 1;\\
39\>\>\>\>\>col[c] $\leftarrow$ r;\\
40\>\>\>\>\>try (c $+$ 1);\\
41\>\>\>\>\>row[r] $\leftarrow$ 0;\\
42\>\>\>\>\>diag1[r $+$ c] $\leftarrow$ 0;\\
43\>\>\>\>\>diag2[r $+$ (size $-$ 1) $-$ c] $\leftarrow$ 0;\\
44\>\>\>\>\ttlEnd{};\\
45\>\>\>\>r $\leftarrow$ r $+$ 1;\\
46\>\>\>\ttlEnd{};\\
47\>\>\ttlEnd{};\\
48\>\ttlEnd{};\\
49\>\ttlFun{} main (argv: \ttlList{} \ttlOf{} \ttlString{}): int\\
50\>\>size $\leftarrow$ 8;\\
51\>\>solutions $\leftarrow$ 0;\\
52\>\>row $\leftarrow$ \ttlArray{} size \ttlOf{} 0;\\
53\>\>col $\leftarrow$ \ttlArray{} size \ttlOf{} 0;\\
54\>\>diag1 $\leftarrow$ \ttlArray{} size * 2 $-$ 1 \ttlOf{} 0;\\
55\>\>diag2 $\leftarrow$ \ttlArray{} size * 2 $-$ 1 \ttlOf{} 0;\\
56\>\>try (0);\\
57\>\>io.put (solutions);\\
58\>\>io.put ("solutions found.$\backslash$n");\\
59\>\>\ttlReturn{} 0;\\
60\>\ttlEnd{};\\
\end{ttlprog}
\caption{N-queens problem (continued)}
\label{prog:example-queens2}
\end{Program}

\newpage
The last example program shows how constraint imperative programs are
written in \turtle{}.  Program~\ref{prog:send-more-money} solves the
famous crypto-arithmetic puzzle:
\vskip1ex
\begin{tabular}{rrrrr}
&S&E&N&D\\
+&M&O&R&E\\
\hline
M&O&N&E&Y
\end{tabular}
\vskip1ex The task is to assign a number between 0 and 9 to each
letter, so that the sum of the first two lines equals the last line,
under the constraint that all letters must get assigned different
numbers.

The example program first defines the two user-defined constraints
{\em all\_different} and {\em domain}.  The former constrains a list
of constrainable variables to be pairwise distinct and the latter
constrains a variable to lie within a lower and an upper bound.  The
function {\em main} declares and initializes variables for each letter
and models the puzzle using constraints.  In the constraint statement
body, the solution is finally displayed.

\index{send-more-money}
\index{examples!send-more-money}
\begin{Program}
\begin{ttlprog}
1\>\ttlModule{} sendmory2;\\
2\>\ttlImport{} io;\\
3\>\ttlConstraint{} all\_different (l: \ttlList{} \ttlOf{} {\bf!}int)\\
4\>\>\ttlWhile{} \ttlTl{} l $\neq$ \ttlNull{} \ttlDo{}\\
5\>\>\>\ttlVar{} ll: \ttlList{} \ttlOf{} {\bf!}int $\leftarrow$ \ttlTl{} l;\\
6\>\>\>\ttlWhile{} ll $\neq$ \ttlNull{} \ttlDo{}\\
7\>\>\>\>\ttlRequire{} \ttlHd{} l $\neq$ \ttlHd{} ll;\\
8\>\>\>\>ll $\leftarrow$ \ttlTl{} ll;\\
9\>\>\>\ttlEnd{};\\
10\>\>\>l $\leftarrow$ \ttlTl{} l;\\
11\>\>\ttlEnd{};\\
12\>\ttlEnd{};\\
13\>\ttlConstraint{} domain (v: {\bf!}int, min: int, max: int)\\
14\>\>\ttlRequire{} v $\geq$ min \ttlAnd{} v $\leq$ max;\\
15\>\ttlEnd{};\\
16\>\ttlFun{} main(args: \ttlList{} \ttlOf{} \ttlString{}): int\\
17\>\>\ttlVar{} s: {\bf!}int $\leftarrow$ \ttlVar{} 0;\\
18\>\>\ttlVar{} e: {\bf!}int $\leftarrow$ \ttlVar{} 0;\\
19\>\>\ttlVar{} n: {\bf!}int $\leftarrow$ \ttlVar{} 0;\\
20\>\>\ttlVar{} d: {\bf!}int $\leftarrow$ \ttlVar{} 0;\\
21\>\>\ttlVar{} m: {\bf!}int $\leftarrow$ \ttlVar{} 0;\\
22\>\>\ttlVar{} o: {\bf!}int $\leftarrow$ \ttlVar{} 0;\\
23\>\>\ttlVar{} r: {\bf!}int $\leftarrow$ \ttlVar{} 0;\\
24\>\>\ttlVar{} y: {\bf!}int $\leftarrow$ \ttlVar{} 0;\\
25\>\>\ttlRequire{} domain (s, 0, 9) \ttlAnd{} domain (e, 0, 9) \ttlAnd{} domain (n, 0, 9) \ttlAnd{}\\
26\>\>\>domain (d, 0, 9) \ttlAnd{} domain (m, 1, 9) \ttlAnd{} domain (o, 0, 9) \ttlAnd{}\\
27\>\>\>domain (r, 0, 9) \ttlAnd{} domain (y, 0, 9) \ttlAnd{}\\
28\>\>\>all\_different ([s, e, n, d, m, o, r, y]) \ttlAnd{}\\
29\>\>\>(s * 1000 $+$ e * 100 $+$ n * 10 $+$ d) $+$ (m * 1000 $+$ o * 100 $+$ r * 10 $+$ e) =\\
30\>\>\>(m * 10000 $+$ o * 1000 $+$ n * 100 $+$ e * 10 $+$ y)\\
31\>\>\ttlIn{}\\
32\>\>\>io.put ("s = "); io.put ({\bf!}s); io.nl ();\\
33\>\>\>io.put ("e = "); io.put ({\bf!}e); io.nl ();\\
34\>\>\>io.put ("n = "); io.put ({\bf!}n); io.nl ();\\
35\>\>\>io.put ("d = "); io.put ({\bf!}d); io.nl ();\\
36\>\>\>io.put ("m = "); io.put ({\bf!}m); io.nl ();\\
37\>\>\>io.put ("o = "); io.put ({\bf!}o); io.nl ();\\
38\>\>\>io.put ("r = "); io.put ({\bf!}r); io.nl ();\\
39\>\>\>io.put ("y = "); io.put ({\bf!}y); io.nl ();\\
40\>\>\ttlEnd{};\\
41\>\>\ttlReturn{} 0;\\
42\>\ttlEnd{};
\end{ttlprog}
\caption{Send-more-money example}
\label{prog:send-more-money}
\end{Program}

\chapter{The Standard Library}
\label{cha:turtle-library}

\newenvironment{modules}%
  {\newcommand\descr[2]{\index{##1@{\tt ##1} (Module)}%
\index{modules!##1@{\tt ##1}}%
{\tt ##1} \> ##2 \\}
   \begin{tabbing}
   \qquad\qquad\qquad\qquad\qquad \= \kill}
  {\end{tabbing}}

Besides the language definition and implementation, a standard library
is extremely important for the usefulness of a programming language.
The reference implementation of \turtle{} includes all modules
mentioned in this appendix.

The modules of the standard library are grouped into functional
categories and a short description is given for each.  The complete
documentation for the modules and their exported functions, data types
and variables is contained in the reference manual which comes with
the distribution.

\index{modules}
\index{standard library}

\subsection*{General}
\index{general modules}

\begin{modules}
\descr{math}{Mathematical constants and trigonometric functions}
\descr{random}{Random number generator}
\descr{compare}{Comparison functions for the basic data types}
\descr{compose}{Function composition}
\descr{identity}{Identity function}
\descr{option}{The {\tt option} data type, as known from Standard ML}
\descr{trees}{Binary trees}
\descr{bstrees}{Binary search trees}
\descr{cmdline}{Command line parsing and option handling}
\descr{hash}{Calculating hash values for basic data types}
\descr{hashtab}{Hash tables}
\descr{binary}{Byte-array handling}
\descr{exceptions}{Exception raising and handling}
\descr{filenames}{Filename manipulation}
\end{modules}


\subsection*{Input and Output}
\index{input and output modules}

\begin{modules}
\descr{io}{Basic input/output facilities for basic data types}
\end{modules}

\subsection*{Data Type Related}
\index{data type related modules}

\begin{modules}
\descr{ints}{Integer constants and functions}
\descr{longs}{Long integer functions}
\descr{reals}{Real constants and functions}
\descr{bools}{Boolean value functions}
\descr{chars}{Character handling}
\descr{strings}{String processing and conversion functions}
\descr{union}{Union data type of the basic data types}
\descr{strformat}{String formatting}
\end{modules}

\subsection*{List Utilities}
\index{list utility modules}

\begin{modules}
\descr{lists}{General list manipulation functions}
\descr{listmap}{Mapping functions over lists}
\descr{listsort}{List sorting}
\descr{listsearch}{Linear search on lists}
\descr{listfold}{Folding functions over lists}
\descr{listreduce}{Reducing lists with initial value}
\descr{listzip}{Combining two lists into one}
\descr{listindex}{List functions operating with indices}
\end{modules}


\subsection*{Array Utilities}
\index{array utility modules}

\begin{modules}
\descr{arrays}{General array manipulation functions}
\descr{arraymap}{Mapping a function over an array}
\descr{arraysort}{Array sorting}
\descr{arraysearch}{Linear and binary search on arrays}
\end{modules}

\subsection*{Tuple Utilities}
\index{tuple utility modules}

\begin{modules}
\descr{pairs}{Handling 2-tuples}
\descr{triples}{Handling 3-tuples}
\end{modules}


\subsection*{Low-Level}
\index{low-level modules}

\begin{modules}
\descr{core}{Low-level functions, like real number formatting}
\end{modules}


\subsection*{System-dependant Modules}
\index{system-dependant modules}

\begin{modules}
\descr{sys.files}{File handling}
\descr{sys.dirs}{Directory handling}
\descr{sys.net}{Network programming}
\descr{sys.times}{Time and date functions}
\descr{sys.users}{Accessing the user data base}
\descr{sys.procs}{Process handling}
\descr{sys.errno}{Operating system error codes}
\descr{sys.signal}{Unix signal handling}
\end{modules}


\subsection*{Internal Modules}
\index{internal modules}

\begin{modules}
\descr{internal.version}{Version numbers}
\descr{internal.random}{Random number generation}
\descr{internal.stats}{Run-time system statistics}
\descr{internal.gc}{Garbage collector interface}
\descr{internal.ex}{Low-level exception raising and handling}
\descr{internal.timeout}{Functions to be run in the background}
\descr{internal.limits}{System limit constants}
\end{modules}

\chapter{The \turtle{} System}
\label{cha:turtle-compiler}

During the work on this thesis, a compiler for the \turtle{} language
was developed which supports all of the language features described in
Chapter~\ref{cha:turtle}.\footnote{Note the limitations of the
  implementation as described in section~\ref{sec:impl-discussion}.}
The compiler translates \turtle{} source code into C source code which
in turn is compiled to executable machine code by a C compiler.

The compiler consists of a command line program which runs on
GNU/Linux and Solaris, and should be portable to other Unices and
Windows without much work.

The source code is available as a gzip'ed tar archive from the
\index{Turtle home page}
\turtle{} home page at {\tt
  http://user.cs.tu-berlin.de/\~{}mgrabmue/turtle/}.

\index{installation}
After downloading the file {\tt turtle-1.0.0.tar.gz}, the archive is
unpacked with the command

{\tt
zcat turtle-1.0.0.tar.gz | tar xf -
}

\noindent
or, if you use GNU tar:

{\tt
tar xzf turtle-1.0.0.tar.gz
}

\noindent
Then the source code must be configured and built with the commands

{\tt ./configure}

{\tt make}

\noindent
After completion, the compiler, the run-time system and the standard
library have been compiled and can be installed (after possibly
switching to administrator privileges) with the command

{\tt make install}

\noindent
Before installation, if you want to make sure everything worked fine,
you can compile and run a few test programs and examples by executing

{\tt make check}

\noindent
The installation requires a C compiler for compiling the compiler and
the run-time system, and for compiling the output of the \turtle{}
compiler.  After the installation, a C compiler is also necessary for
compiling \turtle{} code.  The \turtle{} system was tested with GCC
2.95.2.

Please read the files {\tt README} and {\tt INSTALL} after unpacking
the distribution for more detailed instructions on how to compile and
install the system.  The online documentation for the compiler, the
run-time system and the standard library is available in GNU info
format in the subdirectory {\tt doc}.

The subdirectory {\tt examples} contains some example programs,
ranging from {\tt minimal.t}, which is the minimal executable
\turtle{} program, to {\tt webserver.t} and utility modules, which
implement a simple, working web server.  Some constraint
im\-per\-ative programs for solving crypto-arithmetic puzzles and
simple mathematical problems are also included.

\end{appendix}

%%% Local Variables: 
%%% mode: latex
%%% TeX-master: "da.tex"
%%% End: 

%% End of appendix.tex.



%\nocite{*}
\bibliographystyle{alpha}
\bibliography{cip-literature}
\label{cha:bibliography}

\cleardoublepage
\printindex

\end{document}

%%% Local Variables: 
%%% mode: latex
%%% TeX-master: t
%%% End: 

%% End of da.tex.
