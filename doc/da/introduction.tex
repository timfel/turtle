%% introduction.tex -- Introduction to the Diplomarbeit.
%%
%% Copyright (C) 2003 Martin Grabmueller <mgrabmue@cs.tu-berlin.de>

\chapter{Introduction}
\label{cha:introduction}

\section{Motivation}
\label{sec:motivation}

Problems are solved by computer programs by first developing
simplified models of the real world which capture the objects and
relations of interest.  These models must be represented inside the
computer in a form suitable for machine processing.  The computer can
then manipulate the models' representations to create one or more
solutions and these results must be interpreted in the context of the
original problem.  The task of the programmer consists of the
necessary modelling and the translation of the model into a language
the computer can work with.  Programming languages should help the
programmer with this task, so that the data models and algorithms for
finding solutions can be expressed correctly and conveniently, and
that the computer can find them as fast as possible.

The desire to have both correct and efficient computer programs often
results in two conflicting goals.  Often the first wish is addressed
by using high-level programming languages, which offer a clear syntax
and well-defined semantics, so that the correctness of a program can
be proved using mathematical properties of the semantics.
Additionally, modern high-level languages provide features which are
convenient for modelling and handling data structures, such as
algebraic data types\index{algebraic data type}, pattern
matching\index{pattern matching}, higher-order
functions\index{higher-order function}, unification\index{unification}
and built-in search\index{built-in seach}\index{search} facilities.
Unfortunately, today's implementations of these high-level languages
are often still slower than their lower-level counterparts, the
imperative programming languages.  Many programmers still neglect
higher-level languages as being too slow for their purposes, despite
the incredible advances in compiler technology\footnote{Of course, the
  advances in hardware are important as well, because they change the
  sets of programs which are considered ``slow'' or ``fast''.}, which
let optimizing compilers produce machine code which performs nearly as
efficient as hand-written imperative code. Even though the difference
in performance between low- and high-level languages is insignificant
for most programming problems, many programmers are still using
imperative languages.  There are several reasons for that:

\begin{description}
\item[Performance.\index{performance}] For some tasks, higher-level
  language implementations still do not have the necessary
  performance, e.g.~for real-time systems\index{real-time system}
  where guaranteed response times and low memory usage are often the
  most important requirements.  Higher-level languages normally need a
  large run-time system\index{run-time system} and garbage
  collection\index{garbage collection}, which are hard to be made
  suitable for real-time systems.
  
\item[Development Environments.\index{development environment}] There
  are very good development environments for imperative languages,
  which are convenient to use and generate very efficient programs,
  for instance the integrated development
  environments\index{integrated development environment} from
  companies like Microsoft\index{Microsoft}, Borland\index{Borland},
  IBM\index{IBM} or Sun\index{Sun} for programming
  \cplusplus{}\index{C++} or \java{}.\index{Java} These environments
  contain very good debuggers\index{debugger}, and the elimination of
  program errors is often thought of as the important last phase
  before delivering software, and the need to avoid such errors early
  in the development process is not considered important.
  
\item[Education.]  Imperative programming languages are often the
  first languages which are taught at school, in university and in
  programming courses.  Since most companies look for imperative
  programmers and not for functional or logic programmers, there is
  some pressure on the educators to teach these languages, even if
  they know about the advantages of higher-level languages.

\item[Availability.] Higher-level language implementations are not as
  widely available as imperative ones, at least from commercial
  vendors.  A lot of software companies do not use them at all.
  
\item[Legacy\index{legacy} Software.]  Most of old but still important
  software is written in imperative languages like
  \cobol{}\index{Cobol}, \fortran{}\index{Fortran} or
  \cee{}\index{C}/\cplusplus{}\index{C++} and still needs to be
  maintained and extended.
  
\item[Tools and Libraries.] For older languages, there exist many well
  tested and efficient tools and libraries which help in programming.
  For newer systems, less libraries for addressing every-day problems
  in industry have been written---not because it is not possible, but
  simply because there has not been enough time and personnel to do
  it.
  
\item[Literature.]  A lot of literature is dedicated to imperative
  languages.  Again, this is mostly caused by the fact that there is
  not yet such a wide audience for books on higher-level languages as
  there is for imperative languages.
\end{description}

\noindent
Despite the wide use of imperative languages, it is consensus that
high-level languages decrease the number of errors in software systems
and lead to faster development.

So what is necessary to make programmers use high-level languages?

Most of the problems mentioned above have already been addressed by
research in these fields and the development of hardware and tools.
Compilers for high-level languages produce acceptable code.  Memory
needs of higher-level languages are normally larger than for
imperative languages, but today every personal computer is equipped
with hundreds of megabytes of memory.  The mainstream and the research
results are approaching the same requirements of hardware as
functional and logic language compilers become better and imperative
language implementations become slower, due to widespread use of byte
codes%
\index{byte code} and virtual machines%
\index{virtual machine} to obtain portability, which seems to be more
important nowadays than pure number-crunching performance.  At the
same time, as high-level language implementations grow more mature,
they gain more interest in the programmer community, which leads to
more tools and libraries (often freely available) and encourages
publishers to publish books on this topic.  That is why the author
believes that most of the problems listed above will be simply solved
by the time which is needed to let ideas spread from academia to the
mainstream, as could be witnessed with the success of object-oriented%
\index{object-oriented programming} programming in recent years, which
took more than twenty years until it was widely used.

Unfortunately, one problem has not been mentioned yet:

\begin{description}
\item[Habits.]  Most programmers are used to their favourite
  programming language, which most probably is an imperative one.  And
  habits do not change easily.
\end{description}

\noindent
The idea for solving this problem is to combine technical measures
with ``psychological'' ones, that is, to design a programming language
which lets the programmer get used to declarative programming
incrementally.  That is why a language should have both low-level
concepts with predictable and guaranteed time and space requirements
as well as high-level concepts such as higher-order functions%
\index{higher-order function} for programming more abstractly. Then
the programmer can choose whatever language feature is suitable for a
given task.

Such a strategy in language design leads to multiparadigm programming
languages%
\index{multiparadigm!programming language}%
\index{multiparadigm}, as described by Budd%
\index{Budd}, Justice%
\index{Justice} and Pandey%
\index{Pandey} in~\cite{buddGeneral}.  The idea is to offer the best
in existing programming paradigms and let the user decide which
language features to use when.  The language Leda%
\index{Leda}, for example, designed by Budd%
\index{Budd} and described in~\cite{budd94leda, budd95mppil}, combines
imperative and relational programming in one language.

The language \turtle{}%
\index{Turtle} developed in this thesis merges imperative and
constraint programming\footnote{Actually, functional language elements
  are also added, as far as they fit into the constraint imperative
  framework.}  into one language, resulting in a powerful language
with a wide area of applicability.

Constraints are relations in the mathematical sense, which are used in
constraint-based programming languages for specifying the relations
which hold between different entities of interest.  Since objects and
their relations change over time (for example in interactive
applications or simulations), it is necessary to check continuously
whether these state transitions violate the specification of the
model.  The specification must then be restored by modifying the
objects or their relations.  Additionally, the set of relations must
be capable of changing over time, as well as the set of objects.

This work investigates the integration of two very different
programming paradigms.  On the one hand declarative programming with
constraints, the specification of the properties of objects and their
relations, and on the other hand imperative programming, which deals
explicitly with statements and the change of the program state over
time.  So we have a programming style in which we tell the machine
{\em what} to calculate, and a style in which we tell {\em how} to do
it.  The main goal of this work is to find out how these different
styles can be integrated smoothly, so that we can obtain ``the optimum
of both worlds.''


\section{Background}
\label{sec:background}

This work was motivated mainly by two existing language designs which
integrate imperative and declarative programming.  The first is the
language Alma-0%
\index{Alma-0}, developed by Apt%
\index{Apt} et al.~\cite{apt97search, apt98alma, apt98almaproject},
which is an intermediate result of the Alma%
\index{Alma} project.  This project aims to lead to the design of a
constraint imperative language.  While Alma-0 is on its way to a
constraint imperative language, its designers decided to proceed
step by step by starting with an imperative language, then adding
declarative language elements such as logical variables, the notion of
success/failure and backtracking to their language and then, at later
stages, to integrate constraints.  Alma-0 is based on Modula-2, both
in syntax and semantics.

Borning%
\index{Borning} et
al.~\cite{lopez94kaleidoscope,freemanbenson90design} developed a
family of constraint imperative languages called Kaleidoscope%
\index{Kaleidoscope}.  The first version, Kaleidoscope '90 was
designed to be a constraint imperative language from the beginning.
The language is object-oriented and allows to place constraints on
primitive data types as well as user-defined objects. Later versions
of the language changed in syntax and semantics to solve various
problems of the original design.

Both languages and other approaches are described in
Chapter~\ref{cha:constraint-imperative-programming}.

As already mentioned in section~\ref{sec:motivation}, Budd's works on
multiparadigm languages also influenced this thesis. General work on
constraint solving and satisfaction, constraint hierarchies,
constraint logic and constraint functional programming languages also
had an impact on the statements following.


\section{Goals}
\label{sec:goals}

First, the existing work in the field of constraint imperative
programming will be summarized, so that the following parts can build
on previous results of research in this area.

The next goal for this work is the design and implementation of a
programming language which is suitable for both imperative and
constraint programming.  The design is guided by the investigation of
previous approaches to this problem and related fields, such as
general programming language design, constraint logic programming and
other work on integrating programming language paradigms.  Besides the
design and the specification of the semantics of our integrated
approach, the implementation problems and possible solutions will be
addressed as well.

The final goal is to investigate how well the resulting language meets
the requirements for such a language, such as ease of programming,
easy adaptability for imperative programmers, reduced sources of
programming errors and reasonable efficiency.  This goal is addressed
by using the resulting language in implementing and examining test
programs.


\section{Outline}
\label{sec:outline}

This thesis is organized as follows: after this Introduction,
Chapter~\ref{cha:constraint-programming} gives a general view on
constraint programming.  We give definitions of the notations which
will be used in subsequent chapters, then we present some variations
of constraint problems and their solution techniques.  Finally the
existing approaches to integrate constraint programming with other
(mostly declarative) languages are summarized.

A special combination of constraints and traditional
programming---constraint im\-pe\-ra\-tive programming---is described
in Chapter~\ref{cha:constraint-imperative-programming}. Here the
origins as well as existing approaches and implementations are
presented and we will review the literature on constraint
im\-pe\-ra\-tive programming.  In this chapter, goals for a successful
merge of low- and high-level languages are defined, and possible
problems are identified.

Chapter~\ref{cha:turtle} describes the constraint imperative
programming language \turtle{}, which was designed to address the
problems arising in the integration of these two paradigms.  After
presenting the syntax we specify the semantics of constraint
imperative programs in \turtle{}.  Following the presentation of the
language design, the implementation of the compiler and the run-time
system will be described in Chapter~\ref{cha:turtle-impl} together
with a discussion of efficiency and effectiveness of constraint
imperative programs written in \turtle{}.

In Chapter~\ref{cha:summary} the thesis is summarized and compared to
previous work on the design and implementation of constraint languages
and constraint imperative languages.  Also, ideas for future
directions in the work on constraint imperative and multiparadigm
programming in general, and especially with regard to \turtle{} and
its implementations are outlined.

The Appendix contains a formal description of the grammar of \turtle,
a description of the modules available in the standard library and the
definitions of three \turtle{} modules for illustrating the syntax of
\turtle{} programs.  The Appendix also contains information about the
availability of the \turtle{} system which was developed as a part of
this thesis.


\section{Notation}

In the following chapters, some terms in the text and the example
programs will be emphasized by typesetting them differently.
%
\begin{itemize}
\item Reserved words are written in boldface, for example {\bf if},
  {\bf string} and {\bf module}.

\item Variables will be written in italics, e.g. {\em x} and {\em
    counter}.

\item Function and module names will be written in italics in the
  text, but in normal font in the example programs.
\end{itemize}
%
Since some type names are reserved names and some are not, the names
of data types will sometimes written in boldface and sometimes in
italics, e.g. {\bf string} and {\em bool}.

%%% Local Variables: 
%%% mode: latex
%%% TeX-master: "da.tex"
%%% End: 

%% End of introduction.tex.
