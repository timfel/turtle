%% summary.tex -- Summary and evaluation.
%%
%% Copyright (C) 2003 Martin Grabmueller <mgrabmue@cs.tu-berlin.de>

\chapter{Summary and Evaluation}
\label{cha:summary}

In this work, we introduced constraint programming, examined the
research in the field of constraint integration into imperative
languages, defined the properties these combinations should have, and
finally presented the design and implementation of the higher-order
constraint imperative programming language \turtle{}.

This chapter discusses some alternatives to the design and
implementation decisions made for this work, gives a list of desirable
future works in the field of constraint imperative programming and
finally closes with a conclusion.

\section{Alternative Designs and Implementations}
\label{sec:alternatives}

The integration of backtracking into the language similar to Alma-0
would have made the interaction to constraint solving possible for
problems with multiple solutions.  On the other hand, the
implementation of arbitrary backtracking in the presence of assignment
and higher-order functions would have been much more difficult than it
is in the present solution.

Also, the design of \turtle{} differs significantly from that of
Kaleidoscope, since the \turtle{} design started with an imperative
language and added constrainable variables and constraint statements
later.  This results in a more efficient implementation, because
well-known techniques for implementing imperative languages can be
employed for programs (or program parts) which do not make use of
constraints.  In Kaleidoscope, even the simplest imperative
operations, such as assignment, are modelled in terms of constraints
and a sophisticated optimizer is necessary for discovering which
assignments can be implemented imperatively and thus more efficiently.
The problem of different treatment of object identity and object
equality does not appear in \turtle{}, since only identity is defined
on arbitrary objects.  Unfortunately, the limitation that only
constrainable variables can be determined by constraints and the fact
that only integer and real types may appear in constraint types reduce
the power of the constraint extensions considerably.  For example, it
is not possible in \turtle{} to model the relations between arbitrary
data structures, as it is in Kaleidoscope.

\section{Future Works}
\label{sec:future-works}

The present language design of \turtle{} and the state of the
implementation can not be regarded as a production-quality programming
system.  It is best viewed as a starting point for investigating the
possibilities and limitations of constraint imperative programming,
and for the more ambitious research field of multiparadigm programming
languages, combining not only imperative and constraint languages but
also the functional and relational (logic) programming styles.

\paragraph{Language enhancements.}
\index{language enhancements} Real polymorphism, combined with type
inference would reduce a lot of typing work when writing \turtle{}
programs.  The work on the standard library revealed some
short-comings, because some control structures known from other
languages would have been useful from time to time.  A more useful
loop construct like the {\bf for} loop in C, {\bf break} and {\bf
  continue} statements for exiting loops, or {\bf switch} statements
for multiway-branching were often missed.

Support for object-orientation was sometimes missed, too.  Given the
already available data structures for user-defined types and
higher-order functions, this could be implemented within the current
implementation merely by extending the syntax.

When writing code which handles user-defined data types with many
variants, dispatching on the type of variant is very tedious.  A
pattern-matching construct as known from functional languages would
ease that problem, and could also be implemented by syntax
transformations early in the compilation process.

Currently, the \turtle{} reference implementation uses exceptions to
handle run-time errors such as subscripting errors or uninitialized
variables only internally.  There is no way for the user to intercept
and handle exceptions, or to manually raise them.  This has not been
done for time reasons.

\paragraph{Constraint solvers.}
Constraint solvers for more domains and more powerful constraint
solvers should be integrated into the \turtle{} run-time system.
Finite domain solvers and linear arithmetic solvers capable of
handling cycles in the constraint graph could extend the applicability
of \turtle{} drastically.  An interesting idea is to remove all
knowledge about constraint solving from the compiler, except the
capability to construct a symbolic representation for arbitrary
expressions.  This representation could then be handed over to a Meta
Solver~\cite{hofstedt2001diss} which in turn processes the constraints
and delegates the solving process to one or more solvers which can
handle the constraints.

\paragraph{Standard library.}
To be generally useful, the standard library needs more abstract data
types (for example graphs), code for interoperatibility with existing
libraries (so-called glue code) and modules which implement generally
useful algorithms, e.g. graph algorithms.

\paragraph{Debugger.}
\index{debugger} A source-code debugger, which would not only support
\turtle{} programs, but also the visual presentation of constraints
and constraint networks, would ease debugging in constraint-imperative
languages. Meier~\cite{meier95debugging} describes a debugging
framework for constraint-logic programs, which could be adapted to the
needs of a constraint-imperative environment.

\paragraph{Optimizations.}
\index{optimizations}
%
A lot of well-known and established optimization techniques for
imperative languages could be applied to the imperative subset of
\turtle{}.  Inlining (sometimes referred to as {\em procedure
  integration}), inside of modules and across module boundaries,
constant propagation and folding, some control flow analysis and
optimizations would be beneficial.  Also, some of the optimizations
known from functional language implementations should be considered:
boxing analysis, escape analysis and code fusion.  Especially
important are optimizations which reduce heap allocation, such as
advanced closure representation optimizations, because closure and
environment allocation is an important source of inefficiency in the
current implementation.

A better garbage collector could improve the performance of \turtle{}
programs further.

\paragraph{Byte code compiler/interpreter.}
\index{byte code!compiler}
%
In addition to the C backend a byte code compiler could be
implemented.  This would speed up the development cycle.  Byte code
modules take up less storage and are easier to handle, for example for
dynamic loading.

Using the \java{} byte code as a target code comes to mind, because it
would provide interoperatibility with many applications, tools and
libraries.  On the other hand, direct generation of machine code could
improve the performance of the implementation a lot and should be
considered, too.

\paragraph{Logic Variables and Non-determinism.}
Alma-0 supports non-deterministic calculation and logic variables,
similar to logic programming languages.  The designers of Kaleidoscope
also thought about integrating them, and support for these language
features seems especially useful for solving constraint problems with
several solutions and optimization problems.

\paragraph{Further paradigm integration.}  
More programming paradigms, for example functional or logic
programming, could be integrated into the language, thus leading to a
true {\em multiparadigm programming language}
\index{multiparadigm!programming}%
in the sense of Budd et al.~\cite{buddGeneral}.

\section{Conclusion}

Starting with the literature on the topic of constraint imperative
programming, this thesis has identified the minimal properties a
language in this paradigm should have: constrainable variables,
constraint statements and a constraint solver.  Based on these key
features, the higher-order constraint imperative language \turtle{}
has been designed, starting with an imperative language with
higher-order functions and adding constraint programming extensions
later.  \turtle{} has been implemented for demonstrating the
possibilities of the language design and to explore its limitations.

The resulting language is practical, as has been verified by
implementing programs of varying sizes in \turtle{}.  The set of
example and test programs ranges from trivial ``Hello-World'' programs
to a functional web server (including built-in directory listing and
Wiki support~\cite{cunningham01wiki}) and a
hand-written parser for the \turtle{} language.  The integrated
constraint support was used in programs for solving crypto-arithmetic
puzzles and simple layout problems.  The major drawback when using
\turtle{} for practical programming is not the language design, which
is simple but flexible, but the implementation of the integrated
constraint solvers, which currently limits constraint imperative
programming to the solution of small problems.

The modeling of problems using \turtle{}'s language features,
especially algebraic data types and constraints, is very convenient.
High-level language features such as higher-order functions and
user-defined constraints lead to much cleaner programs than could be
written using traditional imperative languages.  The imperative
features of the language are available whenever a problem of
imperative nature is to be solved.  This flexibility of constraint
imperative programming is especially useful when solving problems with
varying requirements in one project and is worth considering in the
design of new programming languages and systems.

%%% Local Variables: 
%%% mode: latex
%%% TeX-master: "da.tex"
%%% End: 

%% End of summary.tex.
